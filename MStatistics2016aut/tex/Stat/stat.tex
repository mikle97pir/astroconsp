%------------------------------------------------------------
% Description : Matstatistics 
% Author      : taxus-d <iliya.t@mail.ru>
% Created at  : Thu Jan 12 14:26:44 MSK 2017
%------------------------------------------------------------
\documentclass[12pt,timbord]{../../../notes}
\usepackage{silence}
\WarningFilter{latex}{Reference}
\graphicspath{{../../img/}}

\begin{document}

\paragraph{Случаные векторы}
\label{par:stat::randvec}


\begin{defn}[Случаная величина]\label{defn:prob::randvec::randscal}
  Случайной величиной назовём произвольное хорошее отображение $X\colon \Omega \to \R^n$.
  См. примечание про измеримость в \ref{par:prob::randscal}. Борелевские множества можно рассматривать 
  и в $\R^n$, как наименьшую сигма-алгебру, содержащую все полуоткрытые параллелепипеды.

  Так же можно считать, что случайный вектор~--- набор случайных величин.
\end{defn}


\begin{defn}[Функция распределения]\label{defn:prob::randvec::distrfun}
  \[
    F_X\colon \R^n\to [0;1] \colon\; F_X(x) = P(X^1 < x^1, \dotsc, X^n < x^n) 
  \]
  То есть вероятность попадания в параллелепипед, уходящий в бесконечность.
\end{defn}
\begin{rem}\label{rem:stat::randvec::distr}
Тут как раз используется, что борелевские множества <<прямоугольные>>.
\end{rem}


\begin{prop}\label{prop:prob::randscal::distrfun}
  Про $F(x)$ верно следущее:
  \begin{enumerate}
    \item $F$ не убывает по каждому аргументу.
    \item $\displaystyle\lim_{x_i\to -\infty}F(x) = 0$
    \item $\displaystyle\lim_{x\to +\infty}F(x) = 1$
    \item $\displaystyle\lim_{x\to x_0-0}F(x) = F(x_0)$ (по совокупности переменных)
      Это следует из непрерывности меры.
  \end{enumerate}
\end{prop}

тоже важно, так что отдельно

\begin{prop}\label{prop:stat::randvec::inc}
  Пусть $a^1 < b^1, \dotsc, a^n < b^n$, тогда работает формула включений и исключений
  \[
    F(b^1, \dotsc, b^n) - \sum_i F(b^1, \dotsc, a^i, \dotsc, b^n) + \dotsb + F(a^1, \dotsc, a^n) = 
  P(x \in [a^1, b^1)\times[a^n,b^n))
  \]
  По сути следствие формулки про вероятность объединения.
\end{prop}


\begin{defn}\label{defn:stat::randvec::disc}
  Векторная случайная величина называется дискретной, если
  \[
    \exists\,(\{a_i\mid a_i \in \R^n \} \sim \N)\colon \left( \sum_{i} P(X = a_i) = 1 \right)
  \]
  то есть 
  \[
    P(X\in B) = \sum_{\{i\mid a_i\in B\}} p_i, \; p_i = P(X = a_i)
  \]
\end{defn}
\begin{defn}\label{defn:stat::randvec::cony}
  Векторная случайная величина называется непрерывной, если
  \[
    \exists\,(f_X\colon B\to \R )\colon \left( P(X\in B) = \int_{B} f_X(x^1, \dotsc, x^n)\, \del
    x^1 \dotsm \del x^n \right)
  \]
\end{defn}
\begin{rem}\label{rem:stat::randvec::distfun }
  Для функций распределения:
  \[
    F(x^1, \dotsc, x^n) = \int_{-\infty}^{x^n}\dotsi\int_{-\infty}^{x^1} f_X(x^1, \dotsc, x^n)\,
    \del x^1 \dotsm \del x^n 
  \]
\end{rem}

\begin{prop}\label{prop:stat::randvec::ind}
  Пусть $X, Y$~--- независимы. Тогда $p_{X+Y}(x,y)=p_X(x) \cdot p_Y(y)$
\end{prop}
\begin{itlproof}
  По определению функции распределения
  \[
    F_{X,Y}(x,y) = \int_{-\infty}^x\int_{-\infty}^y p(x,y) \, \del x\del y
  \]
  Из независимости $X,Y$
  \[
    \begin{split}
      F_{X,Y} (x,y) &= P(X<x,Y<y) = P({\omega\mid X(\omega)\in (-\infty;x]}\cap {\omega\mid
        Y(\omega)\in (-\infty;y]} \\
        &= P({\omega\mid X(\omega)\in (-\infty;x]})\cdot
        P({\omega\mid Y(\omega)\in (-\infty;y]}) = P(X<x) \cdot P(Y<y) = F_X(x)\cdot F_Y(y)
      \end{split}
  \]
  А тогда из независимости подынтегральных функций
  \[
    F_X(x)\cdot F_Y(y) = \int_{-\infty}^x p_X(x) \, \del x \cdot \int_{-\infty}^y p_Y(y) \, \del y 
    = \int_{-\infty}^x\int_{-\infty}^y p_X(x)\cdot p_Y(y) \, \del x\del y
  \]
  А дальше можно заметить, что нам неважно по какому множеству интегрировать.
  \[
    \int_B (p(x,y) - p_X(x)\cdot p_Y(y)) \, \del x\del y = 0
  \]
  Здесь правда всё ломается на отсутствии непрерывности у $p$. Но если она есть, то дальше
  стандартное рассуждение в окрестности точки где не 0.
\end{itlproof}

\paragraph{Функция от случайного вектора}
\label{par:stat::funcrand}

\begin{defn}[Функция от случайного вектора]\label{defn:stat::funcrand::funcrand}
  $g \colon \R^n \to \R^m$. 
  \begin{itaux}
    Здесь нужно снова говорить про измеримость $g$~--- прообраз борелевского множества должен
    быть борелевским множеством. Иначе $g(X)$ может не получиться случайной величиной. Но всякие
    мерзкие отображения всё равно никому не нужны $\ddot\smile$.
  \end{itaux}
\end{defn}

\begin{prop}\label{prop:stat::funcrand::disc}
  Пусть $X$~--- дискретная случайная величина, $f$~--- обратима, $Y=g(X)$, $b_j= f(a_j)$.
  Тогда $P(Y^i=b^i_j) =P(X^i=a^i_j)$.
\end{prop}
\begin{itlproof}
  \[
  P(Y^i=b^i_j) = P(f(X^i)= f(a^i_j)) = P({\omega \mid f(X^i(\omega))=f(a^i_j)}) 
  \]
  Поскольку $f$~--- обратима, она биективна. Значит $f(X) = f(a_j) \Leftrightarrow X = a_j$.
  Собственно, всё.
\end{itlproof}
\begin{prop}\label{prop:stat::funcrand::cont}
  Пусть $X$~--- непрерывная случайная величина, $f$~--- обратима, $Y=g(X), f^{-1} = g$.
  Тогда $p_y(y) = p_X(g(y)) \left| \frac{\del g}{\del y} \right|$ .
\end{prop}
\begin{itlproof}
  Пусть $D=f(B)$. Тогда $P(Y \in D) = P(X \in B)$ опять-таки в силу биективности $f$. Ну, ничего
  нового туда попасть не может и у всего есть  прообраз. Так что (здесь будем рисовать один значок
  интеграла и дифференциала из экономии размера пдф-ки, хотя в этом замечании данных может и больше)
  \begin{align*}
    \int_D p_Y(y) \, \del y  = \int_B  p_X(x) \del x 
    = \int_D  p_X(g(y)) \left| \frac{\del g}{\del y} \right|  
  \end{align*}
  Якобиан тут под модулем, так как множество неориентированное. Я верю, что нам ещё про это
  расскажут на матане.
\end{itlproof}

\paragraph{Матожидание и дисперсия суммы случайных величин}
\label{par:stat::randsum}

\begin{prop}\label{prop:stat::randsum::discsum}
  Пусть $X,Y,n\in(A\sim\N)$~--- две дискретные независимые случайные величины. Тогда 
  \[
    P(X+Y=n) = \sum_{k\in A} P(X=n-k)\cdot P(Y=k)
  \]
\end{prop}
\begin{itlproof}
  Из формулы полной вероятности ($Y=k, k\in A$ правда полная группа)
  \[
    P(X+Y =n) = \sum_{k\in A} P(X+Y =n \mid Y =k)\cdot P(Y = k) 
    = \sum_{k\in A} P(X= n-k\mid Y=k)\cdot P(Y=k)
  \]
  А вот тут уже поможет независимость $X,Y$.
  \[
    \cdots = P(X=n-k)\cdot P(Y=k)
  \]
\end{itlproof}

\begin{prop}\label{prop:stat::randsum::contsum}
  Пусть $X_1,X_2,n\in\R$~--- две непрерывные независимые случайные величины. Тогда 
  \[
    p_{X_1+X_2}(y) = \int_{\R} p_1(y-t) p_2(t) \,\del t
  \]
\end{prop}
\begin{itlproof}
  Пусть $Y = X_1 + X_2$. Тут видно, что нет биекции, придется руками что-то делать.
  \[
    F_Y(y) = \displaystyle\iint_{x_1+x_2 < y} p(x_1, x_2) \, \del x_1 \del x_2
    = \int_{-\infty}^\infty \del x_1\, \int_{-\infty}^{y-x_1} p(x_1, x_2) \, \del x_2 
    = \int_{-\infty}^\infty \del x_1 \, \int_{-\infty}^y p(x_1, u - x_1) \, \del u
  \]
  Переменные независимы, так что можно поменять местами интегралы (ещё же по области интегрируем,
  неважно как)\note{слишком много раз пользовались на физике, так что оставим на 4 семестр}
  А тогда, убирая внешний интеграл из определения функции распределения, получаем
  \[
    p_Y(y) = \int_{-\infty}^\infty p(t, y-t)\, \del t
  \]
  Поскольку $X_1, X_2$ независимы, то $p(t, y-t) = p_1(t)\cdot p_2(y-t)$. Нумерация никого не
  интересует, так что
  \[
    p_Y(y) =  \int_{-\infty}^\infty p_1(y-t)\cdot p_2(t)\, \del t
  \]
\end{itlproof}

Теперь можно перейти и к содержанию билета
\begin{prop}\label{prop:stat::randsum::exp}
  $\Exp \left(\sum_i X_i\right) = \sum_i \Exp X_i $. Да и вообще оно линейно.
\end{prop}
\begin{itlproof}
  Пусть $f(X,Y)= X+Y$
  \[
    \begin{split}
      \Exp (X+Y) &= \intR f(x,y) p(x,y)\, \del x\del y 
      = \intR x p(x,y)\, \del x \del y + \intR y p(x,y)\, \del x \del y \\
      &= \intR x 
      \left(\intR p(x,y) \del y\right)\, \del x + \intR y \left(\intR p(x,y) \del x\right)\,\del y \\
      &= \Exp X + \Exp Y
    \end{split}
  \]
  Покажем, что $\displaystyle\intR p(x,y) \,\del y = p_X(x)$
  \[
    \begin{split}
      \int_{-\infty}^x \left(\intR p(x,y) \,\del x\right) \, \del y &= F(x, +\infty) = 
      P(X<x, Y<+\infty) = P({\omega\mid X(\omega)\in (-\infty, x], Y\in \R\cup\{+\infty\}}) \\
      &= P(X<x) = F_X(x) = \int_{-\infty}^x p_X(x) \, \del x
    \end{split}
  \]
  Опять-таки интервал можно сжать как угодно, правда снова проблемы с непрерывностью.

  Часть про константу слишком очевидна, не будем её доказывать. 
\end{itlproof}

\begin{prop}[Дисперсия суммы]\label{prop:stat::randsum::var}
  $\Var \left(\sum_i X_i\right) = \sum_i \Var X_i$
\end{prop}
\begin{itlproof}
  Сначала заметим, что $\Var X = \Exp (X- \Exp X)^2$, $\Exp (X - \Exp X) = \Exp X - \Exp X = 0$
  \[
    \begin{split}
      \Var (X+Y) &= \Exp \bigl(X+Y - M(X+Y)\bigr)^2 = \Exp \bigl((X-\Exp X) +(Y-\Exp Y) \bigr)^2 \\
                 &= \Exp (X-\Exp X)^2 + \Exp (Y-\Exp Y)^2 + 2 \Exp (X-\Exp X) \Exp (Y - \Exp Y)\\ 
                 &= \Var X + \Var Y
    \end{split}
  \]
\end{itlproof}

\begin{prop}\label{prop:stat::randsum::expmul}
  Если $X,Y$~--- независимы, то $\Exp XY = \Exp X \Exp Y$
\end{prop}
\begin{itlproof}
  
\end{itlproof}

\paragraph{Матожидание функции случайной величины}
\label{par:stat::expfun}

\begin{defn}[$\ddot\sim$]\label{defn:stat::expfun::expfun}
  Пусть $f(X)$~--- функция от случаной величины. Тогда 
  $\displaystyle \Exp f(X) = \intR f(x) p(x)\, \del x$. В случае чего он многомерный, просто
  прикидывается. Существует, если есть абсолютная сходимость.
\end{defn}
\begin{rem*}
  я ещё подумаю, может это всё же утверждение.
\end{rem*}
\begin{prop}\label{prop:stat::expfun::lin}
  Матожидание функции линейно
\end{prop}
\begin{prop}\label{prop:stat::expfun::mul}
  Если $X,Y$~--- независимы, то $\Exp f_1(X_1)f_2(X_2) = \Exp f_1(X_1) \Exp f_2(X_2)$
\end{prop}


\paragraph{Неравенство Шварца}
\label{par:stat::shwartz}

\begin{prop}\label{prop:stat::shwartz}
  $(\Exp  XY)^2 \leqslant \Exp X^2 \Exp Y^2$
\end{prop}
\begin{itlproof}
  $\Exp (X+ tY)^2 = t^2\,\Exp Y^2 + 2 t \Exp XY + \Exp X^2 \geqslant 0$ из свойств матожидания. Ну
  там и подынтегральная функция положительна. Тогда квадратное уравнение в правой части может
  иметть не более одного корня.
  \[
    (2 \Exp XY)^2 - 4 \Exp X^2 \Exp Y^2 \leqslant 0 \Leftrightarrow 
    (\Exp  XY)^2 \leqslant \Exp X^2 \Exp Y^2
  \]
\end{itlproof}

\paragraph{Характеристическая функция суммы случайных величин}
\label{par:stat::charfunsum}

\begin{prop}\label{prop:stat::charfunsum}
  Пусть $X,Y$~--- независимые случайные величины. Тогда 
  \[
    \Phi_{X+Y}(t) = \Phi_X(t) \cdot \Phi_Y(t)
  \]
\end{prop}
\begin{itlproof}
  Из~\ref{prop:stat::charfunsum::sum}  
  \[
    \Phi_{X+Y}(t) = \Exp e^{itX} e^{itY} = \Exp e^{itX} \cdot \Exp e^{itY} = \Phi_{X}(t) \cdot
    \Phi_Y (t)
  \]
\end{itlproof}
\begin{cor}\label{conj:stat::charfunsum::sumn}
  Если все величины одинаково распределены, то $\Phi_{X_1 + \dotsb + X_n}(t) =
  \bigl(\Phi(t)\bigr)^n$, \[
    p_{X_1 + \dotsb + X_n} = \frac{1}{2\pi} \intR \bigl(\Phi(t)\bigr)^n \, \del t 
  \]
\end{cor}

\paragraph{Суммирование большого числа случайных величин}
\label{par:stat::randlimsum}
\flame\underdev\sour

\begin{thrm}[ЦПТ Линдберга-Леви-Агекяна \sour]\label{prop:stat::randlimsum::sum}
  Пусть $X_1, \dotsc, X_n$~--- независимые одинаково распределённые случайные величины. Пусть к
  тому же $S_n = X_1, \dotsc, X_n$, $0 < \Var X_k < \infty$.
  Пусть $\Exp X_k = a, \Var X_k = \sigma$. Тогда при $n\to \infty$ $Z_n \sim N(0,1)$,
  в вариации из Агекяна $S_n \sim N(na,n\sigma^2)$
\end{thrm}
\begin{itlproof}
  Пусть $\Exp X_k = a, \Var X_k = \sigma^2$.
  Рассмотрим характеристическую функцию $\Phi(t) = \Exp e^{itX_k}$. Введём замену (которая
  z-преобразование.):
  \[
    z_n = \frac{S_n - a n}{\sigma \sqrt n}
  \]
  Давайте ещё немного схитрим и положим $X_k \gets X_k- a$. А то потом будет много возни с бедным
  $a$. При этом $z_n = \frac{S_n}{\sigma \sqrt n}$
  Тогда
  \[
    \Phi_{z_n} (t) = \Exp \left(e^{\frac{itS_n}{\sigma\sqrt n}} \right)=
    \left(\Phi\left(\frac{t}{\sigma \sqrt n} \right)\right)^n 
  \]
  А характеристическая функция дифференцируема дважды из существования дисперсии.
  \begin{align*}
    \Phi'(0) &= 0 \\
    \Phi''(0) &= - \sigma^2 \\
    \Phi\left(\frac{t}{\sigma \sqrt n} \right) 
    &= \Phi(0) + \Phi'(0) \frac{t}{\sigma \sqrt n} + \Phi''(0)\, \frac{t^2}{2\sigma^2 n} 
      + o\left(\frac{1}{n}\right)
    = 1  - \frac{1}{2}\,\frac{t^2}{n} +  o\left(\frac{1}{n}\right) \\
  \end{align*}
  А при $n\to \infty$ \[
    \Phi\left(\frac{t}{\sigma \sqrt n} \right)^n 
    = \left(1  - \frac{1}{2}\,\frac{t^2}{n} +  o\left(\frac{1}{n}\right)\right) ^n
    \rightrightarrows e^{-\lfrac{t^2}{2}}
  \]
  Здесь сходимость есть на любом конечном интервале, но вот про всю прямую этого уже не скажешь.
  Так что снова поднимается вопрос какой теоремой о непрерывном соответствии пользоваться.
  Но если ей воспользоваться (тут по-тихому применили обратное преобразование Фурье), то
  \begin{align*}
    p_{z_n}&= \frac{1}{2\pi} \intR \exp{\frac{- s^2 + s^2 - 2its - t^2}{2}}  =
    \frac{e^{-s^2/2}}{\sqrt{2}\pi } 
    \intR \exp\left(-\left(\underbrace{\frac{t+is}{\sqrt 2}}_\eta \right)^2\right) \del \eta 
    =\frac{e^{-s^2/2}}{\sqrt{2\pi} } \\
    F_{Z_n} (x) &= \int_{-\infty}^x \frac{1}{\sqrt{2\pi}} e^{-s^2/2} \, \del s 
  \end{align*}

  Дальше~--- вариация из Агекяна.
  Используя утверждение \ref{prop:stat::funcrand::cont} про замену переменной как раз получаем
  нормальное распределение. Только тут нужно поменять в процессе $\sigma$
  \begin{align*}
    Z_n &= \frac{S_n}{\sigma \sqrt n} \\
    F_{S_n} (x) &= \int_{-\infty}^x \frac{1}{\sqrt{2\pi}} \frac{1}{\sigma\sqrt n} 
    e^{-\frac{u^2}{2\sigma^2 n} } \, \del u
  \intertext{Вернёмся обратно к ненулевому $a$ }
    S_n &\gets S_n - n a \\
    F_{S_n} &= \frac{1}{\sqrt{2\pi}} \frac{1}{\sigma\sqrt n}\int_{-\infty}^x  
    e^{-\frac{(u-n a)^2}{2\sigma^2 n} } \, \del u \\
    S_n \sim N (na, n \sigma^2) (n \to \infty)
  \end{align*}
\end{itlproof}


\paragraph{Центральная предельная теорема}
\label{par:stat::cpt}
\begin{thrm}[ЦПТ Ляпунова\sour]\label{thrm:stat::cpt::lyap}
  Пусть $ \{ X_k\}$~--- независимые случаные величины (тут нет одинаковости расределений!).
  Введём гору обозначений:
  \begin{align*}
    S_n &= \sum_i x_i \\
    a_k &= \Exp X_k & \sigma^2_k &= \Var x_k & \gamma_k &= \Exp |X_k - a_k|^3 \\
    A_n &= \sum_{k=1}^n a_k & B_n^2 &= \sum_{k=1}^n \sigma_k^2 & C_n &= \sum_{k=1}^n \gamma_k
  \end{align*}
  Тогда \[
    \frac{C_N}{B_n^3} \xrightarrow[n\to\infty]{} 0  \Rightarrow \frac{S_n - A_n}{B_n}  \xrightarrow[n\to\infty]{d} \mathcal N (0,1)
  \]
\end{thrm}
\begin{rem*}
  Тут какая-то жесть. Она мало где формулируется и нигде не доказывается. Что-то есть
  тут:\cite{msu}, а здесь~\cite{chernova1} так другую теорему обозвали

  \flame
\end{rem*}

\paragraph{Обобщённая теорема Муавра-Лапласа}
\label{par:stat::genmuavr}

\begin{defn}\label{defn:stat::genmuavr::chi}
  Пусть $X_1, \dotsc, X_n \sim \mathcal N (0,1)$ и независимы. Тогда говорят, что случайная
  величина $\displaystyle \chi_n^2 = \sum_{k=1}^n X_k^2$ имеет распределение $\chi^2$ c $n$
  степенями свободы.
\end{defn}
\begin{prop}\label{prop:stat::genmuavr::chi}
  $\displaystyle p_{\chi_b}(z) = \frac{1}{2^{n/2}\cdot \Gamma (\frac{n}{2} )} z^{n/2 -1} e^{-z/2} $
\end{prop}
\begin{itlproof}
  Характеристическая функция $\chi$ может быть найдена из~\ref{conj:stat::charfunsum::sumn}
  Найдём сначала характеристическую функцию $X_k^2$. Для этого было бы недурно найти плотность
  соответвующего распределения
  \[
    \begin{split}
      P(y < X^2 < y+ \del y) &= P(\sqrt y < X < \sqrt{y + \del y})+ P(-\sqrt y>X >-\sqrt{y+\del y}) 
      = \frac{2}{\sqrt{2\pi}}  \int_{\sqrt y}^{\sqrt{y+\del y}} e^{-u^2/2} \, \del u \\
      &= \sqrt{\frac{2}{\pi} } \frac{e^{-y/2}}{2\sqrt y} 
    \end{split}
  \]
  А теперь можно и фурье-образ найти
  \[
    \intR\frac{e^{-y/2}}{2\sqrt y}\del y = \int_{0}^{+\infty}\frac{e^{-y/2}}{2\sqrt y}\del y
    = \int_{0}^{+\infty} \exp\left(\frac{-\eta^2(1-2it)}{2}\right)\,\del\eta
    = (1-2it)^{-1/2} \, \sqrt{\frac\pi 2}
  \]
  Вспоминая про коэффициент получим $\Phi_k(t) = (1-2it)^{-1/2}$.
  \[
    \Phi_\chi(t) = \left(\Phi_k(t)\right)^n =  (1-2it)^{-n/2}
  \]
  Тогда \[
    p_\chi(z) = \frac{1}{2\pi} \intR (1-2it)^{-n/2}\, e^{-itz}\del t
  \]
  Дальше немного жесть
  \begin{align*}
    \intR (1-2it)^{-n/2}\, e^{-itz}\del t &= \intR e^{-l} \left(1-2\, \frac{l}{z} \right)^{-n/2}
    \frac{1}{iz} \, \del l 
    = 2\cdot\frac{z^{n/2-1} e^{-z/2}}{2^{n/2}} \int_{0}^{\infty} s^{-n/2} e^{-s} \del s \\
    &= 2\frac{z^{n/2-1} e^{-z/2}}{2^{n/2}} \cdot \Gamma\left(1-\frac{n}{2} \right)
  \end{align*}
  Из правила отражения для $\Gamma$-функции 
  $\Gamma(1-n/2)\, \Gamma(n/2) = \frac{\pi}{\sin\frac{\pi n}{2} } $. А это почти что надо.
  \underdev
  там надо интеграл поаккуратнее брать.
\end{itlproof}

\begin{thrm}[Обобщённая теорема Муавра-Лапласа]\label{thrm:stat::genmuavr}
  Пусть $X_1, \dotsc, X_n$~--- независимые случайные величины с дискретным распределением:
  \[
    X_k \colon
    \begin{array}{c|c|c}
      1 & \cdots & r \\ \hline
      p_1 & \cdots & p_r
    \end{array}
  \]
  Рассмотрим $\nu_k = \# \{ 1 \leqslant i \leqslant n \mid X_i =k\},\; 1\leqslant k \leqslant r$.
  Тогда 
  \[
    \sum_{k=1}^r \left( \frac{\nu_k - n p_k}{\sqrt{n p_k}}  \right) \xrightarrow{d} \chi_{r-1}^2
  \]
\end{thrm}

\paragraph{Метод моментов}
\label{par:stat::mom}

Здесь походу нужно все статистические определения в одном параграфе \flame

\subparagraph{Вводные слова}
В отличие от теорвера, матстатистике неизвестно распределение случайной величины. И нужно
придумать, как его восстановить по конкретной реализации. Пусть $X$~--- та самая величина,
про распределение которой очень хочется узнать

Главная героиня матстатистики~--- выборка
\begin{defn}\label{defn:stat::mom::chosen}
  Выборка объёма $n$~--- 
  \begin{enumerate}
    \item $n$ независимых случайных величин, распределённых так же, как и $X$
    \item набор чисел $X_i(\omega)$, $\omega \in \Omega$~--- какой-то исход. Ещё называется
      реализацией выборки.
  \end{enumerate}
  Собственно, первое определение~--- это до испытания, а второе~--- уже после.
\end{defn}

\subparagraph{Основные задачи}
Пусть $X_1, \dotsc, X_n \sim F(x,\theta)$, $\theta \in \Theta \subset \R^d$~--- множество
параметров.

\begin{enumerate}
  \item Оценивание параметров:
    \begin{itemize}
      \item Точечные оценки: $\hat\theta = T(X_1, \dotsc, X_n)$
      \item Доверительные интервалы: $P_\alpha (T_1 < \theta < T_2) = \alpha$
    \end{itemize}
  \item Проверка гипотез\par
    Пусть $\Theta = \Theta_0 \cup \Theta_1$. А мы хотим узнать чему принадлежит $\theta$. 
    \begin{description}
      \item[$H_0$:] $\theta\in \Theta_0$~--- основная гипотеза 
      \item[$H_1$:] $\theta\in \Theta_1$~--- альтернативная гипотеза 
    \end{description}
\end{enumerate}

\subparagraph{Выборочные характеристики}
Выберем реализацию случайной величины (выборку во втором смысле) $X_1, \dotsc, X_n$.
Рассмотрим новую случайную величину:\[
  \widetilde{X} \colon 
  \begin{array}{c|c|c}
    X_1 & \cdots & X_n \\ \hline
    \frac{1}{n} & \cdots & \frac{1}{n}  
  \end{array}
\]

Ну а дальше все выборочные характеристики определяются уже для этой случайной величины.
Напомним следующее
\begin{defn}[Индикатор]\label{defn:stat::mom::ind}
  $\displaystyle I_A(X) = \begin{cases}
    1, &X\in A\\
    0, & X \not\in A
  \end{cases}, 
  I(X < x) = \begin{cases}
    1, &X < x\\
    0, & X \geqslant x
  \end{cases}
  $
\end{defn}
\begin{defn}\label{defn:stat::mom::varser}
  Если $X_1, \dotsc, X_n$ можно упорядочить, то $X_{(1)} \leqslant \cdots \leqslant X_{(n)}$
  называется вариационным рядом.
\end{defn}
{\def\arraystretch{1.5}
\begin{table}[h] \underdev
  \begin{tabular}{|c|c|c|c|}
    \hline
    & \bf Генеральная совокупность & \bf Выборка &  \\ \hline
    Матожидание & $\Exp X$ & $\overline{X} = \frac{1}{n} \sum_k X_k$ & Выборочное среднее\\
    Дисперсия & $\Var X$ & $S^2 = \frac{1}{n} \sum_k (X_k - \overline X)$ & Выборочная дисперсия\\
    Момент порядка $l$ & $\Exp X^k$ & $m_l = \frac{1}{n} \sum_k X_k^l$ & \\

    Ковариация & $\Exp (X-\Exp X) (Y-\Exp Y)$ & $\frac{1}{n} \sum_k (X_k - \overline X) (Y_k -
    \overline Y)$ & \\
    Ассиметрия ($\gamma_3$) & $\Exp (X- \Exp X)^3/\sigma^3$  
                                      & $\frac 1 n \sum (X- \overline X)^3/S^3$ & \\
    Эксцесс ($\gamma_4$) & $\frac{\Exp (X - \Exp X)^4}{\sigma^4} - 3 $& & \\
    Функция распределения & & $\frac 1 n \sum_k I(X_k < y)$ & \\
    Квантиль порядка $p\in(0;1)$ & $\sup \{x \mid F(x) \leqslant p \}$ & $X_{([pn]+1)}$ 
                                                & член вариационного ряда\\
    \hline
  \end{tabular}
\end{table}}

\subparagraph{\plholdev{Большой кусок про свойства оценок} \flame}
\subparagraph{Метод моментов}

\begin{defn}\label{defn:stat:mom:method}

  Пусть $F(x,\theta)$~--- семейство распределений, $m(x) = \Exp g(x)$~--- какой-то момент этого
  распределения. Пусть известно, что $h(\theta) = m(x)$. Тогда собственно сам метод состоит в том,
  чтобы оценить $\theta$ как решение уравнения выше.
  \[
    \hat \theta = h^{-1} (m(x))
  \]
  В случае чего там векторы, но особо не страшно.

\end{defn}

\begin{exmp}\label{exmp:stat::mom::norm}
  \plholdev{примеры про непрерывные распределения}
\end{exmp}

\paragraph{Метод максимального правдоподобия}
\label{par:stat::maklike}
надо как-то уметь получать хорошие  оценки, для этого этого будет сейчас введён Метод.этого будет
сейчас введён Метод.

\begin{defn}\label{defn:stat::maxlike::dens}
  За $f_\theta (x) $ обозначим плотность функции распределения $F(x,\theta)$ в точке $x$ в случае
  непрерывного распределения и $P(X=x)$ в случае дискретного
\end{defn}
\begin{defn}\label{defn:stat::maxlike::fun}
  $\displaystyle L(\theta):= \prod_{k=1}^n f_\theta(X_k)$~--- функция правдоподобия.
  Ещё иногда берут её логарифм.
\end{defn}

\begin{defn}\label{defn:stat::maxlike::method}
  Сам метод состоит в следующем:
  \[
    \hat \theta \colon L(\hat \theta) = \max_\theta L (\theta)
  \]
\end{defn}

\plholdev{гора примеров}
\end{document}
% vim:wrapmargin=3
