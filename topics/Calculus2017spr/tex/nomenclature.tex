%------------------------------------------------------------
% Description : Стандартные обозначения 
% Author      : taxus-d <iliya.t@mail.ru>
% Created at  : Sat Jan 14 17:39:43 MSK 2017
%------------------------------------------------------------
\documentclass[12pt,timbord]{longnotes}
\usepackage{tmath}
\usepackage{cussymb}
\usepackage{silence}
\WarningFilter{latex}{Reference}
\graphicspath{{../../img/}}


\begin{document}

\section*{Обозначения с лекции}
\begin{description}
  \item[$a:= b$]~--- определение $a$.
  \item[$\displaystyle \bigsqcup_k A_k$]~--- объединение дизъюнктных множеств.
  \item[$\alg$]~--- Алгебра множеств
  \item[$\ov-{A}$]~--- Замыкание $A$.
  \item[$A^c$]~--- $X \setminus  A$.
\end{description}

\section*{Нестандартные обозначения}

\begin{description}
  \item[\underdev]~--- ещё правится. Впрочем, относится почти ко всему.
  \item[$\square\cdots\blacksquare$]~--- начало и конец доказательства теоремы
  \item[$\blacktriangledown\cdots\blacktriangle$]~--- начало и конец доказательства более мелкого 
    утверждения
  \item[\sour]~--- кривоватая формулировка
  \item[\flame]~--- набирающему зело не нравится билет
  \item[\plholdev{что-то}]~--- тут будет \texttt{что-то}, но попозже
  \item[$a\intrng b$]~--- $[a;b]\cap \Z$
  \item[$\equiv$]~--- штуки эквивалентны. Часто используется в этом смысле в
    определениях, когда вводится два разных обозначения одного и того же
    объекта.
  \item[$\that$]~--- В кванторах, <<верно, что>>
  \item[$\alg_\sigma$]~--- Сигма-алгебра множеств
  \item[$f \colon A \leftrightarrow B$]~---биекция
\end{description}
\end{document}

