%------------------------------------------------------------
% Description : 
% Author      : Iliya Tikhonenko <iliya.t@mail.ru>
% Created at  : Sat Jun  3 21:54:32 MSK 2017
%------------------------------------------------------------
\documentclass[timbord]{longnotes}
\usepackage{tmath}
\usepackage{cussymb}
\graphicspath{{img/}}

\begin{document}

\begin{description}
%   \item[1] Ячейки --- конечные объединения?
%   \item[4] Там был предел разности. Почему так можно, может предела нет вовсе?
  \item[5] Забиваем на внешнюю мему?
  \item[5] Надо ли конструктивно доказывать, что дизъюнктное разбиение существует?                     -- нет
  \item[5] На какой алгебре задан объём. Точнее, как формулируется теорема про счётную аддитивность?   -- как продолжать меру с просто алгебры
  \item[5] Забиваем что там объединения (в теореме про счётную аддитивность)?                          -- да
  \item[5] $\lebesgue \supset \borel$ очевидно?                                                        -- да 
  \item[5] Регулярность: бесконечный случай ощутимо хардовее.                                          -- нет, сигма-конечность поможет
  \item[8] Всякие вариации --- сюда?                                                                   -- нет, забить на вариации
  \item[17] Но... мы же доказали лишь для конечных мер? А как здесь быть.?                             -- | 
  \item[17] Выбор пределов интегрирования, какое ему обоснование?                                      -- | после перехода с минус функции
  \item[17] Не, я могу сказать, что тут область конечная, а потом предел посчитать                     -- | все множества конечны
    Мы так даже умеем, по счётной аддитивности интеграла. Так как рассказывать-то?                     -- | мера $[0< a< f< b] = 0$
  \item[19] Так, серьёзно, куда пихать <<почти всюду>>?                                                -- забить
  \item[19] А измеримость в геометрическом смысле не доказана.. И вообще ничего не сказаго             -- забить
    про произведение мер.
  \item[21-22] беда с измеримостью                                                                     -- --//--
  \item[26-27] Переформулировка всяких дифф. на интеграл по мере                                       -- --//-- 
  \item[27] нет док-в?                                                                                 -- --//-- 
  \item[27] область непрерывности $f_t'$, там походу нужна конечность меры.                            -- для интервала
  \item[27] Там таки для почти всех или нет (в 3)? 
  \item[?] Свёртка                                                                                     -- нету её
  \item[35] Скалярное произведение на $T_xM$.                                                          -- ?
  \item[35] Многообразия, говорить ли .                                                                -- можн
  \item[35] Мы же вообще все про площади определяли для простых многообразий.                          -- +
    Что делать со сферой? Резать и говорить что мера ноль?
  \item[41] А что делать с ориентацией кривой? Там проблемы с пересечениями.                           -- разные опр для кривой и пов-ти
    Как это обощать на поверхности? Если это многообразия , то пересечений же нет?
  \item[41] А поверхность --- связна?                                                                  -- да, иначе не простая
  \item[42] Класс гладкости форм.                                                                      -- ?
  \item[44] Связность $D$?                                                                             -- да, иначе не простая 
  \item[52] Полнота тригонометрической системы.                                                        -- забить
  \item[54] В теореме Дини $L_1$, или $L_2$ ?                                                          -- $L_1$
\end{description}
\end{document}
% vim:tw=100 cc=100

