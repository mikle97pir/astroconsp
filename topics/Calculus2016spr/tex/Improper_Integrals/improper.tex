\documentclass[12pt]{../../notes}
\usepackage{silence}
\WarningFilter{latex}{Reference}
\graphicspath{{../../img/}}

\begin{document}
\paragraph{Общие свойства несобственного интеграла}

\begin{defn}\label{defn:impint}
  Пусть $f \in C\big([a;b)\big)$, $b\in \overline{\R}$. 
    Тогда 
    \[
      \int_a^{\to b} f := \lim_{t \to b-0} \int_a^t f
    \]
  Аналогичным образом можно поступить и для нижнего предела (или обоих сразу)
\end{defn}

\begin{defn}\label{defn:intconv}
  $\displaystyle \int_a^{\to b}$ называется \emph{сходящимся}, если он существует и конечен.
\end{defn}
Свойства несобственного интеграла:
\begin{enumerate}
  \item $f \in C \big([a;b]\big)$, $a, b \in \R$. Тогда 
    \[
      \int_a^{\to b} f = \int_{\to a}^b f = \int_a^b f
    \]
  \item $f \in C \big([a;b)\big)$, $c \in (a;b)$ Тогда 
    \[
      \int_a^{\to b} f = \int_a^c f + \int_c^{\to b} f
    \]
    Второе слагаемое ещё называют остатком.
    \[
      \int_a^{\to b} f \text{ сходится } \Leftrightarrow \int_c^{\to b} f \text{ сходится }
    \]
  \item $f \in C \big([a;b)\big)$, $c \in (a;b)$. Тогда 
    \[
      \int_a^{\to b} f \conv \Rightarrow \int_c^{\to b} f \xrightarrow[c\to b]{} 0
    \]
  \item $f,g \in C \big([a;b)\big)$. Тогда 
    \[
      \int_a^{\to b} (f+g) = \int_a^{\to b} f + \int_a^{\to b} g
    \]
  \item $f \in C \big([a;b)\big)$, $c\in \R$
    \[
      \int_a^{\to b} c f = f \int_a^{\to b} f 
    \]
\end{enumerate}
\begin{exmp}
  \[
    \int_1^{+\infty} = \begin{cases} \frac{1}{p-1}, & p > 1 \\ +\infty, & p \leqslant 1 \end{cases}
  \]
\end{exmp}

\paragraph{Признак Больцано-Коши сходимости интеграла}

Этот кусочек не сильно нужен, так что он будет {\footnotesize таким шрифтом}
\begin{footnotesize}

\begin{defn}
  Пусть $f: I \to \R$, $c\in \overline{\R}$~--- точка сгущения. Тогда $f$ сходится в себе при $x\to c$
  $\Leftrightarrow$
  \[
    \forall\, \varepsilon > 0 \;\: \exists\, V(c) \colon \; \forall\,x', x'' \in \overset{\circ} V 
    \;\: |f(x') - f(x'')| < \varepsilon
  \]
\end{defn}

\begin{thrm}[Теорема Больцано-Коши для функций]\label{thrm:bkfun}
  $f$ сходится в себе при $x\to c$ $\Leftrightarrow$ $\exists\,\lim_c f = M \in \R$.
\end{thrm}
\begin{ittproof}
  \begin{description}
    \item[\circlearound{$\Leftarrow$}]  Рассмотрим произвольный $\varepsilon > 0$. Тогда из условия конечности
      предела $\lim_c f$
      \[
        \exists\, V(c) \colon \forall\, x \in V \;\: | f(x) - M | < \varepsilon/2
      \]
      Тогда для $x',x'' \in V(c)$ 
      \[
        |x' - x''| = |(x' - M) - (x''- M)| \leqslant |x' - M| + |x'' - M| < \varepsilon 
      \]
    \item[\circlearound{$\Rightarrow$}] Будем доказывать через аналогичную теорему для последовательностей
      и определение предела по Гейне. Рассмотрим произвольную $(x_n)\colon x_n \to c, x_n \neq c$, 
      произвольный $\varepsilon > 0$.
      Из условия равномерной сходимости
      \[
        \exists\, V(c) \colon \forall\, x', x'' \in \overset{\circ}{V} |f(x') - f(x'')| < \varepsilon
      \]
      $x_n \to c \Rightarrow$
      \[
        \exists\, N \colon \forall\, m,n > N \;\: x_m, x_n \in \overset{\circ}{V}
      \]
      Ну тогда $x' = x_n, x'' = x_m$ и по определению равномерной сходимости последовательностей 
      $y_n = f(x_n)$~--- фундаментальная. А значит, по теореме Больцано-Коши для последовательностей
      $\exists\, \lim y_n = L \in \R$.

      Хорошо, мы получили, что для каждой последовательности $(x_n)$ существует какой-то конечный предел 
      последовательности $y_n = f(x_n)$. Для определения по Гейне необходимо, чтобы они все были равны.

      Хорошо, пусть
      \begin{align*}
        &x_n' \to c  & y_n' = f(x_n') \to L' \\
        &x_n'' \to c  & y_n'' = f(x_n'') \to L''
      \end{align*}
      Тогда $\sphericalangle$ $z_n := (x_1', x_1'', x_2', x_2'', \dotsc)$. Она тоже $\to c$, 
      значит $\exists\, M \in \R\colon f(z_n) \to M$. 
      Но тогда у её подпоследовательностей разные пределы, что странно. 
  \end{description}
\end{ittproof}
\end{footnotesize}

\begin{thrm}\label{thrm:bkconvint}
  Пусть $f \in C\big([a;b)\big)$, $b\in \overline{\R}$. Тогда
  \[
    \int_a^b f \conv \Leftrightarrow \forall\, \varepsilon > 0 \;\: \exists\, V(b) 
    \colon \forall\,t',t''\in \overset{\circ}{V} \;\: \left|\int_{t'}^{t''} f\right| < \varepsilon
  \]
\end{thrm}

\paragraph{Свойства несобственного интеграла от положительных функций}
Пусть $f\in C\big([a;b)\big)$, $f \geqslant 0$, $F'=f$.
\begin{enumerate}
    \setcounter{enumi}{-1}
  \item $\displaystyle \int_a^{\to b} f \conv \Leftrightarrow F(x) \leqslant M \; \forall\, x$ 
  \item\label{stat:intcmp} Признак сравнения интегралов.\\
    Пусть $0 \leqslant f(x) \leqslant g(x)$. Тогда 
    \begin{align*}
      \int_a^{\to b} g \conv \Rightarrow \int_a^{\to b} f \conv \\ 
      \int_a^{\to b} f \noconv \Rightarrow \int_a^{\to b} g \noconv 
    \end{align*}
    Хватит и выполнения неравенства на $[c;b)$, $c \in (a;b)$, всё равно нужен только остаток.
    \item\label{stat:intcmplim} Второй признак сравнения.\\
      Пусть $\displaystyle \exists\, \lim_{t \to b-0} \frac{f(t)}{g(t)} = L$
      Тогда:
      \begin{enumerate}
        \item $\displaystyle L < +\infty 
          \Rightarrow \left(  \int_a^{\to b} g \conv \Rightarrow \int_a^{\to b} f \conv \right)$
        \item $\displaystyle L > 0 
          \Rightarrow \left(  \int_a^{\to b} f \conv \Rightarrow \int_a^{\to b} g \conv \right)$
      \end{enumerate}
      В частности, из эквивалентности следует одинаковый характер сходимости
\end{enumerate}
Попутно в этом же месте нормально определяли всякую тригонометрию, но, кажется, это не нужно в билете.

\paragraph{Абсолютная и условная сходимость интеграла}
\begin{defn}\label{defn:absintconv} 
  Пусть $\dint_a^{\to b}$~--- сходится. Тогда говорят, что он абсолютно сходится, 
  если $\dint_a^{\to b} |f| < + \infty$. В противном случае говорят, что интеграл сходится условно.
\end{defn}

\begin{thrm}\label{thrm:absconv2conv}
  Если интеграл абсолютно сходится, то он сходится.
\end{thrm}
\begin{ittproof}
  $\sphericalangle\,g  = |f| - f$. 
  Тогда $-|f| \leqslant f \leqslant |f| \Rightarrow 0 \leqslant g \leqslant 2|f|$. Теперь всё положительно
  и можно пользоваться признаками сравнения.
  Ещё можно воспользоваться признаком Больцано-Коши~\ref{thrm:bkconvint}~\cite{zorich}.
\end{ittproof}


\paragraph{Признаки Дирихле и Абеля}
\begin{thrm}[Признак сходимости Дирихле]\label{thrm:imprdirconv}
  Пусть $f,g \in C^1\big([a;b)\big)$\footnote{Вообще, хватило бы и просто непрерывности, 
  но для этого нужно доказывать ещё сколько-то интегральных неравенств о среднем 
  такого сорта : 
  \[
    \exists\, \xi \in [a;b]\colon \int_a^b (fg) = g(b)\int_a^\xi f + g(a) \int_\xi^b f
    \text{ \hspace{1cm} (см.~\cite[стр.~469]{zorich}) } \] } и 
  \begin{enumerate}
    \item $|f(x)|$\footnote{Вообще, мы формулировали это без модуля, 
      но для непрерывных функций эти условия эквивалентны} $\searrow 0$ при $x \to b-0$
    \item $\exists\,M: \left|\int_a^t g\right| \leqslant M \;\, \forall\, t $
  \end{enumerate}
  Тогда $\dint_a^{\to b} fg$~--- сходится
\end{thrm}
\begin{thrm}[Признак сходимости Абеля]\label{thrm:imprabelconv}
  Пусть $f,g \in C^1\big([a;b)\big)$ и 
    \begin{enumerate}
      \item $f(x)$ монотонна и ограничена
      \item $\int\limits_a^{\to b} g$ сходится
    \end{enumerate}
  Тогда $\dint_a^{\to b} fg$~--- сходится
\end{thrm}
\begin{ittproof}
  Доказывать всё надо через признак Больцано-Коши~\ref{thrm:bkconvint}. Сначала проинтегрировать по частям,
  а потом долго оценивать и доказывать что всё $\to 0$.
\end{ittproof}


\end{document}
