\documentclass{trnotes}
\setlayout{hardcopy}
\pagenumbering{gobble}
\usepackage{trmath}

\hypersetup{
  pdftitle = {Вопросы по матфизике},
  pdfauthor = {taxus},
  pdfsubject = {exams},
  pdfkeywords = {math-ph; exams; by-taxus}
}
\usepackage{titling}
\setlength{\droptitle}{-6em}
\posttitle{\par\end{center}}
\preauthor{\vspace{1ex}\hfil\large \lineskip 0.5em%
\begin{tabular}[t]{c}}
\postauthor{\end{tabular} $\qquad$}
\predate{\large}
\postdate{\hfil\par}

\usepackage{embedfile}
\embedfile{./\jobname.tex}

\title{ВОПРОСЫ ПО МАТФИЗИКЕ \rlap{$\;$}}
\author{Н.~М.~Ивочкина}
\date{09.06.2018}

\usepackage{enumitem}
\setlist[enumerate,1]{itemsep=-2.4pt, rightmargin=11em}


\newbox \sidetitlebox
\def\sidenotestitle#1{
  \setbox \sidetitlebox=\vbox to 0pt{%
    \setbox0=\hbox{\texttt{#1}}
    \hbox{\kern 2em \unhcopy0\kern 0.5ex}\vskip2pt\hrule
  }%
  \hfill\box \sidetitlebox%
}
\let \olditem = \item
\let \oldlabelenumi=\labelenumi%

\newcommand{\importantitem}{%
  \renewcommand{\labelenumi}{$\to$ \textbf{\oldlabelenumi}}%
  \olditem
}
\renewcommand{\item}{%
  \renewcommand{\labelenumi}{\oldlabelenumi}%
  \olditem
}
\begin{document}
\maketitle
\sidenotestitle{for notes}
\vspace{-2\baselineskip}%
\section*{Основные вопросы}
\begin{enumerate}
  \importantitem  Классификация линейных дифференциальных уравнений в частных производных
  второго порядка. Основные виды уравнении математической физики.  Каноническая
  форма операторов разного типа

\item  Характеристические поверхности линейных дифференциальных операторов и их
  уравнение. Приведение дифференциальных операторов к каноническому виду в
  случае двух переменных.

\importantitem  Волновое уравнение и постановка задач для него. Вывод формулы Даламбера
  решения задачи Коши для однородного волнового уравнении на плоскости.
\item  Принцип Дюамеля дли неоднородного волнового уравнения.

\item  Волновое уравнение в $\R^n$.  Энергетическое неравенство. Единственность
  решения задачи Коши.
\item  Волновое уравнение в $\R^3$.  Формула Кирхгофа и принцип Гюйгенса.
\item  Волновое уравнение на плоскости. Формула Пуассона. Диффузия волн.
\importantitem  Вывод уравнения теплопроводности. Постановка задач.
\item  Закон сохранения для однородного уравнения теплопроводности.
\item  Принцип максимума для однородного уравнения теплопроводности в
  ограниченной области.
\item  Принцип максимума для однородного уравнения теплопроводности в
  полупространстве.
\item Единственность решения задачи Коши и первой
  начально-краевой задачи.
\item Автомодельные решения однородного уравнения теплопроводности.
\item  Построение функции источника для уравнения теплопроводности при помощи
  автомодельного решения в одномерном случае.
\item  Построение функции источника для уравнения теплопроводности в случае
  произвольной размерности.
\item  Свойства функции источника.
\item  Формула Пуассона решения задачи Коши для однородного уравнения
  теплопроводности.
\item  Принцип Дюамеля для уравнения теплопроводности.
\importantitem  Фундаментальное решение уравнения Лапласа.
\importantitem  Представление $C^2$-функции в ограниченной области в виде суммы
  потенциалов.
\importantitem  Интегральное представление гармонической функции.
\importantitem  Теоремы о среднем значении для гармонических функций.
\item  Обратная теорема о среднем значении.
\importantitem  Свойства гармонических функции. Теорема единственности решения задачи
  Дирихле для уравнения Пуассона.
\item  Свойства объёмного потенциала.
\item  Формула Пуассона для гармонической функции в шаре.
\item  Решение задачи Дирихле для уравнения Лапласа в шаре.
\importantitem  Понятие о положительном самосопряжённом в $L^2$ линейном операторе, его
  собственные числа и собственные функции. Примеры.
\item  Собственные числа и собственные функции оператора Лапласа в
  прямоугольнике.
\item  Понятие о функциях Бесселя. Их свойства.
\item  Собственные числа и собственные функции оператора Лапласа в круге.
  Цилиндрические функции.
\importantitem  Вариационная природа гармонических функций. Интеграл Дирихле.
\item  Постановка задач вариационного исчисления. Примеры.
\importantitem  Первая вариация функционала и уравнения Эйлера-Лагранжа.
\item  Привести примеры вариационных задач, в которых выполнение уравнений
  Эйлера-Лагранжа не достаточно для обращения в ноль первой вариации
  функционала.
\importantitem  Достаточные условия экстремума функционала. Примеры.
\item  Естественные граничные условия и условие трансверсальности.
\item  Изопериметрическая задача. Экстремальное свойство наименьшего
  собственного числа оператора Лапласа (рассмотреть одномерный случай).
\importantitem  Понятие об обобщённых функциях. Дифференцирование обобщённых функций.
\item  Прямое произведение и свёртка обобщённых функций.
\item  Фундаментальные решения линейных дифференциальных операторов.
\item  Построение общего решения линейного обыкновенного
  дифференциального уравнения.
\end{enumerate}
\newpage
\sidenotestitle{for notes}
\vspace{-2\baselineskip}%
\section*{Дополнительные вопросы}
\begin{enumerate}
\item Описать качественные различия постановки и решения задач для линейных
  уравнений с операторами разного типа на примере канонических.
\item  Сравнить способы доказательства теорем единственности для уравнений с
  каноническими операторами.
\item  Автомодельные решения канонических операторов и их роль в построении
  теории.
\item  Сравнить фундаментальные решения различных задач математической физики.
\item  Найти явные представлении решений различных задач математической физики
  и сравнить их.
\item  Вывести достаточное условие экстремума вариационной задачи с условием
  трансверсальности.
\item  Найти фрагменты теории, в которых явно или неявно присутствует
  $\delta$-функция, и выявить связь с современным (в рамках теории обобщённых
  функций) её определением.
\item  Решение задачи Дирихле для уравнения Пуассона в шаре.
\end{enumerate}
\end{document}
