\documentclass{trnotes}
\usepackage{trmath}
\setlayout{hardcopy}
\setmainfont{PT Serif}[
  BoldFont = {Lato Semibold}
]
\setsansfont{Lato}
\setmathfont{STIX Two Math}
\setlist[enumerate]{itemsep=0pt,parsep=1.2pt}
\usepackage{embedfile}
\embedfile{./\jobname.tex}
\begin{document}
{\let\newpage\relax
  ${}$\par
  \vspace{-2.8em}\par 
\begin{titlepage}
  \centering
  \Large МЕТОДЫ ВЫЧИСЛЕНИЙ\par\vskip 1ex
  \normalsize Б.~А.~Самокиш\par\vskip 1ex
  \normalsize 10.01.2019\par
\end{titlepage}
}
\vskip 2ex

\begin{enumerate}
\item Краевая задача для обыкновенного дифференциального уравнения 2-го порядка: 
сведение к задаче Коши.
\item Метод дифференциальной прогонки для краевой задачи 2-го порядка.
\item Двухточечная краевая задача для системы уравнений 1-го порядка. Метод
 дифференциальной прогонки.
\item Ортогональная прогонка для систем уравнений 1-го порядка.
\item Разностный метод для краевой задачи 2-го порядка: составление
разностных уравнений.
\item Метод разностной прогонки.
\item Лемма об оценке для системы разностных уравнений.
\item Теорема о сходимости разностного метода для обыкновенной краевой задачи.
\item Жесткие системы обыкновенных дифференциальных уравнений. Простейшие
 методы. Понятие $A$-устойчивости.
\item Понятие $L$-устойчивости. Неявные методы Рунге-Кутта, общее понятие.
 Диагонально-неявные методы.
 \stepcounter{enumi}
\item Вопрос об устойчивости собственных чисел и собственных векторов
 при возмущении матрицы. Отрицательный пример.
\item Теорема Бауэра-Файка о возмущении собственных чисел симметричной матрицы.
\item Устойчивость собственных векторов при возмущении матрицы.
\item Степенной метод для отыскания старшего собственного числа.
\item Обратный степенной метод.
\item Двумерные вращения, их виды.
\item Лемма о правиле знаков при исключении.
\item Метод Гивенса.
\item Метод Якоби.
\item Две леммы о факторизации матрицы.
\item Теорема о сходимости итерированных подпространств.
\item Треугольно-степенной метод. Сходимость.
\item Ортогонально-степенной метод.
\item $LR$-алгоритм. Практическая реализация.
\item $QR$-алгоритм. Практическая реализация.
\item Интегральное уравнение 2-го рода. Метод замены ядра на вырожденное.
\item Метод квадратур для интегрального уравнения.
\item Вариационный принцип для ограниченного оператора. Метод Ритца для интегрального уравнения 2-го рода.
\item Интегральное уравнение 1-го рода. Понятие корректности. Некорректность уравнения 
1-го рода.
\item Условная корректность по Тихонову. Метод квазирешений.
\item Метод регуляризации для уравнения 1-го рода.Сходимость.
\item Вариационный принцип для уравнения с неограниченным оператором.
\item Метод Ритца. Сходимость.
\item Метод Ритца для обыкновенной краевой задачи. Вид энергетического пространства.
Естественные граничные условия.
\item ВРМ-1 для обыкновенной краевой задачи.
\item ВРМ-2 для обыкновенной краевой задачи.
\item Метод Ритца для эллиптического уравнения. Вид естественного граничного условия.
Вид энергетического пространства.
\item Разностный метод для общего уравнения теплопроводности. Явная схема.
\item Неявная схема для уравнения теплопроводности.
\item Явная схема для простейшего уравнения теплопроводности. Решение разностных
уравнений. Явление неустойчивости.
\item Общее определение устойчивости. Теорема об устойчивости и сходимости.
\item Разностные схемы для задач с начальными условиями. Дискретное 
преобразование Фурье.
\item Необходимое условие устойчивости по фон-Нейману.
\item Простейшие схемы для уравнения бегущей волны.
\item Схема Куранта-Рисса.
\item Явная схема для уравнения колебаний струны.
\item Явная и неявная схемы для двумерного уравнения теплопроводности.
\item Схема продольно-поперечной прогонки.
\item Задача Дирихле для двумерного эллиптического уравнения, составление разностных уравнений.
\item Итерационный метод  решения сеточной системы для эллиптического
доказательство сходимости.
\item Анализ сходимости простейшего итерационного метода для модельной задачи.
\item Метод оптимальной верхней релаксации, описание.
\end{enumerate}
\end{document}
