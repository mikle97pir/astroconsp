\documentclass{trlnotes}
\setlayout{hardcopy}
\usepackage{silence}
\WarningFilter{latex}{Reference}
\graphicspath{{../../img/}}

\begin{document}
    \paragraph{Пространства, отображения}
    Бесконечномерные пространства во многом похожи на конечномерные, но есть и различия. Приведём наглядный пример:

    \begin{thm}(Рисса)
        В бесконечномерном пространстве с нормой единичный замкнутый шар не компактен. \footnote{Верно и обратное утверждение: если в нормированном пространстве единичный замкнутый шар компактен, то оно конечномерно.}
        \begin{proof}
            Чтобы доказать, что что-то не компактно, нужно найти там последовательность, у которой нет сходящейся подпоследовательности. Здесь это нетрудно: подойдёт любой счётный ортнормированный набор векторов!

            Представьте себе: у вас есть $n$ единичных ортогональных друг другу векторов. Вы можете добавить ещё один, и ещё, и ещё... Конечно, в такой последовательности не выбрать сходящейся.
        \end{proof}
    \end{thm}

    В том, что касается линейных отображений, тоже есть тонкости. Мы знаем, что любое линейное отображение конечномерных пространств непрерывно и \ti{ограничено} (т.е. образ единичного замкнутого шара при нём ограничен). В бесконечномерном случае это не так! Однако выполняется такое утверждение:

    \begin{st}
        Для нормированных пространств непрерывность и ограниченность линейных отображений равносильны.
    \end{st}

    В реальности почти все интересные отображения ограничены. Да и у неограниченных слишком плохие свойства, поэтому в большинстве теорем ограниченность предполагается.

    \begin{rem}
        Будем все гильбертовы пространства считать \ti{сепарабельными}. Это по сути равносильно тому, что в них есть счётный базис.
    \end{rem}


    \paragraph{Пара фактов про гильбертовы пространства}

    \begin{rem}
        В бесконечномерных пространствах не все подпространства замкнуты; в частности, там бывают всюду плотные подпространства (как, например, многочлены в пространстве непрерывных функций). Об этом не стоит забывать.
    \end{rem}

    Оказывается, в гильбертовых пространствах ортогональные дополнения устроены почти так же, как и в конечномерной ситуации.

    \begin{st}\label{st:hilb-orth-compl}
        Ортогональное дополнение любого множества является замкнутым линейным подпространством. Если $A \subset H$~--- замкнутое линейное подпространство, то $H = A \oplus A^{\perp}$.
    \end{st}

    Этот факт используется для того, чтобы доказать теорему Рисса: линейные функционалы в гильбертовом пространстве~--- просто скалярные умножения на какие-то вектора.

    \begin{thm}[Рисс]\label{thm:rietz-repr}
        Пусть $H$~--- гильбертово пространство. Тогда каждый вектор $e$ задаёт ограниченный функционал $f_e \col \; H \to \C$ по правилу $x \mapsto (x, \, e)$, и каждый ограниченный функционал на $H$ есть $f_e$ для некоторого однозначно определённого вектора $e \in H$. Определённая этим биекция $H \to H^{*}$ есть сопряжённо-линейный изометрический изоморфизм нормированных пространств.
    \end{thm}

    \paragraph{Спектр оператора}

    Ещё одно различие, не столь наглядное, но очень важное, связано со \ti{спектром} оператора.

    \begin{de}
        Пусть $H$~--- гильбертово пространство, $A\col \; H \to H$~--- ограниченный оператор. \ti{Спектром} $A$ называют множество таких $\lambda \in \C$, что оператор $A - \lambda I$ необратим.
    \end{de}
    Понятие спектра тесно связано с собственными числами:
    \begin{de}
        Говорят, что $\lambda \in \C$~--- \ti{собственное число} оператора $A$, если есть такой вектор $v \in H$, что $Av = \lambda v$.
    \end{de}
    Собственные числа можно охарактеризовать в терминах оператора $A - \lambda I$:
    \begin{st}
        $\lambda$~--- собственное число $A$ тогда и только тогда, когда оператор $A - \lambda I$ не инъективен (то есть склеивает какие-то векторы в один).
        \begin{proof}
            Пусть $\lambda$~--- собственное число, $v$~--- собственный вектор. Тогда $(A - \lambda I)v = 0 = A0$, поэтому оператор не инъективен.

            Докажем в обратную сторону. Пусть оператор $A - \lambda I$ не инъективен. Тогда есть вектор из ядра~--- такой, что $(A - 
            \lambda I)v = 0$, т.е. $Av = \lambda v$.
        \end{proof}
    \end{st}

    Отсюда сразу следует утверждение:
    \begin{st}
        Для конечномерных пространств спектр и множество собственных чисел~--- одно и то же.
        \begin{proof}
            Как мы знаем,
            \[
                \text{необратимость} \eqv \text{неинъективность или несюръективность}.
            \]
            Но в конечномерном случае 
            \[
                \text{несюръективность} \so \text{неинъективность}.
            \]
            Это связано с тем, что несюръективный оператор понижает размерность пространства, что вынуждает его склеивать вектора.

            Поэтому необратимость либо сразу влечёт неинъективность, либо сначала влечёт несюръективность, а потом уже неинъективность. Отсюда
            \[
                \text{необратимость} \eqv \text{неинъективность},
            \]
            что и требовалось доказать.
        \end{proof}
    \end{st}

    В бесконечномерном случае всё не так. Из необратимости неинъективность больше не следует, и у оператора появляются два разных способа быть необратимым:

    \begin{enumerate}
        \item Оператор склеивает векторы.
        \item Образ оператора меньше, чем всё пространство.
    \end{enumerate}

    Поэтому спектр оператора $A$ в бесконечномерном пространстве разбивается на собственные числа и те точки, в которых $A - \lambda I$ не является сюръективным (хоть и векторы не склеивает). 

    \begin{rem}
        Это не мифическая ситуация: обычный оператор умножения на координату (т.е. $Af(x) = x f(x)$) в $L^2\big([a, \, b]\big)$ не имеет собственных чисел, но его спектр равен всему отрезку! 

        Когда мы занимались квантовой механикой, мы находили <<собственные вектора>>~--- дельта-функции. То, что они на самом деле не функции и в $L^2$ не лежат~--- свидетельство описанного феномена!
    \end{rem}

    \paragraph{Компактные операторы}

    Обсудим один класс операторов, очень полезный на практике.

    \begin{de}
        Пусть $H$~--- гильбертово пространство, $B$~--- единичный замкнутый шар в нём. Оператор $A\col \; H \to H$ называют \ti{компактным}, если замыкание множества $A(B)$ компактно.
    \end{de}

    \begin{rem}
        На самом деле, компактный оператор переводит любое ограниченное множество в множество с компактным замыканием.
    \end{rem}

    Мы знаем, что даже единичный шар в $H$ не компактен. Это значит, что $A$~--- оператор с очень маленьким образом, он сжимает всё пространство во что-то крохотное! Это объясняет простоту (и близость к конечномерию) свойств компактных операторов.

    \begin{st}
        Если операторы $A_n$ компактны и $\|A_n - A\| \to 0$, то оператор $U$ компактен.
    \end{st}

    \begin{cor}
        Если операторы $A_n$ конечного ранга (т.е. их образы конечномерны), и $\|A - A_n\| \to 0$, то оператор $A$ компактен.
    \end{cor}

    Главный пример компактного оператора~--- \ti{интегральный оператор}.

    \begin{exm}
        Пусть $\square = [a, \, b] \times [a, \, b]$. Рассмотрим оператор $A$ на $L^2\big([a, \, b]\big)$, действующий по правилу
        \[
            Af(x) = \int\limits_a^b K(x, y) f(y) \, \mathrm{d}y,
        \]
        где $K \in L^2(\square)$. Такой оператор называют \ti{интегральным}, а функцию $K$ называют его \ti{ядром}. В принципе, вместо $L^2$ можно жить в $C$~--- пространстве непрерывных функций, но оно не гильбертово.   
    \end{exm}

    \begin{st}
        Интегральный оператор компактен.
        \begin{proof}[Почти доказательство]
            Разложим функцию $K$ по базису (так можно, правда):
            \[
                K(x, \, y) = \sum\limits_{n, \, m = 0}^{\infty} c_{nm} e_n(x) e_m(y).
            \]
            Рассмотрим последовательность интегральных операторов $A_N$ с ядрами
            \[
                K_N(x, \, y) = \sum\limits_{n, \, m = 0}^{N} c_{nm} e_n(x) e_m(y).
            \]
            Простым преобразованием находим, что
            \[
                A_N f(x) = \sum\limits_{n = 1}^N \left(\,\sum\limits_{m = 1}^N c_{nm} \int\limits_a^b e_m(y) f(y) \, \del y\right) e_n(x).
            \]

            Образ оператора $A_N$ находится внутри линейной оболочки векторов $e_1, \, \ldots, \, e_N$! Это значит, что наш оператор $A$ приближается операторами конечного ранга, а потому компактен.
        \end{proof}
    \end{st}

    \paragraph{Спектры компактных операторов}

    Спектр компактного оператора обладает замечательным свойством:

    \begin{st}
        Пусть $A$~--- компактный оператор. Для любого $\delta > 0$ множество собственных чисел $A$ таких, что $|\lambda| \geqslant \delta$ конечно. Собственное пространство любого $\lambda \neq 0$ конечномерно. 
    \end{st}

    Спектр произвольного самосопряжённого оператора, с другой стороны, обладает такими свойствами:

    \begin{st}
        $\hphantom{.}$
        \begin{enumerate}
            \item Собственные значения самосопряжённого оператора вещественны.
            \item Собственные векторы самосопряжённого оператора, отвечающие разным собственным значениям, ортогональны.
        \end{enumerate}
    \end{st}

    Для операторов, одновременно компактных и самосопряжённых, удаётся доказать вариант \ti{спектральной теоремы}~--- бесконечномерного аналога утверждения о том, что симметричную матрицу можно привести к диагональному виду:

    \begin{thm}[Гильберта-Шмидта] \label{thm:hilb-sch}
        Пусть $A$~--- компактный и самосопряжённый оператор в гильбертовом пространстве $H$. Существует ортогональный базис $\{e_i\}$, состоящий из собственных векторов $A$.
    \end{thm}

    \paragraph{Альтернатива Фредгольма}

    \begin{de}
        \ti{Фредгольмовым} называют такой оператор $T$ на гильбертовом пространстве, что $T = I - A$, где $A$ компактен. 
    \end{de}

    \begin{st}
        Сопряжённый к компактному оператор компактен.
    \end{st}

    \begin{thm}[Альтернатива Фредгольма]\label{thm:fred-alt}
        $\hphantom{.}$
        \begin{enumerate}
            \item Уравнение $T\varphi = f$ разрешимо тогда и только тогда, когда $f$ ортогонально любому решению уравнения $T^{*} \psi_0 = 0$.
            \item Либо уравнение $T\varphi = f$ имеет при любом $f$ ровно одно решение, либо уравнение $T \varphi_0$ имеет ненулевое решение.
            \item Уравнения $T^*\psi_0 = 0$ и $T\varphi_0 = 0$ имеют одно и то же конечное число линейно независимых решений.
        \end{enumerate}
    \end{thm}

    \begin{rem}
        Эту теорему называют \ti{альтернативой}, потому что трудно вынести безальтернативность приближения сдачи вычей. Представьте себе, что вы смотрите на уравнение $T\varphi = f$. Есть два варианта:
        \begin{enumerate}
            \item Уравнение $T \varphi_0$ не имеет ненулевых решений, и ваша задача разрешима единственным способом. Всё прекрасно!
            \item Оно их таки имеет, и всё не столь прекрасно.
        \end{enumerate}
        Пусть вы попали во второй вариант. Снова выбор:
        \begin{enumerate}
            \item $f$ ортогонально всем решениям уравнения $T^{*} \psi_0 = 0$ (которые теперь уже точно есть по третьему пункту). Тогда ваша задача разрешима, но не одним способом (видимо, их будет бесконечно много).
            \item $f$ не такое. Тогда ваша задача неразрешима.
        \end{enumerate}
    \end{rem}
\end{document}
% vim:wrapmargin=3
