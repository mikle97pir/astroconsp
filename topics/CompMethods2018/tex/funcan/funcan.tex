\documentclass{trlnotes}
\setlayout{hardcopy}
\usepackage{silence}
\WarningFilter{latex}{Reference}
\graphicspath{{../../img/}}

\begin{document}
    \paragraph{Пространства, отображения}
    Бесконечномерные пространства во многом похожи на конечномерные, но есть и различия. Приведу наглядный пример:

    \begin{thm}(Рисса)
        В бесконечномерном пространстве с нормой единичный замкнутый шар не компактен. \footnote{Верно и обратное утверждение: если в нормированном пространстве единичный замкнутый шар компактен, то оно конечномерно.}
        \begin{proof}
            Чтобы доказать, что что-то не компактно, нужно найти там последовательность, у которой нет сходящейся подпоследовательности. Здесь это нетрудно: подойдёт любой счётный ортнормированный набор векторов!

            Представьте себе: у вас есть $n$ единичных ортогональных друг другу векторов. Вы можете добавить ещё один, и ещё, и ещё... Конечно, в такой последовательности не выбрать сходящейся.
        \end{proof}
    \end{thm}

    В том, что касается линейных отображений, тоже есть тонкости. Мы знаем, что любое линейное отображение конечномерных пространств непрерывно и \ti{ограничено} (т.е. образ единичного замкнутого шара при нём ограничен). В бесконечномерном случае это не так! Однако выполняется такое утверждение:

    \begin{st}
        Для нормированных пространств непрерывность и ограниченность линейных отображений равносильны.
    \end{st}

    В реальности почти все интересные отображения ограничены. Да и у неограниченных слишком плохие свойства, поэтому в большинстве теорем ограниченность предполагается.

    \paragraph{Спектр оператора}

    Ещё одно различие, не столь наглядное, но очень важное, связано со \ti{спектром} оператора.

    \begin{de}
        Пусть $H$~--- гильбертово пространство, $A\col \; H \to H$~--- ограниченный оператор. \ti{Спектром} $A$ называют множество таких $\lambda \in \C$, что оператор $A - \lambda I$ необратим.
    \end{de}
    Понятие спектра тесно связано с собственными числами:
    \begin{de}
        Говорят, что $\lambda \in \C$~--- \ti{собственное число} оператора $A$, если есть такой вектор $v \in H$, что $Av = \lambda v$.
    \end{de}
    Собственные числа можно охарактеризовать в терминах оператора $A - \lambda I$:
    \begin{st}
        $\lambda$~--- собственное число $A$ тогда и только тогда, когда оператор $A - \lambda I$ не инъективен (то есть склеивает какие-то векторы в один).
        \begin{proof}
            Пусть $\lambda$~--- собственное число, $v$~--- собственный вектор. Тогда $(A - \lambda I)v = 0 = A0$, поэтому оператор не инъективен.

            Докажем в обратную сторону. Пусть оператор $A - \lambda I$ не инъективен. Тогда есть вектор из ядра~--- такой, что $(A - 
            \lambda I)v = 0$, т.е. $Av = \lambda v$.
        \end{proof}
    \end{st}

    Отсюда сразу следует утверждение:
    \begin{st}
        Для конечномерных пространств спектр и множество собственных чисел~--- одно и то же.
        \begin{proof}
            Как мы знаем,
            \[
                \text{необратимость} \eqv \text{неинъективность или несюръективность}.
            \]
            Но в конечномерном случае 
            \[
                \text{несюръективность} \so \text{неинъективность}.
            \]
            Это связано с тем, что несюръективный оператор понижает размерность пространства, что вынуждает его склеивать вектора.

            Поэтому необратимость либо сразу влечёт неинъективность, либо сначала влечёт несюръективность, а потом уже неинъективность. Отсюда
            \[
                \text{необратимость} \eqv \text{неинъективность},
            \]
            что и требовалось доказать.
        \end{proof}
    \end{st}

    В бесконечномерном случае всё не так. Из необратимости неинъективность больше не следует, и у оператора появляются два разных способа быть необратимым:

    \begin{enumerate}
        \item Оператор склеивает векторы.
        \item Образ оператора меньше, чем всё пространство.
    \end{enumerate}

    Поэтому спектр оператора $A$ в бесконечномерном пространстве разбивается на собственные числа и те точки, в которых $A - \lambda I$ не является сюръективным (хоть и векторы не склеивает). 

    \begin{rem}
        Это не мифическая ситуация: обычный оператор умножения на координату (т.е. $Af(x) = x f(x)$) в $L^2\big([a, \, b]\big)$ не имеет собственных чисел, но его спектр равен всему отрезку! 

        Когда мы занимались квантовой механикой, мы находили <<собственные вектора>>~--- дельта-функции. То, что они на самом деле не функции и в $L^2$ не лежат~--- свидетельство описанного феномена!
    \end{rem}

    \paragraph{Компактные операторы}
\end{document}
% vim:wrapmargin=3
