\documentclass{trlnotes}
\usepackage{trmath}
\addcompatiblelayout{commonplace}
\setlayout{commonplace}
\usepackage{trthm}
% \RequirePackage{stmaryrd}

% \RequirePackage{etoolbox}


%\RequirePackage{cmap}
% \RequirePackage{hyperref}
% \theoremstyle{definition}
% \newtheorem{thm}{Теорема}
% \newtheorem{de}{Определение}
% \newtheorem{lm}{Лемма}
% \newtheorem{exm}{Пример}
% \newtheorem{pr}{Свойство}
% \newtheorem{exc}{Упражнение}
% \newtheorem{cor}{Следствие}
% \newtheorem{st}{Утверждение}
% \theoremstyle{remark}
% \newtheorem{rem}{Замечание}
\newcommand*{\icom}{\ensuremath\text{\textit{,}}}
\newcommand*{\icol}{\ensuremath\text{\textit{:}}}
\newcommand*{\iscol}{\ensuremath\text{\textit{;}}}
\newcommand*{\com}{\ensuremath\text{\text{,}}}
\newcommand*{\col}{\ensuremath\text{\text{:}}}
\newcommand*{\scol}{\ensuremath\text{\text{;}}}
% \newcommand*{\so}{\ensuremath\Rightarrow}
\newcommand*{\bso}{\ensuremath\Leftarrow}
\newcommand*{\eqv}{\ensuremath\Leftrightarrow}
\newcommand*{\all}{\forall}
\newcommand*{\ex}{\exists \,}
% \newcommand*{\ov}{\overline}
\newcommand*{\un}{\underline}
\newcommand*{\ova}{\overrightarrow}
\newcommand*{\auth}[1]{\hfill \textit{#1}}
\newcommand*{\pd}{\partial}
% \newcommand*{\R}{\mathbb{R}}
% \newcommand*{\N}{\mathbb{N}}
\newcommand*{\DD}{\mathbb{D}}
\newcommand*{\K}{\mathbb{K}}
% \newcommand*{\C}{\mathbb{C}}
% \newcommand*{\C}{\mathbb{C}}
% \newcommand*{\Q}{\mathbb{Q}}
\newcommand{\An}{\wedge}
\newcommand{\Or}{\vee}
% \newcommand*{\Z}{\mathbb{Z}}
\newcommand*{\J}{\mathbb{J}}
\newcommand*{\bas}[2]{\overset{\vspace{-3pt}\tiny{\mb{#1}}}{#2}}
\newcommand*{\mc}{\mathcal}
\newcommand*{\mb}{\mathbf}
\newcommand*{\mf}{\mathfrak}
\newcommand*{\ti}{\textit}
\newcommand*{\la}{\langle}
\newcommand*{\ra}{\rangle}
\newcommand*{\tb}{\textbf}
\newcommand*{\mr}{\mathrm}
\newcommand*{\wt}{\widetilde}



% \DeclareMathOperator{\rk}{rk}
\DeclareMathOperator{\Dom}{Dom}
% \DeclareMathOperator{\id}{id}
\DeclareMathOperator{\Cl}{Cl}
% \DeclareMathOperator{\Res}{Res}
\DeclareMathOperator{\im}{Im}
% \DeclareMathOperator{\rot}{rot}
\DeclareMathOperator{\Div}{div}
% \DeclareMathOperator{\grad}{grad}
\DeclareMathOperator{\Id}{Id}
% \DeclareMathOperator{\Aut}{Aut}
% \DeclareMathOperator{\Stab}{Stab}
\DeclareMathOperator{\const}{const}

%\titleformat{\section}
%  {\sffamily\mdseries\upshape\LARGE}
%  {Билет \thesection:}{0.5em}{}



\usepackage{silence}
\WarningFilter{latex}{Reference}
\graphicspath{{../../img/}}

\newtagform{roman}[\renewcommand{\theequation}{\Roman{equation}}]()
\begin{document}
\paragraph{Краевая задача для ОДУ 2 порядка и сведение к задаче Коши}
\label{par:ode::bprobl}
\begin{defn}\label{defn:ode::bprobl}
  Рассмотрим ОДУ 2 порядка
  \[
    y'' + p(x) y' + q(x) y = f(x) \qquad y \in C^2([a;b])
  \]
  и 3 варианта условий на $y$
  \begin{enumerate}[I]
    \item $y(a) = A, \quad y(b) = B$ \label{it:ode::bprobl::cond:i}
    \item $y'(a) = A, \quad y'(b) = B$ \label{it:ode::bprobl::cond::ii}
    \item $y'(a) = α y(a) + A, \quad y'(b) = βy(b) + B$ \label{it:ode::bprobl::cond::iii}
    \end{enumerate}
  
  Если $y$~--- решение для которого выполнено какое-то из условий выше, то $y$~--- решение
  граничной задачи.
\end{defn}

\begin{thrm}[об альтернативе]\label{thrm:ode::bprobl::alt}
  Пусть в граничной задаче \ref{defn:ode::bprobl} $A =0, B=0$.
  Рассмотрим решение однородной задачи $y_0$. 

  Тогда
  \begin{enumerate}
    \item $y_0 \equiv 0$~--- единственное решение однородной задачи \so неоднородная краевая 
      задача имеет единственное решение
    \item $y_0 \equiv 0$~--- неединственное решение однородной задачи \so неоднородная краевая 
      задача имеет бесконечно много или не имеет решений вовсе
  \end{enumerate}
\end{thrm}


\begin{prf}
  Рассмотреть решение неоднородной краевой в виде $y(x) = y_0(x) + c_1 y_1(x) + c_2 y_2(x)$
  и подставить граничные условия, а дальше все следует из линейной алгебры.
\end{prf}


\end{document}
% vim:wrapmargin=3
