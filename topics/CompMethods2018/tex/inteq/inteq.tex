\documentclass{trlnotes}
\setlayout{hardcopy}
\usepackage{silence}
\WarningFilter{latex}{Reference}
\graphicspath{{../../img/}}

\begin{document}
    \paragraph{Интегральное уравнение II рода, метод замены ядра на вырожденное}

    \begin{de}
        \ti{Интегральным уравнением Фредгольма II рода} называется уравнение вида
        \begin{equation} \label{eq:fred2}
            \varphi(x) = f(x) + \mu \int\limits_a^b K(x, \, t) \varphi(t) \, \del t.
        \end{equation}
        Функция $K$~--- его \ti{ядро}, а $\mu$~--- \ti{характеристическое число}.\footnote{Кажется, иногда в определении полагают $\mu = 1$, но всегда ведь можно внести его в ядро. Мы иногда тоже будем на него забивать.}
    \end{de}

    Обозначим через $K$ (хм, да, вольность) оператор 
    \[
        \varphi(t) \mapsto \int\limits_a^b K(x, \, t) \varphi(t) \, \del t.
    \]
    Ясно, что он компактен. Уравнение теперь примет вид
    \[
        (I - \mu K)\varphi = f.
    \]
    Оператор $T = I - \mu K$, конечно, фредгольмов.

    \begin{st}
        Сопряжённый в $L^2\big([a, \, b]\big)$ оператор к $K$ выражается следующим образом:
        \[
            K^* \varphi(x) = \int\limits_a^b \ov{-}{K(t, \, x)} \varphi(t) \, \del t.
        \]
        \begin{proof}
            Прямым вычислением (ну, там внутри ещё теорема Фубини) проверяется, что 
            \[
                \langle K \varphi, \, \psi \rangle = \langle \varphi, \, K^{*} \psi \rangle.
            \]
        \end{proof}
    \end{st}

    \begin{rem}
        У ядра меняются местами аргументы и оно сопрягается~--- точно так же, как транспонирование вместе с комплексным сопряжением дают матрицу сопряжённого оператора в конечномерном случае!
    \end{rem}

    Сформулируем альтернативу Фредгольма \ref{thm:fred-alt} для такого уравнения:

    \begin{st}
        $\hphantom{.}$
        \begin{enumerate}
            \item Уравнение $T\varphi = f$ разрешимо однозначно тогда и только тогда, когда $\mu^{-1}$~--- не собственное число оператора $K$.
            \item В противном случае уравнение $T \varphi = f$ разрешимо тогда и только тогда, когда функция $f$ ортогональна всем собственным векторам оператора $K^*$, соответствующим числу $\ov{-}{\mu}^{-1}$.
            \item $\mu^{-1}$ и $\ov{-}{\mu}^{-1}$~--- собственные числа операторов $K$ и $K^*$ соответственно одинаковой конечной кратности.
        \end{enumerate}
    \end{st}

    \begin{rem}
        Для симметричного ядра (т.е. когда $K = K^*$) то же самое несложно доказать, используя разложение по собственному базису оператора $K$ (которое есть по теореме Гильберта-Шмидта \ref{thm:hilb-sch}). Так можно быстро понять, что если $\mu^{-1}$~--- собственное число $K$, то решений либо нет, либо их бесконечно много.
    \end{rem}

    Рассмотрим уравнение \ref{eq:fred2} с вырожденным ядром
    \[
        K(x, \, t) = \sum\limits_{i = 1}^n \alpha_i(x) \beta_i(t).
    \]
    Функции $\alpha_i$ и $\beta_i$ можно считать ЛНЗ: если это не так, нетрудно выразить одну из них через другие и избавиться от неё. Подставляя ядро в уравнение \ref{eq:fred2}, получим

    \begin{equation}\label{eq:deg-ker-repr}
        \varphi(x) = f(x) + \sum\limits_{j = 1}^n A_j \alpha_i(x), \text{ где } A_j = \mu \int\limits_a^b \beta_j(t) \varphi(t) \, \del t.
    \end{equation}

    Это представление для функции $\varphi$ теперь подставим в исходное уравнение:

    \[
        f(x) + \sum\limits_{i = 1}^n A_i \alpha_i(x) = f(x) + \mu \int\limits_a^b \sum\limits_{i = 1}^n \alpha_i(x) \beta_i(t) \left(f(t) + \sum\limits_{j = 1}^n A_j \alpha_j(t) \right) \, \del t
    \]
    Чтобы переписать это покороче, введём обозначения
    \[
        \beta_{ij} = \int\limits_a^b \beta_i(t) \alpha_j(t) \, \del t, \quad f_i = \int\limits_a^b f(t) \beta_i(t) \, \del t.
    \]
    и получим
    \[
        \sum\limits_{i = 1}^n A_i \alpha_i(x) = \mu \sum\limits_{i = 1}^n \left(f_i + \sum\limits_{j = 1}^n \beta_{ij} A_j \right) \alpha_i(x).
    \] 
    Поскольку $\alpha_i$ линейно независимы, коэффициенты при них слева и справа должны быть равны. Записав эти равенства, мы приходим к системе линейных уравнений
    \[
        A_i = \mu f_i + \mu \sum\limits_{j = 1}^n \beta_{ij} A_j .
    \]
    В векторном виде она будет выглядеть так:
    \[
        A = \mu (\beta A + f),
    \]
    где $A$ и $f$~--- векторы, $\beta$~--- матрица, а $\mu$ всё ещё число.

    Эта система решается так:
    \[
        (I - \mu \beta) A = \mu f \so \boxed{A = \mu (I - \mu \beta)^{-1} f}\,, \text{ если } \det (I - \mu \beta) \neq 0.
    \]

    Пусть $\Delta = \det(I - \mu \beta)$ и $\Delta_{ij}$~--- алгебраическое дополнение элемента $\delta_{ij} - \mu \beta_{ij}$. Тогда можно записать явную формулу для $A$\footnote{Это просто формула для обратной матрицы через алгебраические дополнения.}:
    \[
        A_{i} = \dfrac{\mu}{\Delta} \sum\limits_{j = 1}^n \Delta_{ji} f_j
    \]
    Подставляя теперь найденные $A_i$ в \ref{eq:deg-ker-repr}, найдём, что
    \[
        \varphi(x) = f(x) + \lambda \int\limits_a^b \Gamma(x, \, t) f(t) \, \del t,
    \]
    где \ti{резольвента} $\Gamma$ имеет вид
    \[
        \Gamma(x, \, t) = \dfrac{1}{\Delta} \sum\limits_{i, \, j = 1}^n \Delta_{ji} \alpha_i(x) \beta_j(s).
    \]

    Трудная задача~--- приблизить произвольное ядро вырожденным. Есть несколько способов:

    \begin{enumerate}
        \item Разложить ядро в ряд Тейлора.
        \item Интерполировать ядро.
        \item Разложить ядро по ортогональной системе функций.
    \end{enumerate}


    Подробнее про них можно прочитать в книге \cite{comp-krilov-2}.

    Заменяя ядро на вырожденное, мы надеемся, что и решения тоже изменятся не сильно. Надо бы это обосновать (хотя бы как-то). Пусть есть уравнение
    \[
        Au = f, \quad A = I - K
    \]
    и приближающее его уравнение
    \[
        A_n u_n = f, \quad A_n = I - K_n. 
    \]
    Нетрудно видеть, что
    \[
        u - u_n = (A^{-1} - A_n^{-1})f \so \|u - u_n\| \leqslant \big\|A^{-1} - A_n^{-1}\big\| \cdot \|f\|.
    \]
    Поэтому интересно оценить норму разности обратных операторов. Займёмся этим.

    \begin{st}\label{st:close-zero-inv}
        Пусть $P$~--- ограниченный оператор, $\|P\| < 1$. Тогда оператор $I - З$ обратим, причём
        \[
            (I - P)^{-1} = \sum\limits_{i = 1}^{\infty} P^n,
        \]
        где сходимость~--- по операторной норме.
    \end{st}

    \begin{st} \label{st:close-inv}
        Пусть $P$ и $H$~--- ограниченные операторы, $P$ обратим, а $\|H\| < \|P^{-1}\|^{-1}$. Тогда элемент $P - H$ обратим, причём
        \[
            \big\| \, (P - H)^{-1}\big\|  \leqslant \dfrac{\|P\|^{-1}}{1 - \|H\| \, \|P^{-1}\|}. 
        \]
        и
        \[
            \big\| \, (P - H)^{-1} - P^{-1}\big\| \leqslant \dfrac{\|H\| \, \|P^{-1}\|^2}{1 - \|H\| \, \|P^{-1}\|}.
        \]
        \begin{proof}
            Позволим себе иногда использовать дроби и $1$ вместо $I$, как если бы операторы были числами. Не составит труда переписать всё через обратные!

            Заметим, что первое из двух утверждений теоремы для $P = I$ следует из \ref{st:close-zero-inv}:
            \begin{equation}\label{eq:close-zero-inv}
                \big\| \, (I - H)^{-1}\big\| = \left \| \, \sum\limits_{i = 1}^{\infty} H^n \, \right\| \leqslant \sum\limits_{i = 1}^{\infty} \|H\|^n = \dfrac{1}{1 - \|H\|}.
            \end{equation}

            Далее,
            \[
                \left\|\, \dfrac{1}{P - H} \,\right \| = \left\|\, P^{-1} \dfrac{1}{1 - P^{-1}H} \,\right \| \leqslant \|P^{-1}\| \cdot \left\|\,\dfrac{1}{1 - P^{-1}H} \,\right \| \leqslant \dfrac{\|P^{-1}\|}{1 - \|P^{-1}H\|} \leqslant \dfrac{\|P\|^{-1}}{1 - \|H\| \, \|P^{-1}\|}.
            \]
            В предпоследнем переходе используется соотношение \ref{eq:close-zero-inv}, где $H \to P^{-1}H$.

            Наконец,
            \[
                \left\|\, \dfrac{1}{P - H} - \dfrac{1}{P}\,\right \| = \left\|\, \dfrac{1}{P}\left(\dfrac{1}{1 - P^{-1}H} - 1 \right)\right \| =  \left\|\, \dfrac{1}{P} \, \dfrac{P^{-1}H}{1 - P^{-1}H} \right \| \leqslant \dfrac{\|H\| \, \|P^{-1}\|^2}{1 - \|H\| \, \|P^{-1}\|}.
            \]
        \end{proof}
    \end{st}

    Отсюда сразу же следует утверждение

    \begin{st}\label{st:dif-inv}
        При достаточно больших $n$
        \[
            \big\|A^{-1} - A_n^{-1}\big\| \leqslant \dfrac{\rho \, \|A^{-1}\|^2}{1 - \rho \, \|A^{-1}\|} \text{ и } \big\|A^{-1} - A_n^{-1}\big\| \leqslant \dfrac{\rho \, \|A_n^{-1}\|^2}{1 - \rho \, \|A_n^{-1}\|},
        \]
        где $\rho = \|A - A_n\| = \|K - K_n\|$.
    \end{st}

    \begin{rem}
        Рассмотрим теперь задачу с симметричным ядром (т.е. с самосопряжённым $K$). В ней есть ортонормированный собственный базис $\alpha_i$, поэтому
        \[
            u = \sum\limits_{i = 1}^{\infty} \langle u, \, \alpha_i \rangle \alpha_i \so Ku = \sum\limits_{i = 1}^{\infty} \langle u, \, \alpha_i \rangle \lambda_i \alpha_i,
        \]
        где $\lambda_i$~--- соответствующее собственное число. Расположим $\lambda_i$ в порядке убывания модуля и положим
        \[
            K_n u = \sum\limits_{i = 1}^{n} \langle u, \, \alpha_i \rangle \lambda_i \alpha_i
        \]
        Это интегральный оператор с вырожденным ядром
        \begin{equation}\label{eq:eig-deg-ker}
            K_n(x, \, t) = \sum\limits_{i = 1}^n \lambda_i \alpha_i(x) \ov{-}{\alpha_i(t)}.
        \end{equation}

        %unsure
        Можно доказать, что он является лучшей аппроксимацией ранга $n$ для оператора $K$ по операторной $L^2$-норме.
        %unsure

        Посмотрим на разность:
        \[
            (K - K_n)u = \sum\limits_{i = n+1}^{\infty} \langle u, \, \alpha_i \rangle \lambda_i \alpha_i.
        \]
        Найдём её норму:
        \[
            \big\|(K - K_n)u \,\big\|^2 = \sum\limits_{i = n + 1}^{\infty} |u_i|^2 \, |\lambda_i|^2, \quad u_i = \langle u, \, \alpha_i \rangle.
        \]
        При этом
        \[
            \|K - K_n\| = \sup \dfrac{\big\|(K - K_n)u \,\big\|}{\|u\|},
        \]
        и
        \[
            \dfrac{\big\|(K - K_n)u \,\big\|^2}{\|u\|^2} = \dfrac{\sum\limits_{i = n + 1}^{\infty} |u_i|^2 \, |\lambda_i|^2}{\sum\limits_{i = n + 1}^{\infty} |u_i|^2} \leqslant \dfrac{\sum\limits_{i = n + 1}^{\infty} |u_i|^2 \, |\lambda_{n + 1}|^2}{\sum\limits_{i = n + 1}^{\infty} |u_i|^2} = |\lambda_{n+1}|^2.
        \]
        С другой стороны, эта оценка достигается, когда $u$~--- собственный вектор числа $\lambda_{n+1}$. Поэтому
        \[
            \boxed{\|K - K_n\| = |\lambda_{n + 1}|} \, .
        \]
        Отсюда и из утверждения \ref{st:dif-inv} ясно: чем быстрее убывают собственные числа, тем лучше наша оценка! 
        %unsure
        Из уравнения \ref{eq:eig-deg-ker} видно, что собственные числа~--- что-то вроде коэффициентов в ряде Фурье по собственным функциям для ядра. Видимо, поэтому скорость их убывания возрастает, если ядро становится более гладким... А ядра гладкие не всегда.
        %unsure
    \end{rem}

    \begin{rem}
        Есть способ сгладить ядро. Надо в уравнение \ref{eq:fred2} подставить
        \[
            \varphi(t) = f(t) + \mu\int\limits_a^b K(t, \, \xi) \varphi(\xi) \, \del \xi.
        \]
        Получится уравнение
        \[
            \varphi(x) = f_2(x) + \mu\int\limits_a^b  K_2(x, \xi) \varphi(\xi) \, \delta \xi,
        \]
        где
        \[
            f_2(x) = f(x) + \mu \int\limits_a^b K(x, \, t) f(t) \, \del t, \quad K_2(x, \, \xi) = \mu \int\limits_a^b K(x, \, t) K(t, \, \xi) \, \del t. 
        \]
        У $K_2$ с гладкостью получше, но его надо считать.
    \end{rem}

    \paragraph{Метод квадратур для интегрального уравнения}

    Идея заключается в том, чтобы в уравнении
    \[
        u(x) = f(x) + \int\limits_a^b K(x, \, t) u(t)\, \del t
    \]
    заменить интегрирование на вычисление по какой-нибудь квадратурной формуле:
    \[
        \int\limits_a^b u (x) \, \del x = \sum\limits_{k = 1}^n A_k u(x_k) + R.
    \]
    Получится
    \[
        u(x) = f(x) + \sum\limits_{k = 1}^n A_k K(x, \, x_k) u(x_k) + R.
    \]
    Пусть $\tilde{u}$~--- решение этого уравнения с отброшенным $R$, $u_k = \tilde{u}(x_k)$, $f_k = f(x_k)$ и $K_{ik} = K(x_i, \, x_k)$.
    Получаем систему линейных уравнений
    \[
        u_i = f_i + \sum\limits_{k = 1}^n A_k K_{ik} u_k.
    \]
    Её можно решить обычными методами; зная $u_k$, можно оценить $u(x)$ в любой точке:
    \[
        u(x) = f(x) + \sum\limits_{k = 1}^n A_k K(x, \, x_k) u_k.
    \]

    Попробуем оценить погрешность результата. Для многих стандартных квадратурных методов верна формула
    \[
        R[\theta] = \delta(n) \max |\theta^{(m)}(x)|.
    \]
    Нас интересует $R\big[K(x, \, t)u(t)\big]$ при фиксированном $x$. $m$-е производные функции $K(x, \, t)u(t)$ выражаются через производные известной $K(x, \, t)$ и через производные $u(t)$ порядка не более $m$.

    Чтобы оценить их, продифференцируем наше интегральное уравнение:
    \[
        u^{(l)}(x) = f^{(l)}(x) + \int\limits_a^b K^{(l)}_x(x, \, t) u(t)\, \del t.
    \]
    Отсюда можно найти оценку для $u^{(l)}$ через известные $f$ и $K$ и максимум модуля решения. Решение же можно записать, как
    \[
        u = (I - K)^{-1} f \so \|u\| \leqslant \big\|(I - K)^{-1}\big\| \cdot \|f\| \leqslant \dfrac{\|f\|}{1 - \|K\|} \leqslant \dfrac{\|f\|}{1 - \varkappa},
    \]
    где 
    \[
        \varkappa = (b - a) \max \big|K(s, \, t)\big|.
    \]
    Предпоследний переход обусловлен утверждением \ref{st:close-inv}. 

    \begin{rem}
        Во-первых, сейчас у нас все нормы~--- $L^1$, от этого ничего не портится. Во-вторых, мы только что неявно предположили, что $|\varkappa| < 1$.
    \end{rem}

    Получив оценку для модуля решения, мы можем найти оценку
    \[
        \left|\dfrac{\pd^{m}}{\pd t^m} \big(K(x, \, t) u(t)\big)\right| \leqslant M,
    \]
    зависящую только от известных функций.

    Перейдём теперь непосредственно к оценке ошибки. У нас есть два уравнения 
    \begin{align*}
        Au &= f, \quad A = I - K; \\
        \tilde{A} \tilde{u} &= f, \quad \tilde{A} = I - \tilde{K},
    \end{align*}
    где 
    \[
        \tilde{K}\varphi(x) = \sum\limits_{i = 1}^n A_i K(x, \, x_i) \varphi(x_i).
    \]
    Заметим, что
    \[
        \tilde{A}(u - \tilde{u}) = \tilde{A}u - Au \so \|u - \tilde{u}\| \leqslant \boxed{\|\tilde{A}^{-1}\| \, \|\tilde{A}u - Au\|}\,.
    \]
    Оценим норму $\tilde{A}^{-1}$. Для этого сначала оценим норму $\tilde{K}$:
    \[
        \left | \, \sum\limits_{i = 1}^n A_i K(x, \, x_i) \varphi(x_i) \right | \leqslant \max |K| \cdot \|\varphi\| \cdot \sum\limits_{i = 1}^n A_i = (b - a) \max |K| \cdot \|\varphi\|,
    \]
    поэтому $\|\tilde{K}\| \leqslant \kappa$.

    Отсюда
    \[
        \|\tilde{A}^{-1}\| = \big\|(I - \tilde{K})^{-1}\big\| \leqslant \dfrac{1}{1 - \kappa}.
    \]
    Теперь оценим $\|\tilde{A}u - Au\|$:
    \[
        \|\tilde{A}u - Au\| = \max \big|R\big[K(x, \, t) u(t)\big]\big| \leqslant M\delta(n).
    \]
    В конечном итоге находим
    \[
        \boxed{\|u - \tilde{u}\| \leqslant \dfrac{M\delta(n)}{1 - \kappa}}\,.
    \]

    Подробнее про этот метод можно прочитать в книгах \cite{gavurin} и \cite{comp-krilov-2}.
\end{document}
% vim:wrapmargin=3
