%------------------------------------------------------------
% Description : Стандартные обозначения 
% Author      : taxus-d <iliya.t@mail.ru>
% Created at  : Sat Jan 14 17:39:43 MSK 2017
%------------------------------------------------------------
\documentclass[12pt]{../../notes}
\usepackage{silence}
\WarningFilter{latex}{Reference}
\graphicspath{{../../img/}}


\begin{document}
\begin{description}
  \item[$\Omega$]~--- пространство элементарных исходов
  \item[$\omega$]~--- элемент пространства элементарных исходов
  \item[$\mathcal A$]~--- алгебра множеств 
  \item[$\mathcal F$]~--- $\sigma$-алгебра множеств (случайные события)
  \item[$P$]~--- вероятность
  \item[$p(x_1, \dotsc, x_n)$]~--- плотность вероятности
  \item[$X(\omega), Y(\omega), \xi(\omega)$]~--- случайная величина
  \item[$F_X(x_1, \dotsc, x_n)$]~--- функция распределения случайной величины $X$
  \item[$\Phi_X(t)$]~--- характеристическая функция случайной величины $X$.
\end{description}

\noindent\rule{\textwidth}{0.01em}

\begin{description}
  \item[\underdev]~--- ещё правится. Впрочем, относится почти ко всему.
  \item[$\square\cdots\blacksquare$]~--- начало и конец доказательства теоремы
  \item[$\blacktriangledown\cdots\blacktriangle$]~--- начало и конец доказательства более мелкого 
    утверждения
  \item[\sour]~--- кривоватая формулировка
  \item[\flame]~--- набирающему зело не нравится билет
  \item[\plholdev{что-то}]~--- тут будет \texttt{что-то}, но попозже
  \item[$a\intrng b$]~--- для $a,b \in \Z$ это просто $[a;b]\cap \Z$
  \item[$\equiv$]~--- штуки эквивалентны. Часто используется в этом смысле в определениях, когда вводится
    два разных обозначения одного и того же объекта.
\end{description}
\end{document}

