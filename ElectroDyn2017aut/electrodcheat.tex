\documentclass{trchesh}
\usepackage{trmath}
\usepackage{trsym} 

\setlayout[page-orient=landscape]{hardcopy}

\hypersetup{
  pdftitle={Шпора по электроду},
  pdfauthor={taxus},
  pdfsubject={Электродинамика},
  pdfkeywords={Электродинамика;СПбГУ;by-taxus}
}
\columnseprule=0.1dd
\def\arraystretch{1.3}
\begin{document}

\begin{multicols*}{3}
  \begin{enumerate}
    \item Уравнения Максвелла \par
      $
      \begin{aligned}
        &\div \v E &&=&& 4 \pi \rho \\
        &\div \v H &&=&& 0 \\
        &\rot \v E &&=&& - \frac 1 c \, \pder{\v H}{t} \\
        &\rot \v H &&=&& \frac 1 c\, \pder{\v E}{t} + \frac{4 \pi}{c}\v j
      \end{aligned}$
    \item Они же в интегральной
    \item $\pder \rho t  + \div \v j = 0 $
    \item В среде\\
      $
      \begin{aligned}
        \v P \that\v j_{pol} &=\pder{\v P}t,\;\rho_{pol} =-\div \v P,\\
      \v M \that \v j_m &= c \rot \v M\end{aligned}$ \par
    В сильнопеременных 
      $\v j_{int} = \pder{\v P}{t}  + c \rot \v M$
    \item $\v D = \v E + 4 \pi \v P$, $\v H = \v B - 4 \pi \v M$ \par
      $ 
      \begin{aligned}
        \div \v D &= 4 \pi \rho_{ex} \\
        \rot \v H &= \frac 1 c \, \pder {\v D} t + \frac{4\pi}{c} \,
        (\v j_{ex} + \v j _{c})           
      \end{aligned}$
    \item Материальные уравнения (простейшие)\par
      $\v D = \varepsilon \v E,\;\v B = \mu \v H ,\;\v j_c = \sigma \v E$
    \item Дисперсия, варианты \par
      $ \begin{aligned}
        \v D (\v r,t) &= \int_{-\infty}^t f (t' - t, \v r)\,\v E(\v r, t')\, \del t \\
        \v D (\v r,t) &= \int_{\Delta V} g(\v r' -\v r, t)\,\v E(\v r, t')\, \del V
      \end{aligned} $
    \item Энергетические соотношения \par
      $
      \begin{aligned}
        w &= \frac{1}{8 \pi} \, (\varepsilon E^2 + \mu H^2) \\
        \v B &= \frac{c}{4 \pi} \, \v E \times \v H \\
        \pder{w}{t} + \div \v S &= - \sigma E^2 - \v E \cdot \v j_{ex}
      \end{aligned}
      $ \par
      Так что если внешние силы не совершают работы, 
      энергия лишь убывает.
    \item Потенциал: \par
      $ \v E = - \frac 1 c \pder{\v A} t  - \nabla \varphi$,
      $\v B = \rot \v A$ \par
      Калибровка Лоренца: 
      $\frac{\varepsilon \mu}{c} \, \pder{\varphi}t + \div \v A = 0$
    \item Волны в однородной среде\par
      $\begin{aligned}
      \Box \v E = 0, \; \Box \v B  =0 \\
      \Box \v A =0, \; \Box \varphi  =0 , \; \Box f  = 0 
      \end{aligned}$ \par
      Тут $f$~--- функция, зависящая от калибровки.
    \item Плоская волна\par
      $\begin{aligned}
      \frac 1 {v^2} \, \pder[2]{u}{t} - \pder[2]{u}{x} = 0 \\
      u = f(x-vt) + g(x+vt)
      \end{aligned}$ \par
      Ещё $\v B = \frac cv \,\v n \times \v E$, a $\v S = v w \v n$.
    \item Сферические волны \underdev
    \item Монохроматические волны \par
      $\begin{aligned}
        u \propto \cos(\omega t + \alpha) \\
        \Delta u + \frac{\omega^2}{v^2} u =0, \quad k = \frac \omega v\\
      \end{aligned}$
    \item Поляризационная матрица, параметры Стокса\par
      $\rho = \begin{pmatrix}
        \averg{|E_x|^2} & \averg{E_x E_y^*} \\
        \averg{E_y E_x^*} & \averg{ |E_y|^2} \\
      \end{pmatrix}=\frac 12 \, \begin{pmatrix}
        I+Q & U-iV \\
        U+iV & I-Q \\
      \end{pmatrix}
      $
    \item Частные случаи \par
      \begin{enumerate}
        \item $Q=U=V=0$~"--- белый свет
        \item $Q=U=0$~"--- круговая поляризация
        \item $V=0$~"--- линейная поляризация
      \end{enumerate}
    \item Геометрическая оптика \par
      $\begin{aligned}
        u = u_{0} e^{i \psi} \\
        \frac{1}{v^2} \left(\pder{\psi}t\right)^2 - (\nabla \psi)^2 = 0\\
        \psi = - \omega t + \psi_1, \; (\nabla \psi_1)^2 = n^2(\v r)
      \end{aligned}$
    \item Гадость в неоднородной среде
    \item $E,H$-волны.\par
      $E''(z) + f(z) \, E(z) = 0$, $f(z) = k^2 - \varkappa^2$
  \end{enumerate}
\end{multicols*}
\end{document}
