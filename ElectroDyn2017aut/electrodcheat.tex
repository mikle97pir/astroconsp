\documentclass[draft]{trchesh}
\usepackage{trmath}
\usepackage{trsym} 

\setlayout[page-orient=landscape]{hardcopy}
\makeatletter
\def\note{\ifexpmode{\@gobble}{\endnote}}
\makeatother
\hypersetup{
  pdftitle={Шпора по электроду},
  pdfauthor={taxus},
  pdfsubject={Электродинамика},
  pdfkeywords={Электродинамика;СПбГУ;by-taxus}
}
\columnseprule=0.1dd
\def\arraystretch{1.3}
\setlist[enumerate,1]{leftmargin=2em}
\setlist[itemize,1]{leftmargin=2em, label=$\triangleright$}

\def\waveop{\mathop{\boldsymbol\Box}}
\begin{document}

\begin{multicols*}{3}\raggedright
\parindent=0pt
\paragraph{Уравнения Максвелла}
\begin{enumerate}
  \item Теорема Гаусса: $\oint \v E \cdot \del \v s = 4 \pi Q$.
  \item Закон Фарадея: $\oint\v E \cdot \del \v l = - \frac{1}{c} \pder{\Phi}{t}$, \\
    $\Phi=\int \v B \cdot \del \v s$
  \item Закон Био-Савара-Лапласа: $\v B = \frac{1}{c} \, \frac{\v j \times \v R}{R^3}$
  \item $\oint \v B \cdot \del \v s = 0$
  \item Закон Ампера: $\oint \v B \cdot \del \v l = \frac{4 \pi}{c} \int \v j \cdot \del \v s$
  \item Уравнение неразрывности: $\pder \rho t  + \div \v j = 0 $
  \item Сами уравения Максвелла: \par\vspace{1ex}$
    \begin{aligned}
      &\div \v E &&=&& 4 \pi \rho \\
      &\div \v B &&=&& 0 \\
      &\rot \v E &&=&& - \frac 1 c \, \pder{\v B}{t} \\
      &\rot \v B &&=&& \frac 1 c\, \pder{\v E}{t} + \frac{4 \pi}{c}\v j
    \end{aligned}$
\end{enumerate}

\paragraph{В среде}
\begin{enumerate}
  \item Поляризация и намагниченность \\
    $\begin{aligned}
    \v P \that\v j_{\mathrm{pol}} &=\pder{\v P}t,\;\rho_{\mathrm{pol}} =-\div \v P,\\
    \v M \that \v j_{\mathrm{m}} &= c \rot \v M \\
    \{\rho, \v j\}_{\mathrm{int}} &= \{\rho, \v j\}_{\mathrm{pol}} + \{\rho, \v j\}_{\mathrm{m}}
    \end{aligned}$
\item В сильнопеременных  \par
$
\begin{aligned}
  \rho_{\mathrm{int}} &= - \div \v P \\
  \v j_{\mathrm{int}} &= \pder{\v P}{t}  + c \rot \v M
\end{aligned}
$
\item $\v D = \v E + 4 \pi \v P$, $\v H = \v B - 4 \pi \v M$
\item Уравнения Максвелла в среде: \par
  $ \begin{aligned}
        \div \v D &= 4 \pi \rho_{ex} \\
        \rot \v H &= \frac 1 c \, \pder {\v D} t + \frac{4\pi}{c} \,
        (\v j_{ex} + \v j _{c})           
      \end{aligned}$
\item Материальные уравнения (простейшие)\par
  $\v D = \varepsilon \v E,\;\v B = \mu \v H ,\;\v j_c = \sigma \v E$
\item Дисперсия, варианты \par
  $ \begin{aligned}
    \v D (\v r,t) &= \int_{-\infty}^t f (t' - t, \v r)\,\v E(\v r, t')\, \del t \\
    \v D (\v r,t) &= \int_{\Delta V} g(\v r' -\v r, t)\,\v E(\v r, t')\, \del V
  \end{aligned} $ \par
$f,g$~--- функция отклика.
\end{enumerate}


\paragraph{Энергетические соотношения}
$
  \begin{aligned}
    w &= \frac{1}{8 \pi} \, (\varepsilon E^2 + \mu H^2) \\
    \v S &= \frac{c}{4 \pi} \, \v E \times \v H \\
    \pder{w}{t} + \div \v S &= - \sigma E^2 - \v E \cdot \v j_{ex}
  \end{aligned}
$ \par
Так что если внешние силы не совершают работы, 
энергия лишь убывает (за счёт выделения тепла).

\paragraph{Потенциал}
\begin{enumerate}
  \item Вид потенциала:
    $ \v E = - \frac 1 c \pder{\v A} t  - \nabla \varphi$,
    $\v B = \rot \v A$
  \item Калибровочная инвариантность:
    $ \left\{\begin{aligned}
      \v A' &= \v A - \nabla \chi \\
      \varphi' &= \varphi + \frac{1}{c} \, \pder{\chi}{t}
  \end{aligned}\right.$
  \item Калибровка  Лоренца: 
      $\frac{\varepsilon \mu}{c} \, \pder{\varphi}t + \div \v A = 0$%
      \note{при этом подходят все $\chi \that \waveop \chi = 0$}
    \item Уравнения Максвелла примут вид: 
      $
      \begin{aligned}
        \waveop \varphi &= \frac{4 \pi}{\varepsilon}\, \rho, \\
        \waveop \v A &= \frac{4 \pi \mu}{c} \, \v j, 
        \text{ где }\waveop = \frac{1}{v^2} \, \pder[2]{}{t} - \nabla,
        v = \frac{c}{\sqrt{\varepsilon \mu}} 
      \end{aligned}
      $
\end{enumerate}
\paragraph{Волновые уравнения}
$
\begin{aligned}
\waveop \v E = 0, \; \waveop \v B  =0 \\
\waveop \v A =0, \; \waveop \varphi  =0  \qquad (\waveop \chi = 0 )
\end{aligned}
$

Ещё можно $\varphi$ занулить, выбрав нужную $\chi$
\note{В предыдущем нельзя, может не оказаться решением}

\paragraph{Плоские и сферические волны}
\begin{enumerate}
  \item Одномерное волновое уравнение и его решение:
    $
    \begin{aligned}
    \frac 1 {v^2} \, \pder[2]{u}{t} - \pder[2]{u}{x} = 0 \\
    u = f(x-vt) + g(x+vt)
    \end{aligned}
    $
  \item Плоская волна: $A =  A(\v n \cdot \v r - vt)$\note{вторую волну выкинули,
    нам обычно хватает какого-то частного решения.}
  \item Условие поперечности: $\div \v A = 0$ $\Rightarrow$ $\v B = \frac cv \,\v n \times \v E$ 
  \item $\v S = v \, w \, \v n$.
  \item Уравнение сферической волны: $\frac{1}{v^2}\,\pder[2]{u}{t} - \Delta_r u = 0$
  \item Его решение: $u(r,t) = \frac{1}{r}\bigl(f(r-vt) + g(r+vt)\bigr) $
    Если рассматривать монохроматические волны, произвольные функции станут выражаться
    через функции Бесселя.
\end{enumerate}
\paragraph{Монохроматические волны}
$
\begin{aligned}
  u &\propto \cos(-\omega t + \alpha) \\
  \Delta u + \frac{\omega^2}{v^2} u &= 0, \; \v k = \frac \omega v\, \v n
  \;\Rightarrow\; u = \Re\left(\v E_0 \, e^{i\,(\v k \cdot \v r - \omega t)}\right)\\
\end{aligned}
$
\paragraph{Поляризация монохроматической волны (общий случай)}
\begin{enumerate}
  \item $\alpha, \v b$ \\
    $
    \begin{aligned}
      \alpha &\that \v E_0^2 = |E_0^2| \, e^{-2i \varphi_0 } \\
      \v b   &\that \v E_0 = \v b \,e ^{-i \varphi_0 },\;\v b^2 = |E_0^2|,\;\v b = \v b_1 + i\,\v b_2
    \end{aligned}
    $
  \item $b^2 \in \R \Rightarrow \v b_1 \perp \v b_2$
  \item $\frac{(\v E \cdot \v b_1)^2}{b_1^2} + \frac{(\v E \cdot \v b_2)^2}{b_2^2} = 1$,
    $(\v E \in \R^3)$.
\end{enumerate}
\paragraph{Почти монохроматические волны}
$\v E = \v E_0 (t) \, e^{-i \omega t}$, \quest
\paragraph{Поляризационная матрица, параметры Стокса}
$|\averg{\v S}| = \frac{\varepsilon v}{8 \pi} \, \averg{\v E^\dagger \v E} $
$\rho = \frac{\varepsilon v}{8 \pi} \, \averg{\v E \v E^\dagger} =  \begin{pmatrix}
  \averg{|E_x|^2} & \averg{E_x E_y^*} \\
  \averg{E_y E_x^*} & \averg{ |E_y|^2} \\
\end{pmatrix}=\frac 12  \begin{pmatrix}
  I+Q & U-iV \\
  U+iV & I-Q \\
\end{pmatrix}
$
\begin{enumerate}
  \item $\det \rho = I^2 - Q^2 - U^2 - V^2$
  \item $\det \rho = 0 \Leftrightarrow E_x^0 \propto E_y^0$ \note{В поляризационной матрице 
    все $E$ можно позаменять на $E^0$ (фазы всё равно сокращаются), а в предпредыдущем пунке
  у нас как раз $E^0_x = b_1\, e^{-i \varphi_0 }$, $E^0_y = ib_2\, e^{-i \varphi_0 }$}
\item $I^2, V^2, U^2+Q^2$~--- инварианты \note{Отсюда, кстати, очевидно преобразование параметров
  Стокса при поворотах}
\item $I(\psi, \delta) = \averg{|\v S|} = \v \ell_\delta^\dagger\, \rho \, \v \ell_\delta
  = \frac12 (I + Q\, \cos 2\psi + U \, \sin 2\psi \, \cos \delta - V \, \sin 2\psi \, \sin \delta)$,
  $\v \ell_\delta = (\cos \psi,\; \sin \psi \, e^{-i \delta}) ^ \top$,
  а вот выводится это неприятно.
\end{enumerate}
\paragraph{Частные случаи поляризации, параметры поляризации}
$
\begin{aligned}
  I &= \frac{\varepsilon v}{8 \pi} \, (\averg{|E_x|^2} +  \averg{|E_y|^2}) = \averg{|\v S|} \\
  Q &= \frac{\varepsilon v}{8 \pi} \, (\averg{|E_x|^2} -  \averg{|E_y|^2}) \\
  U &= \frac{\varepsilon v}{8 \pi} \, (\averg{E_x^* E_y} +  \averg{E_xE_y^*}) 
  = \frac{\varepsilon v}{8 \pi} \, 2 \Re \averg{E_x^* E_y} \\
  V &= \frac{\varepsilon v}{8 \pi} \,i (\averg{E_x^* E_y} -  \averg{E_xE_y^*}) 
  = \frac{\varepsilon v}{8 \pi} \, 2 \Im \averg{E_x^* E_y} \\
\end{aligned}
$
\begin{enumerate}
  \item $Q=U=V=0$~"--- белый свет
  \item $\det \rho = 0$~"--- эллиптическая поляризация
\begin{enumerate}
  \item $Q=U=0$~"--- круговая поляризация
  \item $V=0$~"--- линейная поляризация
\end{enumerate}

Ещё всякие величины:
\begin{itemize}
  \item $R_d^2 = Q^2 + U^2 + V^2$, $r_d^2 = Q^2 + U^2$
  \item $P = \lfrac{R_d}{I}$~--- степень поляризации
  \item $p = \lfrac{r_d}{I}$~--- степень линейной поляризации
  \item $p_s = \lfrac{V}{I}$~--- степень круговой поляризации
  \item $\tg 2 \alpha = \lfrac{U}{D}$, $\alpha$~--- угол между базисом и осями эллипса.
\end{itemize}

\item Частичная поляризация:  
\begin{itemize}
  \item белый свет + эллитическая
  \item сумма 2 ортогональных эллиптических
\end{itemize}
\end{enumerate}
\paragraph{Геометрическая оптика} 
%% TODO: \text{} and duplicated endnotes
$\begin{aligned}
  u &= u_{0} e^{i \psi}, \; \psi \hbox{~"--- эйконал\note{$\psi_1$~"--- то, что названо эйконалом
  у Бутикова. Вроде у него правильнее, но \quest}}\\
  \frac{1}{v^2} \left(\pder{\psi}t\right)^2 - (\nabla \psi)^2 &= 0 \\
  \psi = - \omega t + \frac{\omega}{c} \psi_1, \; (\nabla \psi_1)^2 &= n^2(\v r) \text{~"--- уравнение эйконала.} \\
  \frac{\omega}{c}\, \psi_1 - \omega t &= \mathrm{const} \text{~"--- волновая поверхность}
\end{aligned}$

Здесь торжественно забили на вторые прозводные эйконала.
      
\paragraph{Гадость в неоднородной среде}
\begin{enumerate}
  \item $\varepsilon = \varepsilon(r)$, $\mu = 1$
  \item Волновые уравнения поменяются:\par
    $
    \begin{aligned}
      \waveop \v E &- \nabla \bigl(\v E \cdot  \nabla (\ln \varepsilon)\bigr) = 0 \\
      \waveop \v H &- \nabla (\ln \varepsilon) \times \rot H  = 0 
    \end{aligned}
    $
  \item Монохроматический случай:\par
    $
    \begin{aligned}
      [\Delta + k^2(r)] \,&\v E + \nabla \bigl(\v E \cdot  \nabla (\ln \varepsilon)\bigr) = 0 \\
      [\Delta + k^2(r)] \,&\v H + \nabla (\ln \varepsilon) \times \rot H  = 0 
    \end{aligned}
    $
\end{enumerate}
\paragraph{E,H-волны}
$\varepsilon = \varepsilon(z)$
\begin{enumerate}
  \item $\v E \coori \mathrm{Oy},\; E = (0,1,0)\, E(z) e^{i \varkappa x}$~--- E-волны
  \item $\v H \coori \mathrm{Oy},\; H = (0,1,0)\, H(z) e^{i \varkappa x}$~--- H-волны
\end{enumerate}
Если переписать волновое уравнение выше:
\begin{enumerate}
  \item $E''(z) + f(z) \, E(z) = 0$, $f(z) = k^2 - \varkappa^2$
  \item $w''(z) + f(z) \, w(z) = 0$, $H(z) = \sqrt{\varepsilon(z)}\, w(z)$,
    $f(z) = k^2 - \varkappa^2
    + \frac 12\, \frac{\varepsilon''}{\varepsilon} 
    - \frac 34\, \left(\frac{\varepsilon'}{\varepsilon}\right)$
\end{enumerate}
\paragraph{Метод ВКБ}
Метод решения таких уравнений: $\frac 1{s^2} u'' + f\, u = 0$, $\lfrac 1{s^2}$~--- малый параметр.
\begin{enumerate}
  \item $z = s\,\tau$, $u = e^{is \psi}$ 
  \item В ряд его: $\psi = \psi_0 + \frac is\,\psi_1 + \dotsb $
  \item ВКБ-решения (первое приближение)
    $
    \begin{aligned}
      u_{1,2} &= f^{-\lfrac 14}\, \exp \left({\pm is\int\sqrt{f}\, \del \tau}\right)\\
      u       &= c_1 u_1 + c_2 u_2
    \end{aligned}
    $
  \item Условия применимости \quest:\\ 
    $
    \left\lvert \fder{\psi_0}{\tau} \right\rvert^2 \!\!\gg
    \frac{1}{s} \left\lvert\fder[2]{\psi_0}{\tau}\right\rvert
    \Leftrightarrow 
    |f| \gg \frac 1s \left\lvert \frac{f'}{2\sqrt f} \right\rvert
    \Leftrightarrow 
    \left\lvert \fder{{\sqrt \frac 1f}}{z} \right\rvert \ll 1
    $
\end{enumerate}
Для предыдущего параграфа просто $ \lambda = \frac{2 \pi}{\sqrt{f}}$

\paragraph{Диспергирующая среда, частотная и пространственная дисперсия}
Если пространство однородно (и по времени): \\
$\begin{aligned}
  \v D (\v r,t) &= \int_{-\infty}^t f (t' - t, \v r)\,\v E(\v r, t')\, \del t \\
  \v D (\v r,t) &= \int_{\Delta V} g(\v r' -\v r, t)\,\v E(\v r, t')\, \del V
\end{aligned}$\par\vspace{1ex}
Для монохроматических можно сказать  чуть  больше:
\begin{itemize}
  \item $\v D(\v r,t) = \varepsilon (\omega,\v k) E(\v r,t)$
  \item $\varepsilon = \varepsilon(\omega)$~--- частотная дисперсия
  \item $\varepsilon = \varepsilon(\v k)$~--- пространственная дисперсия
  \item $\varepsilon = \varepsilon_1 + i \varepsilon _2$, 
    $\varepsilon_1(-\omega) = \varepsilon_1(\omega)$, $\varepsilon_2(-\omega) = -\varepsilon_2(\omega)$,
    $\omega \to \infty \quad \varepsilon(\omega) \to 0$
\end{itemize}

\paragraph{Что-то про преобразование Фурье}
\begin{itemize}
  \item $\ov~f(\omega) = \intR u(t)\, e^{-i \omega t} \, \del t$
  \item $2 \pi \delta(\omega) = \intR e^{i \omega t} \, \del t$
  \item $\ov~{f\ast g} = \ov~{f} \cdot \ov~{\vphantom{f}g}$
\end{itemize}

\paragraph{Материальные уравнения для быстропеременных процессов}
\begin{itemize}
  \item $\v D(\omega) = \varepsilon(\omega) \, \v E(\omega)$
  \item $\v B(\omega) = \varepsilon(\omega) \, \v H(\omega)$
  \item $\varepsilon (\omega) = 1 - \frac{4 \pi N e^2}{m \omega^2}$
  \item $\mu \sim 1$
  \item \underdev
\end{itemize}

\paragraph{Энергетические соотношения при дисперсии}
$\div \v S = - \frac 1 {4\pi} \, 
    \left(\v E \cdot \pder{\v D}{t} + \v H \cdot \pder{\v B}{t}\right)$

Для монохроматических волн:
\begin{itemize}
  \item $\v D = \varepsilon(\omega) \, \v E$, $\v B = \mu(\omega) \, \v H$
  \item $\varepsilon(\omega) = \varepsilon_1 + i \varepsilon_2$, $\mu(\omega) = \mu_1 + i \mu_2$
  \item $\averg{\div \v S} = \frac{-\omega}{8\pi}\, 
    \left(\varepsilon_2 \averg{|\v E|^2} + \mu_2 \averg{|\v H|^2}\right)$ 
    $ \Rightarrow \varepsilon_2 > 0, \mu_2 > 0$ \quest\note{Тут непонятно что с плотностью
    энергии. Но, вроде, если амплитуда сохраняется и колебания гармонические, 
  то $\pder[2]{\averg{w}}{t}=0$. }
\end{itemize}
$\{\varepsilon, \mu \}_2 \ll \{\varepsilon, \mu\}_1$~--- прозрачная среда. Тогда можно ввести 
плотность энергии, как-то так:
\begin{enumerate}
  \item припомнить $\div \v S$
  \item первый член: $\frac{1}{16 \pi} \left(\v E \,\pder{\v D^*}{t} + \v E^* \, \pder{\v D}{t}\right)$
  \item $\pder{\v D}{t} = -i \omega \varepsilon (\omega) \v E + \fder{(\omega \varepsilon)}{\omega}
    \, \pder{\v E^0}{t} \, e^{-i \omega t}$
  \item $\div \v S = - \pder{w}{t}$
  \item $\averg{w} = \frac{1}{8\pi} 
    \left( \fder{(\omega \varepsilon)}{\omega} \, \averg{|\v E^0|^2} + 
    \fder{(\omega \mu)}{\omega} \, \averg{|\v H^0|^2} \right)$
\end{enumerate}

\paragraph{Волны [монохроматические] в диспергирующей среде\note{Бардак в конспекте, писал по Бутикову}}
Здесь
$k := \sqrt{\varepsilon(\omega) \, \mu(\omega)} \, \frac{\omega}{c} = \v k_1 + i\v k_2$,
$\{\varepsilon, \mu \}_2 \ll \{\varepsilon, \mu\}_1$.
\begin{description}
  \item[$\v k_1 \nparallel \v k_2$] Неоднородная плоская волна:
    $\v E = \v E_0 \, e^{-\v k_2\cdot \v r}\, e^{i(\v k_1 \cdot \v r - \omega t)}$
  \item[$\v k_1 \coori \v k_2$] Однородная плоская волна: 
    \begin{enumerate}
      \item $k = (n + i \varkappa)\, \lfrac{\omega}{c}$~--- показатель преломления и затухания,
      \item $E(z,t) = E_0 \, e^{-\varkappa \omega \lfrac zc} \, e^{-i \omega (t - n \lfrac z/c )}$
      \item $\averg{S(z)} = S_0 \, e^{-2 \varkappa \omega \,\lfrac z c} = S_0 e^{-\alpha z}$,
        $\alpha$~--- к-т поглошения.
  \end{enumerate}
\end{description}

\paragraph{Групповая скорость}

\begin{enumerate}
  \item $v_{\mathrm{gr}} = \frac{\averg{|\v S|}}{\averg{w}} = \frac{c}{\fder{n \omega}{\omega}}$
  \item $v_{\mathrm{gr}} = \frac{1}{\fder k \omega} = \frac{c}{\fder{n \omega}{\omega}}$
\end{enumerate}
Отсюда $v_{\mathrm{gr}} = v_\phi \cdot \frac{1}{1 + \frac \omega n\, \fder{n}{\omega}}$
\begin{itemize}
  \item $\fder{n}{\omega} > 0$~--- нормальная дисперсия, $v_{\mathrm{gr}} < v_\phi$
  \item $\fder{n}{\omega} < 0$~--- аномальная дисперсия, $v_{\mathrm{gr}} > v_\phi$
\end{itemize}

\end{multicols*}
\end{document}
