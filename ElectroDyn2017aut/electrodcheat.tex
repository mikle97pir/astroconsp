\documentclass[draft]{trchesh}
\usepackage{trmath}
\usepackage{trsym} 

\setlayout[page-orient=landscape]{hardcopy}
\let\note=\endnote
\hypersetup{
  pdftitle={Шпора по электроду},
  pdfauthor={taxus},
  pdfsubject={Электродинамика},
  pdfkeywords={Электродинамика;СПбГУ;by-taxus}
}
\columnseprule=0.1dd
\def\arraystretch{1.3}
\setlist[enumerate,1]{leftmargin=2em}

\def\waveop{\mathop{\boldsymbol\Box}}
\begin{document}

\begin{multicols*}{3}\raggedright
\parindent=0pt
\paragraph{Уравнения Максвелла}
\begin{enumerate}
  \item Теорема Гаусса: $\oint \v E \cdot \del \v s = 4 \pi Q$.
  \item Закон Фарадея: $\oint\v E \cdot \del \v l = - \frac{1}{c} \pder{\Phi}{t}$, \\
    $\Phi=\int \v H \cdot \del \v s$
  \item Закон Био-Савара-Лапласа: $\v H = \frac{1}{c} \, \frac{\v j \times \v R}{R^3}$
  \item $\oint \v H \cdot \del \v s = 0$
  \item Закон Ампера: $\oint \v H \cdot \del \v l = \frac{4 \pi}{c} \int \v j \cdot \del \v s$
  \item Уравнение неразрывности: $\pder \rho t  + \div \v j = 0 $
  \item Сами уравения Максвелла: \par\vspace{1ex}$
    \begin{aligned}
      &\div \v E &&=&& 4 \pi \rho \\
      &\div \v H &&=&& 0 \\
      &\rot \v E &&=&& - \frac 1 c \, \pder{\v H}{t} \\
      &\rot \v H &&=&& \frac 1 c\, \pder{\v E}{t} + \frac{4 \pi}{c}\v j
    \end{aligned}$
\end{enumerate}

\paragraph{В среде}
\begin{enumerate}
  \item Поляризация и намагниченность \\
    $\begin{aligned}
    \v P \that\v j_{\mathrm{pol}} &=\pder{\v P}t,\;\rho_{\mathrm{pol}} =-\div \v P,\\
    \v M \that \v j_{\mathrm{m}} &= c \rot \v M \\
    \{\rho, \v j\}_{\mathrm{int}} &= \{\rho, \v j\}_{\mathrm{pol}} + \{\rho, \v j\}_{\mathrm{m}}
    \end{aligned}$
\item В сильнопеременных  \par
$
\begin{aligned}
  \rho_{\mathrm{int}} &= - \div \v P \\
  \v j_{\mathrm{int}} &= \pder{\v P}{t}  + c \rot \v M
\end{aligned}
$
\item $\v D = \v E + 4 \pi \v P$, $\v H = \v B - 4 \pi \v M$
\item Уравнения Максвелла в среде: \par
  $ \begin{aligned}
        \div \v D &= 4 \pi \rho_{ex} \\
        \rot \v H &= \frac 1 c \, \pder {\v D} t + \frac{4\pi}{c} \,
        (\v j_{ex} + \v j _{c})           
      \end{aligned}$
\item Материальные уравнения (простейшие)\par
  $\v D = \varepsilon \v E,\;\v B = \mu \v H ,\;\v j_c = \sigma \v E$
\item Дисперсия, варианты \par
  $ \begin{aligned}
    \v D (\v r,t) &= \int_{-\infty}^t f (t' - t, \v r)\,\v E(\v r, t')\, \del t \\
    \v D (\v r,t) &= \int_{\Delta V} g(\v r' -\v r, t)\,\v E(\v r, t')\, \del V
  \end{aligned} $
\end{enumerate}


\paragraph{Энергетические соотношения}
$
  \begin{aligned}
    w &= \frac{1}{8 \pi} \, (\varepsilon E^2 + \mu H^2) \\
    \v B &= \frac{c}{4 \pi} \, \v E \times \v H \\
    \pder{w}{t} + \div \v S &= - \sigma E^2 - \v E \cdot \v j_{ex}
  \end{aligned}
$ \par
Так что если внешние силы не совершают работы, 
энергия лишь убывает (за счёт выделения тепла).

\paragraph{Потенциал}
\begin{enumerate}
  \item Вид потенциала:
    $ \v E = - \frac 1 c \pder{\v A} t  - \nabla \varphi$,
    $\v B = \rot \v A$
  \item Калибровочная инвариантность:
    $ \left\{\begin{aligned}
      \v A' &= \v A - \nabla \chi \\
      \varphi' &= \varphi + \frac{1}{c} \, \pder{\chi}{t}
  \end{aligned}\right.$
  \item Калибровка  Лоренца: 
      $\frac{\varepsilon \mu}{c} \, \pder{\varphi}t + \div \v A = 0$%
      \note{при этом подходят все $\chi \that \waveop \chi = 0$}
    \item Уравнения Максвелла примут вид: 
      $
      \begin{aligned}
        \waveop \varphi &= \frac{4 \pi}{\varepsilon}\, \rho, \\
        \waveop \v A &= \frac{4 \pi \mu}{c} \, \v j, 
        \text{ где }\waveop = \frac{1}{v^2} \, \pder[2]{}{t} - \nabla,
        v = \frac{c}{\sqrt{\varepsilon \mu}} 
      \end{aligned}
      $
\end{enumerate}
\paragraph{Волновые уравнения}
$
\begin{aligned}
\waveop \v E = 0, \; \waveop \v B  =0 \\
\waveop \v A =0, \; \waveop \varphi  =0  \qquad (\waveop \chi = 0 )
\end{aligned}
$

Ещё можно $\varphi$ занулить, выбрав нужную $\chi$%
\note{В предыдущем нельзя, может не оказаться решением}

\paragraph{Плоские и сферические волны}
\begin{enumerate}
  \item Одномерное волновое уравнение и его решение:
    $
    \begin{aligned}
    \frac 1 {v^2} \, \pder[2]{u}{t} - \pder[2]{u}{x} = 0 \\
    u = f(x-vt) + g(x+vt)
    \end{aligned}
    $
  \item Плоская волна: $A =  A(\v n \cdot \v r - vt)$\note{вторую волну выкинули,
    нам обычно хватает какого-то частного решения.}
  \item Условие поперечности: $\div \v A = 0$ $\Rightarrow$ $\v B = \frac cv \,\v n \times \v E$ 
  \item $\v S = v \, w \, \v n$.
  \item Уравнение сферической волны: $\frac{1}{v^2}\,\pder[2]{u}{t} - \Delta_r u = 0$
  \item Его решение: $u(r,t) = \frac{1}{r}\bigl(f(r-vt) + g(r+vt)\bigr) $
    Если рассматривать монохроматические волны, произвольные функции станут выражаться
    через функции Бесселя.
\end{enumerate}
\paragraph{Монохроматические волны}
$
\begin{aligned}
  u &\propto \cos(-\omega t + \alpha) \\
  \Delta u + \frac{\omega^2}{v^2} u &= 0, \; \v k = \frac \omega v\, \v n
  \;\Rightarrow\; u = \Re\left(\v E_0 \, e^{i\,(\v k \cdot \v r - \omega t)}\right)\\
\end{aligned}
$
\paragraph{Поляризация монохроматической волны (общий случай)}
\begin{enumerate}
  \item $\alpha, \v b$ \\
    $
    \begin{aligned}
      \alpha &\that \v E_0^2 = |E_0^2| \, e^{-2i \alpha} \\
      \v b   &\that \v E_0 = \v b \,e ^{i \alpha},\; \v b^2 = |E_0^2|, \; \v b = \v b_1 + i\,\v b_2
    \end{aligned}
    $
  \item $b^2 \in \R \Rightarrow \v b_1 \perp \v b_2$
  \item $\frac{(\v E \cdot \v b_1)^2}{b_1^2} + \frac{(\v E \cdot \v b_2)^2}{b_2^2} = 1$,
    $(\v E \in \R^3)$.
\end{enumerate}
\paragraph{Почти монохроматические волны}
$\v E = \v E_0 (t) \, e^{-i \omega t}$, \quest
\paragraph{Поляризационная матрица, параметры Стокса}
      $\rho = \begin{pmatrix}
        \averg{|E_x|^2} & \averg{E_x E_y^*} \\
        \averg{E_y E_x^*} & \averg{ |E_y|^2} \\
      \end{pmatrix}=\frac 12 \, \begin{pmatrix}
        I+Q & U-iV \\
        U+iV & I-Q \\
      \end{pmatrix}
      $
      \paragraph{Частные случаи поляризации}
      \begin{enumerate}
        \item $Q=U=V=0$~"--- белый свет
        \item $Q=U=0$~"--- круговая поляризация
        \item $V=0$~"--- линейная поляризация
      \end{enumerate}
      \paragraph{ Геометрическая оптика} 
      $\begin{aligned}
        u = u_{0} e^{i \psi} \\
        \frac{1}{v^2} \left(\pder{\psi}t\right)^2 - (\nabla \psi)^2 = 0\\
        \psi = - \omega t + \psi_1, \; (\nabla \psi_1)^2 = n^2(\v r)
      \end{aligned}$
      \paragraph{Гадость в неоднородной среде}
      \paragraph{\(E,H\)-волны}
      $E''(z) + f(z) \, E(z) = 0$, $f(z) = k^2 - \varkappa^2$
\end{multicols*}
\end{document}
