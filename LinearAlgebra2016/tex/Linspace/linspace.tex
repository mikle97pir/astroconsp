\documentclass[12pt]{../../notes}
\usepackage{silence}
\WarningFilter{latex}{Reference}
\graphicspath{{../../img/}}

\begin{document}

\setcounter{paragraph}{0}
\paragraph{Определения}
\begin{defn}\label{defn:linspace}
  Пусть $K$~--- поле. Рассмотрим множество  $V$ с двумя операциями
  \begin{align*}  
    + &: V \times V \to V \\
    \cdot &: K\times V \to V
  \end{align*}
  Тогда $V$~--- линейное пространство над $K$, если 
  $\forall\, \mathbf{x},\mathbf{y},\mathbf{z}\in V,\; \alpha_i \in K$
  \begin{enumerate}
    \item $(\mathbf{x}+\mathbf{y}) + \mathbf{z} = \mathbf{x} + (\mathbf{y}+\mathbf{z})$
    \item $\mathbf{x} + \mathbf{y} = \mathbf{y} + \mathbf{z} $
    \item $\exists\, \mathbf{0}\in V : \mathbf{x} + \mathbf{0} = \mathbf{x}$
    \item $\exists\, (-\mathbf{x})\in V : \mathbf{x} + (-\mathbf{x}) = \mathbf{0}$
    \item $(\alpha_1+\alpha_2) \mathbf{x} = \alpha_1 \mathbf{x} + \alpha_2 \mathbf{x}$
    \item $\alpha (\mathbf{x}_1 + \mathbf{x}_2) = \alpha \mathbf{x}_1 + \alpha \mathbf{x}_2$
    \item $1 \cdot \mathbf{x} = \mathbf{x}$
    \item $(\alpha_1 \alpha_2) \mathbf{x} = \alpha_1 (\alpha_2 \mathbf{x})$
  \end{enumerate}
\end{defn}

{ \defn\label{defn:subspace}
  Пусть $U,V$~--- линейные пространства над $K$, $U \subset V$. Тогда $U$~--- подпространство 
  $V$.
}

{ \defn\label{defn:lincomb}
Пусть $V$~--- линейное пространства над $K$, $\mathbf{x}_1, \dotsc, \mathbf{x}_n\in~V$, 
$\alpha_1, \dotsc, \alpha_n \in K$. Тогда $\alpha_1 \mathbf{x}_1 + \dotsb + \alpha_n \mathbf{x}_n$~--- 
линейная комбинация $\mathbf{x}_1, \dotsc, \mathbf{x}_n$.
}


\begin{lem}\label{lem:linspsign}
  Пусть $U,V$~--- линейные пространства над $K$, $U \subset V$. Тогда если $U$ замкнуто относительно
  $+, \cdot$  из  $V$, то  $U$~--- подпространство
  $V$.
\end{lem}
\begin{itlproof}
  Формулировка леммы аналогична тому, что всякая линейная комбинация элементов $U$ лежит в нем же.
  Нужная дистрибутивность, ассоциативность и т.д. унаследуется от соответствующих операций в 
  надпространстве, так как их свойства заданы на всём множестве $V$, а значит и на подмножестве $U$.
  Однако в некоторых свойствах требовалось существование в множестве чего-нибудь.
  Покажем, что все эти требования равносильны существованию линейной комбинации.
  \begin{enumerate}
      \setcounter{enumi}{2}
    \item $\exists\, \mathbf{0}\in U \Leftarrow \exists\, 0\cdot \mathbf{x},\; \mathbf{x}\in U$
    \item $\exists\, \mathbf{-x}\in U \Leftarrow \exists\, (-1)\cdot \mathbf{x},\; \mathbf{x}\in U$
  \end{enumerate}
\end{itlproof}

{ \defn\label{defn:linshell}
Пусть $V$~--- линейное пространства над $K$, $M \subset V$
\[
  \langle M \rangle = 
  \left\{ \alpha_1 \mathbf{x}_1 + \dotsb + \alpha_n \mathbf{x}_n \middle|
  \Big\{ \begin{array}{l}
    \alpha_1, \dotsc, \alpha_n \in K\\
    \mathbf{x}_1, \dotsc, \mathbf{x}_n \in M
  \end{array} \right\}
\] 
$\langle M \rangle$~--- линейная оболочка $M$.
}
\begin{lem}\label{lem:linshell}
  Верны утверждения:
  \begin{enumerate}
    \item $\langle M \rangle$~--- подпространство $V$
    \item $ \langle M \rangle = \bigcap\limits_i W_i$, $W_i \supset M$, $W_i$~--- подпространство $V$
  \end{enumerate}
\end{lem}
\begin{itlproof}
  Доказательства очень похожи на соответствующие в теории групп.
\end{itlproof}

\paragraph{Линейная независимость системы векторов}

{\defn\label{defn:linindp} Пусть $\mathbf{x}_1,\dotsc, \mathbf{x}_n \in V$, 
$\alpha_1, \dotsc, \alpha_n \in K$.
Тогда если 
\[
  \alpha_1\,\mathbf{x}_1 + \dotsb + \alpha_n\,\mathbf{x}_n = 0 \Leftrightarrow \forall\,i\;\alpha_i = 0
\]
то система векторов $\mathbf{x}_1, \dotsc, \mathbf{x}_n$ линейно независима.
}
\end{document}
