\documentclass[12pt]{../../../notes}
\usepackage{silence}
\WarningFilter{latex}{Reference}
\graphicspath{{../../img/}}

\begin{document}

\setcounter{paragraph}{0}
\paragraph{Определения}
\begin{defn}\label{defn:linspace}
  Пусть $K$"--- поле. Рассмотрим множество  $V$ с двумя операциями
  \begin{align*}  
    + &: V \times V \to V \\
    \cdot &: K\times V \to V
  \end{align*}
  Тогда $V$"--- линейное пространство над $K$, если 
  $\forall\, \mathbf{x},\mathbf{y},\mathbf{z}\in V,\; \alpha_i \in K$
  \begin{enumerate}
    \item $(\mathbf{x}+\mathbf{y}) + \mathbf{z} = \mathbf{x} + (\mathbf{y}+\mathbf{z})$
    \item $\mathbf{x} + \mathbf{y} = \mathbf{y} + \mathbf{z} $
    \item $\exists\, \mathbf{0}\in V : \mathbf{x} + \mathbf{0} = \mathbf{x}$
    \item $\exists\, (-\mathbf{x})\in V : \mathbf{x} + (-\mathbf{x}) = \mathbf{0}$
    \item $(\alpha_1+\alpha_2) \mathbf{x} = \alpha_1 \mathbf{x} + \alpha_2 \mathbf{x}$
    \item $\alpha (\mathbf{x}_1 + \mathbf{x}_2) = \alpha \mathbf{x}_1 + \alpha \mathbf{x}_2$
    \item $1 \cdot \mathbf{x} = \mathbf{x}$
    \item $(\alpha_1 \alpha_2) \mathbf{x} = \alpha_1 (\alpha_2 \mathbf{x})$
  \end{enumerate}
\end{defn}

{ \defn\label{defn:linsubspace}
  Пусть $U,V$"--- линейные пространства над $K$, $U \subset V$. Тогда $U$"--- подпространство 
  $V$.
}

{ \defn\label{defn:lincomb}
Пусть $V$"--- линейное пространства над $K$, $\mathbf{x}_1, \dotsc, \mathbf{x}_n\in~V$, 
$\alpha_1, \dotsc, \alpha_n \in K$. Тогда $\alpha_1 \mathbf{x}_1 + \dotsb + \alpha_n \mathbf{x}_n$"--- 
линейная комбинация $\mathbf{x}_1, \dotsc, \mathbf{x}_n$.
}


\begin{lem}\label{lem:linspsign}
  Пусть $U,V$"--- линейные пространства над $K$, $U \subset V$. Тогда если $U$ замкнуто относительно
  $+, \cdot$  из  $V$, то  $U$"--- подпространство
  $V$.
\end{lem}
\begin{itlproof}
  Формулировка леммы аналогична тому, что всякая линейная комбинация элементов $U$ лежит в нём же.
  Нужная дистрибутивность, ассоциативность и т.д. унаследуется от соответствующих операций в 
  надпространстве, так как их свойства заданы на всём множестве $V$, а значит и на подмножестве $U$.
  Однако в некоторых свойствах требовалось существование в множестве чего-нибудь.
  Покажем, что все эти требования равносильны существованию линейной комбинации.
  \begin{enumerate}
      \setcounter{enumi}{2}
    \item $\exists\, \mathbf{0}\in U \Leftarrow \exists\, 0\cdot \mathbf{x},\; \mathbf{x}\in U$
    \item $\exists\, \mathbf{-x}\in U \Leftarrow \exists\, (-1)\cdot \mathbf{x},\; \mathbf{x}\in U$
  \end{enumerate}
\end{itlproof}

{ \defn\label{defn:linshell}
Пусть $V$"--- линейное пространства над $K$, $M \subset V$
\[
  \langle M \rangle = 
  \left\{ \alpha_1 \mathbf{x}_1 + \dotsb + \alpha_n \mathbf{x}_n \middle|
  \Big\{ \begin{array}{l}
    \alpha_1, \dotsc, \alpha_n \in K\\
    \mathbf{x}_1, \dotsc, \mathbf{x}_n \in M
  \end{array} \right\}
\] 
$\langle M \rangle$"--- линейная оболочка $M$.
}

\begin{lem}\label{lem:linspan}
  Верны утверждения:
  \begin{enumerate}
    \item $\langle M \rangle$"--- подпространство $V$
    \item $ \langle M \rangle = \bigcap\limits_i W_i$, $W_i \supset M$, $W_i$"--- подпространство $V$
  \end{enumerate}
\end{lem}
\begin{itlproof}
  Доказательства очень похожи на соответствующие в теории групп.
\end{itlproof}

%%% XXX так пробелы вокруг тире лучше: "---  чем так : ~---
%%%
\begin{defn}\label{defn:linspanprop}
  Пусть $V$"--- линейное пространство. Тогда
  $M \subset V$"--- порождающая система, если $\langle M\rangle = V$
\end{defn}


\paragraph{Линейная независимость системы векторов}

\begin{defn}\label{defn:linindp} Пусть $\mathbf{x}_1,\dotsc, \mathbf{x}_n \in V$, 
$\alpha_1, \dotsc, \alpha_n \in K$.
Тогда если 
\[
  \alpha_1\,\mathbf{x}_1 + \dotsb + \alpha_n\,\mathbf{x}_n = 0 \Rightarrow \forall\,i\;\alpha_i = 0
\]
то система векторов $\mathbf{x}_1, \dotsc, \mathbf{x}_n$ линейно независима.
В противном случае система линейно зависима.
\end{defn}

\subparagraph{Свойства}
\begin{enumerate}
  \item Подсистема линейно независимой системы линейно независима.
  \item В ЛНЗ системе ни один вектор не выражается через другие
\end{enumerate}

\paragraph{Лемма о линейной зависимости линейных комбинаций. Базис}
\begin{lem}[Линейная зависимость линейных комбинаций]\label{lem:ldlincomp} 
  Пусть $V = \{\mathbf{v}_1, \mathbf{v}_2,\dotsc, \mathbf{v}_n\}$~--- ЛНЗ, а 
  $U = \{\mathbf{u}_1, \dotsc , \mathbf{u}_m\}$"--- линейные комбинации векторов из $M$.
  Тогда если $m > n$, то $U$"--- линейно зависимы.  
\end{lem}

\begin{itlproof} 
  Там получается ОСЛУ, в которой уравнений больше, чем неизвестных.  Решение найдётся.
\end{itlproof}

\begin{defn}\label{defn:basis}
  Базис"--- линейно независимая (\ref{defn:linindp}), порождающая (\ref{defn:linspanprop})
  система векторов.
\end{defn}

\begin{defn}\label{defn:lindim}
  Размерность ($\dim$) линейного пространства"--- число векторов в базисе.
\end{defn}

\begin{lem}[Корректность определения размерности]\label{lem:lindimcorr}
  Пусть $\{u_i\}_{1 \leqslant i \leqslant n}, \{v_i\}_{1 \leqslant i \leqslant m}$"--- базисы
  $V$. Тогда $m=n$.
\end{lem}
\begin{itlproof}
  Иначе одна система выражается через другую и по~\ref{lem:ldlincomp} она ЛЗ, что странно.
\end{itlproof}


\paragraph{Базис в конечномерных пространствах}
Во всех следующих теоремах действие происходит в конечномерных пространствах.

\begin{thrm}\label{thrm:basisfromspan}
  Из всякой порождающей системы можно выделить базис
\end{thrm}

\begin{imp}
  Базис"--- минимальная порождающая система векторов
\end{imp}

\begin{thrm}\label{thrm:basicfrimlinind}
  Из всякой линейно независимой системы можно выделить базис.
\end{thrm}
\begin{imp}
  Базис"--- максимальная линейно независимая система
\end{imp}

\paragraph{Сумма и пересечение ЛП}
\begin{defn}\label{defn:linspsum}
  Пусть $\forall\, i \in I \;\: U_i \subset V$. Тогда 
  \[
    \sum_{i\in I} U_i := \left\{ u_{1} + \dotsb + u_{n} \mid u_{i} \in U_i\right\}
  \]
  То есть совокупность всевозможных сумм
\end{defn}
\begin{defn}\label{defn:linspintersec}
  \[
    \bigcap_{i\in I} U_i := \{u \mid \forall\, i \;\: u \in U_i\}
  \]
\end{defn}

\begin{rem*}
  Пересечение"--- подпространство.
\end{rem*}

\begin{thrm}\label{thrm:dimsumlinsp}
  Пусть $U_1, U_2 $"--- подпространства $V$.
  Тогда
  \[
    \dim (U_1+U_2) = \dim U_1 + \dim U_2 + \dim (U_1 \cap U_2)
  \]
\end{thrm}
\begin{ittproof}
  Пусть $\dim(U_1 \cap U_2) = k$, $\dim U_1 = k + l$, $\dim U_2 = k + n$. Тут мы просто дополняем
  базис пересечения до базиса пространства, умеем же по \ref{thrm:basicfrimlinind}.
  \begin{enumerate}
    \item Сначала доказываем, что $k + l + n$ нужных векторов вообще хватит, чтобы породить всё
      $U_1 + U_2$
    \item Потом доказываем, что построенный таким образом базис линейно независим
  \end{enumerate}
\end{ittproof}

\paragraph{Внутренняя прямая сумма}
\begin{defn}\label{defn:dirsum}
  Пусть $\{U_i\}_{i\in I} \subset 2^V$, $U = \sum_I U_i$. Тогда
  \[
    \left( \sum_{\substack{i\in I \\ u_i\in U_i}} u_i = 0 
    \Rightarrow \forall\, i \;\: u_i = 0\right)
    \Leftrightarrow U = \bigoplus_{i\in I} U_i
  \]
\end{defn}

\begin{lem}\label{lem:dirsumuniq}
  Элемент из прямой суммы единственным образом представляется суммой $u_i \in I_i$.
\end{lem}

\begin{thrm}[Критерий $\oplus$]\label{thrm:dirsumcrit}
  Пусть  
  \begin{align*}
    U   &= \sum_{i \in I}  U_i \\
    W_i &= \sum_{j\in I, j\neq i} U_j
  \end{align*}
  Тогда $U$"--- прямая сумма $\Leftrightarrow$ 
  \[
    \forall\, i \;\: U_i \cap W_i = \{0\}
  \]
\end{thrm}
\begin{ittproof}
  Более-менее ясно из определения прямой суммы. И вообще, похоже на свойства ЛНЗ системы
  векторов. 
\end{ittproof}

\paragraph{Размерность прямой суммы конечного числа ЛП}
\begin{thrm}\label{thrm:dimdirsum}
  \[
    \dim \underbrace{\bigoplus_{i\in I} U_i}_{V} = \sum_{i\in I} \dim U_i
  \]
\end{thrm}
\begin{ittproof}
%%Тут у Карпова индукция была, проще же можно..
(По мотивам \cite[стр.~195]{vinberg}).

  Сначала заметим, что объединение базисов~--- точно порождающая система в $V$.
  Теперь докажем, что она линейно независима.
  Пусть для начала ${e_{ij}}$~--- базис $U_i$.
  Тогда 
  \[
    V \ni v = \sum_{i,j} \alpha_{ij} e_{ij} 
  \]
  Сгруппируем
  \begin{align*}
    u_i &= \sum_j \alpha_{ij} e_{ij} \\
      v &= 0 \Leftrightarrow \sum_i u_i = 0 \Leftrightarrow \forall\, i \;\: u_i = 0
  \end{align*}
  Ну а ноль в подпространствах представляется единственным образом
  \[
    u_i = 0 \Rightarrow \forall\, j \;\: \alpha_{ij} = 0
  \]
\end{ittproof} 

\begin{stat}[Непонятно зачем нужное утверждение]\label{stat:dirsumsubfact}
  Пусть 
  \[
    V_{k+1} = \bigoplus_{i=1}^{k+1} U_i, \; V_k = \sum_{i=1}^k U_i
  \]
  Тогда
  \[
    V_k = \bigoplus_{i=1}^k U_i
  \]
\end{stat}

\paragraph{Аффинные подпространства}

\begin{defn}\label{defn:affinsubspc}
  Пусть $U$"--- подпространство $V$, $a\in V$. Тогда $W = U + a = \{x+a\mid x\in U\}$~--- аффинное 
  подпространство. 
\end{defn}

\begin{lem}\label{lem:affinfact}
  Пусть $U$"--- подпространство $V$. Тогда
  \[
    U + a = U + b \Leftrightarrow a - b \in U
  \]
\end{lem}
\begin{lem}\label{lem:affinprop}
  Пусть $V$~--- линейное пространство над $K$, $W \subset V$, $a\in V$.
  Тогда если:
  \begin{enumerate}
    \item $\forall\, \alpha \in K, x \in W \;\: \alpha(x-a) + a \in W$  
    \item $\forall\, x_1,x_2 \in W \;\: x_1 + x_2 - a \in W$  
  \end{enumerate}
  то $W$~--- аффинное подпространство
\end{lem}

\paragraph{Факторпространство}
\begin{defn}\label{defn:factorspace}
  Пусть $U$~--- подпространство линейного пространства $V$ над полем $K$.
  Тогда такая структура называется факторпространством:
  \begin{align*}
    V / U &:= \{ U + a \mid a\in  V \} \\
    \overline a &:= U+a \\
    \overline a + \overline b &:= \overline{a+b} \\
    \alpha \cdot \overline a &:= \overline{\alpha \cdot a}
  \end{align*}
\end{defn}

\begin{stat}\label{stat:corrfactspc}
  Определение~\ref{defn:factorspace} корректно
\end{stat}

\begin{stat}\label{stat:factspcISspace}
  Структура которую описали в~\ref{defn:factorspace}"--- векторное пространство.
\end{stat}

\begin{thrm}\label{thrm:dimfactspc}
  \[
    \dim (V/U) = \dim V - \dim U
  \]
\end{thrm}

\begin{defn}\label{defn:relbasis}
  Дополнение базиса $U$ до базиса $V$ называется базисом $V$ относительно $U$ (относительным
  базисом).
\end{defn}


\end{document}
