%------------------------------------------------------------
% Description : Chapter about linear operators
% Author      : tis-p30 <iliya.t@mail.ru>
% Created at  : Sat Jun 18 12:38:13 MSK 2016
%------------------------------------------------------------
\documentclass[12pt]{../../../notes}
\usepackage{silence}
\WarningFilter{latex}{Reference}
\graphicspath{{../../img/}}

\begin{document}

\setcounter{paragraph}{7}

\paragraph{Кольцо линейных операторов}

\begin{defn}\label{defn:linop::operring::linop}
  Пусть $V$~--- линейное пространство над полем $K$. Пусть также $\varphi \colon V \to V$, и
  $\varphi$~--- линейное отображение. Тогда $\varphi$~--- линейный оператор.
\end{defn}

\begin{defn}[Сложение и умножение операторов]\label{defn:linop::operring::linopaddmul}
  Введём 2 операции:
  \begin{alignat*}{3}
    +     &\colon V\times V \to V & \:&\wedge\: & (\psi + \varphi)(x) &= \psi(x) + \varphi(x) \\
    \circ &\colon V\times V \to V & \:&\wedge\: & (\psi \circ \varphi)(x) &= \psi(\varphi(x))
  \end{alignat*}
\end{defn}

\begin{thrm}\label{thrm:linop::operring::operring}
  Множество эндоморфизмов $\End(V)$ с операциями, определёнными 
  в~\ref{defn:linop::operring::linopaddmul}~--- кольцо.
\end{thrm}

\begin{thrm}\label{thrm:linop::operring::isommtx}
  Пусть $\dim V = n$. Тогда
  \[
    (\End (V),\circ,+) \cong (M_n(K), \cdot, +)
  \]
\end{thrm}
\begin{ittproof}
  Пусть $\varphi, \psi \in \End(V)$, $A, B \in M_n(K)$.
  Выберем базис в $V$ и рассмотрим отображение $f \colon \varphi \mapsto A_\varphi$,
  композиция переходит в умножение матриц, сложение~--- в сложение матриц.

  Такое отображение обратимо, действительно, 
  \[
    \forall\, A \in M_n(K) \;\: \big( \omega(x) := A x \big) \in \End(V)
  \]
  А значит $f$~--- биекция.

  Пусть в выбранном базисе $\varphi(x) =  Ax$, $\psi(x) = Bx$. 
  Уже доказывали, что матрица, соответствующая композиции $\psi\varphi$, равна $BA$.
  Теперь разберёмся с матрицей суммы
  \[
    (\varphi + \psi)(x) = \varphi(x) + \psi(x) = A x + B x = (A+B) x
  \]
  То есть, в фиксированном базисе $(\varphi+\psi)$ соответствует $A+B$.
  
  Таким образом, раз базис выбирали произвольно, то в любом базисе $V$.
  \begin{enumerate}
    \item $f$~--- биекция.
    \item $f(\varphi + \psi) = A_{\varphi+\psi} = A_\varphi + A_\psi = f(\varphi) + f(\psi)$
    \item $f(\varphi \psi) = A_{\varphi\psi} = A_\varphi \cdot A_\psi = f(\varphi) \cdot f(\psi)$
  \end{enumerate}
  Следовательно, $f$~--- изоморфизм.
\end{ittproof}

\setcounter{paragraph}{11}
\begin{defn}\label{defn:linop::poly::poly}
  Пусть $\varphi\in \End(V)$, $A$~--- матрица $\varphi$ в фиксированном базисе, $p\in K[t]$.
  Тогда
  \begin{align*}
    p(\varphi)(x) &= a_k \varphi^l(x) + \dotsb + a_1 \varphi(x) + a_0
    p(A) &= a_k A^l + \dotsb + a_1 A + a_0
  \end{align*}
\end{defn}
\begin{lem}\label{lem:linop::poly::mtxoprel}
  Если в некотором базисе $\varphi$ имеет матрицу $A$, то в том же базисе $f(\varphi)$ имеет
  матрицу $f(A)$.
\end{lem}
\begin{lem}\label{lem:linop::poly::commute}
  Многочлены от одного и того же оператора и его матрицы в фиксированном базисе коммутируют.
\end{lem}
\begin{itlproof}
  По сложению оно все коммутативно. В слагаемых переставлять нужно степени одного и того же.
  Циклические группы обычно абелевы.
\end{itlproof}

\paragraph{ Инвариантные подпространства }
\begin{defn}\label{defn:linop::invsubsp::invsubsp}
  Пусть $\varphi\in \End(V)$, $W$~--- подпространство $V$. Тогда если $\varphi(W) \subset V$,
  то $W$~--- $\varphi$-инвариантное подпространство $V$.
\end{defn}

\begin{lem}\label{lem:linop::invsubsb::dirsum}
  Пусть:
  \begin{itemize}
    \item $V = \bigoplus_{i=1}^n W_i$
    \item $W_{i}$~--- $\varphi$-инвариантное подпространство
    \item базис $V$ разбивается на базисы $W_i$.
    \item $\varphi\left.\vphantom{g_{\int}}\right|_{W_i} \in \End(V)$
    \item В фиксированном базисе $A_i, A$~--- матрицы $\varphi_i, \varphi$ соответственно. 
  \end{itemize}
  Тогда
  \[
    A = 
    \begin{pmatrix}
      \boxed{A_1} & 0   & 0      & 0    \\
      0   & \mkern -15mu \boxed{A_2} & 0      & 0    \\
      0   & 0   & \ddots & 0    \\
      0   & 0   & 0      & \boxed{A_n}  \\
    \end{pmatrix}
  \]
\end{lem}
\begin{itlproof}
  Можно рассмотреть один такой <<блок>>. Если сверху/снизу него не ноли, то с инвариантностью
  проблемы. 
\end{itlproof}

\paragraph{Характеристический многочлен оператора}
\begin{defn}\label{defn:linop::charpoly::charpoly}
  Пусть $\varphi\in \End(V)$, $A$~--- его матрица в выбранном базисе.
  Тогда
  \[
    \chi_\varphi(t) = \det (A - t E_n)
  \]
\end{defn}

\begin{stat}\label{stat:linop::charpoly::corr}
  Какой бы базис не выбрали в $V$, характеристический многочлен не изменится.
\end{stat}
\begin{itlproof}
  Матрицы оператора во всевозможных базисах подобны. Единичная матрица не поменяется при смене
  базиса. А определители подобных матриц равны.
\end{itlproof}

\subparagraph{Свойства}
\begin{enumerate}
  \item $\deg \chi_\varphi = n$
  \item Пусть $\chi_\varphi(t) = a_n t^n + \dotsb + a_0$. 
    Тогда
    \begin{align*}
      &a_n = (-1)^n \\
      &a_{n-1} = (-1)^n (a_{11} + \dotsb + a_{nn})
    \end{align*}
    {\defn\label{defn:linop::charpoly::trace} $\Tr A = a_{11} + \dotsb + a_{nn}$ }
  \item $A \sim A' \Rightarrow \Tr A = \Tr A'$
\end{enumerate}

\setcounter{paragraph}{19}
\paragraph{Корневые подпространства}
\begin{defn}\label{defn:linop::rootsp::rootv}
  Пусть $\varphi\in \End(V)$, $\lambda \in K$\footnote{У меня тут в конспекте баг, а у вас?}
  Корневой вектор~--- такой вектор $x$, что
  \[
    \exists\, k \in \N \colon (\varphi -  \lambda \id)^k(x) = 0
  \]
\end{defn}

\begin{defn}\label{defn:linop::rootsp::rootsp}
  Корневое подпространство~--- множество всех корневых векторов для данного числа $\lambda$.
  Обозначается $V(\lambda)$.
\end{defn}

\begin{stat}\label{stat:linop::rootsp::eigenrel}
  \[
    \lambda \in \Spec \varphi \Rightarrow V_\lambda \subset V(\lambda) 
  \]
\end{stat}

\begin{lem}\label{lem:linop::rootsp::lindep}
  Пусть $\psi \in \End(V)$, $x\in V$, $x \neq 0$. Пусть также $k \in \N$~--- минимальное
  $k$, что $\psi^k (x) =  0$. Тогда
  \[
    \{x, \psi(x) , \dotsc , \psi^{k-1}(x) \} \text{~--- линейно независимы}
  \]
\end{lem}
\begin{itlproof}
  Пусть оно линейно зависимо. Тогда
  \[
    \exists\, \beta_i \neq 0 \colon \beta_0 x + \beta_1 \psi(x) + \dotsb + \beta_{k-1}
    \psi^{k-1} (x) = 0
  \]
  Пусть $\ell$~--- наименьший индекс $\beta$ не равного нулю. Тогда если применить к обеим
  частям предыдущего равенства $\psi^{k-1-\ell}$, то
  \[
    0 + \dotsb + 0 +\beta_\ell \psi ^{k-1} (x) + 0+ \dotsb + 0 = 0 \Rightarrow \beta_l = 0 
  \]
  А таким методом можно получить что все $\beta_i = 0$. (?!?)
\end{itlproof}

\begin{stat}\label{stat:linop::rootsp::heightlessdim}
  \[
    V(\lambda) = \{ x\in V \mid (\varphi - \lambda \id)^n(x) = 0 \}, \; n = \dim V
  \]
\end{stat}
\begin{itlproof}
  Больше размерности линейно независимых векторов не наберёшь.
\end{itlproof}

\paragraph{Сумма корневых подпространств}

\begin{lem}\label{lem:linop::rootspsum::gcd}
  Пусть $f, g\in K[t]$, $(f, g) = 1$. Тогда 
  \[
    \big( f(\varphi)(x) = g(\varphi)(x) = 0 \big) \Rightarrow x = 0
  \]
\end{lem}


\begin{thrm}\label{thrm:linop::rootspsum}
  Пусть $\Spec \varphi = \{\lambda_1, \dotsc , \lambda_n\}$, $n = \dim V$\footnote{Тут важно,
    что именно размерность берётся, иначе~\ref{stat:linop::rootsp::heightlessdim} не сработает}.
  Тогда 
  \[
    \sum_{i=1}^{n} V(\lambda_i)^n\text{~--- прямая}
  \]
\end{thrm}
\begin{ittproof}
  Воспользуемся тут критерием прямой суммы. Докажем, что $W_i \cap V(\lambda_i) = \{0\}$.
  Хорошо, пусть это не так. Выберем $x$ из этого пересечения. Тогда $x= \sum_{i\neq j} x_j$,
  где $x_j \in W_j$.

  Рассмотрим:
  \begin{align*}
    f(\varphi) = \prod_{j\neq i} (\varphi -\lambda_j \id)^n
  \end{align*}
  Тогда 
  \[
    \forall\, j \;\: f(\varphi)(x_j) = 0 \Rightarrow f(\varphi)(x) = 0
  \]
  В нём попросту найдётся нужное корневое число. 
  
  С другой стороны, \[
    g(\varphi)(x) = (\varphi - \lambda_i)^n(x) = 0
  \]

  А поскольку $f, g$~--- взаимно просты, то по лемме \ref{lem:linop::rootspsum::gcd} $x=0$.
\end{ittproof}
\newpage
\fbox{А вот тут начинается совсем жестище\ldots}
\newpage
\paragraph{Про инвариантность корневых подпространств}
\begin{thrm}\label{thrm:linop::rootspinv}
  Пусть $\varphi \in \End (V)$, $\Spec \varphi = \{\lambda_1, \dotsc , \lambda_m\}$.
  Пусть ещё характеристический многочлен разложился на множители (ну в $\C$ заберёмся)
  \[
    \chi_\varphi(t) = \prod_{i=1}^m (t-\lambda_i)^{k_i}
  \]
  Тогда:
  \begin{enumerate}
    \item $\displaystyle V = \bigoplus\limits_{i=1}^m V(\lambda_i)$
    \item $\displaystyle V(\lambda_i)$~--- $\varphi$-inv.
  \end{enumerate}
\end{thrm}

\begin{ittproof}
  Соорудим $i$ многочленов
  \[
    f_i (t) = \prod_{j\neq i} (t -\lambda_j)^{k_j}
  \]
  Они все взаимно просты. Тогда есть такое линейное представление $\id$ :
  \[
    (h_1 f_1)(\varphi) + \dotsb + (h_m f_m) (\varphi) = \id
  \]
  посчитаем такую штуку для каждого $x\in V$.
  \begin{align*}
    \intertext{Пусть}
    &W_i = \big((h_i f_i)(\varphi)\big) (V) \\
    \intertext{Тогда}
    \id(V) = &W_1 + \dotsb + W_m = V
  \end{align*}
  \begin{enumerate}[а)]
    \item Докажем, что $W_i$~--- $\varphi$-inv.
      Там многочлены в процессе коммутируют, мы это доказывали в~\ref{lem:linop::poly::commute}
      \[
        \begin{split}
          \varphi(W_i) &= \varphi\big(h_i f_i(\varphi)(V)\big) 
          = \big(\varphi \cdot h_i(\varphi) \cdot f_i(\varphi) \big)(V) \\
          &= \big( h_i(\varphi) \cdot f_i(\varphi) \big) (\varphi (V)) 
          < \big( h_i(\varphi) \cdot f_i(\varphi) \big) (V) = W_i
        \end{split}
      \]
    \item Докажем, что $W_i \subset V(\lambda_i)$. Пусть $y \in W_i$
  \end{enumerate}<++>

\end{ittproof}<++>

\end{document}

