%------------------------------------------------------------
% Description : Chapter about linear operators
% Author      : tis-p30 <iliya.t@mail.ru>
% Created at  : Sat Jun 18 12:38:13 MSK 2016
%------------------------------------------------------------
\documentclass[12pt]{../../../notes}
\usepackage{silence}
\WarningFilter{latex}{Reference}
\graphicspath{{../../img/}}

\begin{document}

\setcounter{paragraph}{7}

\paragraph{Кольцо линейных операторов}

\begin{defn}\label{defn:linop::operring::linop}
  Пусть $V$~--- линейное пространство над полем $K$. Пусть также $\varphi \colon V \to V$, и
  $\varphi$~--- линейное отображение. Тогда $\varphi$~--- линейный оператор.
\end{defn}

\begin{defn}[Сложение и умножение операторов]\label{defn:linop::operring::linopaddmul}
  Введём 2 операции:
  \begin{alignat*}{3}
    +     &\colon V\times V \to V & \:&\wedge\: & (\psi + \varphi)(x) &= \psi(x) + \varphi(x) \\
    \circ &\colon V\times V \to V & \:&\wedge\: & (\psi \circ \varphi)(x) &= \psi(\varphi(x))
  \end{alignat*}
\end{defn}

\begin{thrm}\label{thrm:linop::operring::operring}
  Множество эндоморфизмов $\End(V)$ с операциями, определёнными 
  в~\ref{defn:linop::operring::linopaddmul}~--- кольцо.
\end{thrm}

\begin{thrm}\label{thrm:linop::operring::isommtx}
  Пусть $\dim V = n$. Тогда
  \[
    (\End (V),\circ,+) \cong (M_n(K), \cdot, +)
  \]
\end{thrm}
\begin{ittproof}
  Пусть $\varphi, \psi \in \End(V)$, $A, B \in M_n(K)$.
  Выберем базис в $V$ и рассмотрим отображение $f \colon \varphi \mapsto A_\varphi$,
  композиция переходит в умножение матриц, сложение~--- в сложение матриц.

  Такое отображение обратимо, действительно, 
  \[
    \forall\, A \in M_n(K) \;\: \big( \omega(x) := A x \big) \in \End(V)
  \]
  А значит $f$~--- биекция.

  Пусть в выбранном базисе $\varphi(x) =  Ax$, $\psi(x) = Bx$. 
  Уже доказывали, что матрица, соответствующая композиции $\psi\varphi$, равна $BA$.
  Теперь разберёмся с матрицей суммы
  \[
    (\varphi + \psi)(x) = \varphi(x) + \psi(x) = A x + B x = (A+B) x
  \]
  То есть, в фиксированном базисе $(\varphi+\psi)$ соответствует $A+B$.
  
  Таким образом, раз базис выбирали произвольно, то в любом базисе $V$.
  \begin{enumerate}
    \item $f$~--- биекция.
    \item $f(\varphi + \psi) = A_{\varphi+\psi} = A_\varphi + A_\psi = f(\varphi) + f(\psi)$
    \item $f(\varphi \psi) = A_{\varphi\psi} = A_\varphi \cdot A_\psi = f(\varphi) \cdot f(\psi)$
  \end{enumerate}
  Следовательно, $f$~--- изоморфизм.
\end{ittproof}

\setcounter{paragraph}{11}
\begin{defn}\label{defn:linop::poly::poly}
  Пусть $\varphi\in \End(V)$, $A$~--- матрица $\varphi$ в фиксированном базисе, $p\in K[t]$.
  Тогда
  \begin{align*}
    p(\varphi)(x) &= a_k \varphi^l(x) + \dotsb + a_1 \varphi(x) + a_0
    p(A) &= a_k A^l + \dotsb + a_1 A + a_0
  \end{align*}
\end{defn}
\begin{lem}\label{lem:linop::poly::mtxoprel}
  Если в некотором базисе $\varphi$ имеет матрицу $A$, то в том же базисе $f(\varphi)$ имеет
  матрицу $f(A)$.
\end{lem}
\begin{lem}\label{lem:linop::poly::commute}
  Многочлены от одного и того же оператора и его матрицы в фиксированном базисе коммутируют.
\end{lem}
\begin{itlproof}
  По сложению оно все коммутативно. В слагаемых переставлять нужно степени одного и того же.
  Циклические группы обычно абелевы.
\end{itlproof}

\paragraph{ Инвариантные подпространства }
\begin{defn}\label{defn:linop::invsubsp::invsubsp}
  Пусть $\varphi\in \End(V)$, $W$~--- подпространство $V$. Тогда если $\varphi(W) \subset V$,
  то $W$~--- $\varphi$-инвариантное подпространство $V$.
\end{defn}

\begin{lem}\label{lem:linop::invsubsb::dirsum}
  Пусть:
  \begin{itemize}
    \item $V = \bigoplus_{i=1}^n W_i$
    \item $W_{i}$~--- $\varphi$-инвариантное подпространство
    \item базис $V$ разбивается на базисы $W_i$.
    \item $\varphi\left.\vphantom{g_{\int}}\right|_{W_i} \in \End(V)$
    \item В фиксированном базисе $A_i, A$~--- матрицы $\varphi_i, \varphi$ соответственно. 
  \end{itemize}
  Тогда
  \[
    A = 
    \begin{pmatrix}
      \boxed{A_1} & 0   & 0      & 0    \\
      0   & \mkern -15mu \boxed{A_2} & 0      & 0    \\
      0   & 0   & \ddots & 0    \\
      0   & 0   & 0      & \boxed{A_n}  \\
    \end{pmatrix}
  \]
\end{lem}
\begin{itlproof}
  Можно рассмотреть один такой <<блок>>. Если сверху/снизу него не ноли, то с инвариантностью
  проблемы. 
\end{itlproof}

\paragraph{Характеристический многочлен оператора}
\begin{defn}\label{defn:linop::charpoly::charpoly}
  Пусть $\varphi\in \End(V)$, $A$~--- его матрица в выбранном базисе.
  Тогда
  \[
    \chi_\varphi(t) = \det (A - t E_n)
  \]
\end{defn}

\begin{stat}\label{stat:linop::charpoly::corr}
  Какой бы базис не выбрали в $V$, характеристический многочлен не изменится.
\end{stat}
\begin{itlproof}
  Матрицы оператора во всевозможных базисах подобны. Единичная матрица не поменяется при смене
  базиса. А определители подобных матриц равны.
\end{itlproof}

\subparagraph{Свойства}
\begin{enumerate}
  \item $\deg \chi_\varphi = n$
  \item Пусть $\chi_\varphi(t) = a_n t^n + \dotsb + a_0$. 
    Тогда
    \begin{align*}
      &a_n = (-1)^n \\
      &a_{n-1} = (-1)^n (a_{11} + \dotsb + a_{nn})
    \end{align*}
    {\defn\label{defn:linop::charpoly::trace} $\Tr A = a_{11} + \dotsb + a_{nn}$ }
  \item $A \sim A' \Rightarrow \Tr A = \Tr A'$
\end{enumerate}

\setcounter{paragraph}{19}
\paragraph{Корневые подпространства}
\begin{defn}\label{defn:linop::rootsp::rootv}
  Пусть $\varphi\in \End(V)$, $\lambda \in K$\footnote{У меня тут в конспекте баг, а у вас?}
  Корневой вектор~--- такой вектор $x$, что
  \[
    \exists\, k \in \N \colon (\varphi -  \lambda \id)^k(x) = 0
  \]
\end{defn}

\begin{defn}\label{defn:linop::rootsp::rootsp}
  Корневое подпространство~--- множество всех корневых векторов для данного числа $\lambda$.
  Обозначается $V(\lambda)$.
\end{defn}

\begin{stat}\label{stat:linop::rootsp::eigenrel}
  \[
    \lambda \in \Spec \varphi \Rightarrow V_\lambda \subset V(\lambda) 
  \]
\end{stat}

\begin{lem}\label{lem:linop::rootsp::lindep}
  Пусть $\psi \in \End(V)$, $x\in V$, $x \neq 0$. Пусть также $k \in \N$~--- минимальное
  $k$, что $\psi^k (x) =  0$. Тогда
  \[
    \{x, \psi(x) , \dotsc , \psi^{k-1}(x) \} \text{~--- линейно независимы}
  \]
\end{lem}
\begin{itlproof}
  Пусть оно линейно зависимо. Тогда
  \[
    \exists\, \beta_i \neq 0 \colon \beta_0 x + \beta_1 \psi(x) + \dotsb + \beta_{k-1}
    \psi^{k-1} (x) = 0
  \]
  Пусть $\ell$~--- наименьший индекс $\beta$ не равного нулю. Тогда если применить к обеим
  частям предыдущего равенства $\psi^{k-1-\ell}$, то
  \[
    0 + \dotsb + 0 +\beta_\ell \psi ^{k-1} (x) + 0+ \dotsb + 0 = 0 \Rightarrow \beta_l = 0 
  \]
  А таким методом можно получить что все $\beta_i = 0$. (?!?)
\end{itlproof}

\begin{stat}\label{stat:linop::rootsp::heightlessdim}
  \[
    V(\lambda) = \{ x\in V \mid (\varphi - \lambda \id)^n(x) = 0 \}, \; n = \dim V
  \]
\end{stat}
\begin{itlproof}
  Больше размерности линейно независимых векторов не наберёшь.
\end{itlproof}

\paragraph{Сумма корневых подпространств}

\begin{lem}\label{lem:linop::rootspsum::gcd}
  Пусть $f, g\in K[t]$, $(f, g) = 1$. Тогда 
  \[
    \big( f(\varphi)(x) = g(\varphi)(x) = 0 \big) \Rightarrow x = 0
  \]
\end{lem}


\begin{thrm}\label{thrm:linop::rootspsum}
  Пусть $\Spec \varphi = \{\lambda_1, \dotsc , \lambda_m\}$, $n = \dim V$.Тогда 
  \[
    \sum_{i=1}^{m} V(\lambda_i)^n\text{~--- прямая}
  \]
\end{thrm}
\begin{ittproof}
  Воспользуемся тут критерием прямой суммы. Докажем, что $W_i \cap V(\lambda_i) = \{0\}$.
  Хорошо, пусть это не так. Выберем $x$ из этого пересечения. Тогда $x= \sum_{i\neq j} x_j$,
  где $x_j \in W_j$.

  Рассмотрим:
  \begin{align*}
    f(\varphi) = \prod_{j\neq i} (\varphi -\lambda_j \id)^n
  \end{align*}
  Тогда 
  \[
    \forall\, j \;\: f(\varphi)(x_j) = 0 \Rightarrow f(\varphi)(x) = 0
  \]
  В нём попросту найдётся нужное корневое число. 
  
  С другой стороны, \[
    g(\varphi)(x) = (\varphi - \lambda_i)^n(x) = 0
  \]

  А поскольку $f, g$~--- взаимно просты, то по лемме \ref{lem:linop::rootspsum::gcd} $x=0$.
\end{ittproof}
\newpage
\fbox{А вот тут начинается совсем жестище\ldots}
\newpage
\paragraph{Про инвариантность корневых подпространств}
\begin{thrm}\label{thrm:linop::rootspinv}
  Пусть $\varphi \in \End (V)$, $\Spec \varphi = \{\lambda_1, \dotsc , \lambda_m\}$.
  Пусть ещё характеристический многочлен разложился на множители (ну в $\C$ заберёмся)
  \[
    \chi_\varphi(t) = \prod_{i=1}^m (t-\lambda_i)^{k_i}
  \]
  Тогда:
  \begin{enumerate}
    \item $\displaystyle V = \bigoplus\limits_{i=1}^m V(\lambda_i)$
    \item $\displaystyle V(\lambda_i)$~--- $\varphi$-inv.
  \end{enumerate}
\end{thrm}

\begin{ittproof}
  Соорудим $i$ многочленов
  \[
    f_i (t) = \prod_{j\neq i} (t -\lambda_j)^{k_j}
  \]
  Они все взаимно просты. Тогда есть такое линейное представление $\id$ :
  \[
    (h_1 f_1)(\varphi) + \dotsb + (h_m f_m) (\varphi) = \id
  \]
  посчитаем такую штуку для каждого $x\in V$.
  \begin{align*}
    \intertext{Пусть}
    &W_i = \big((h_i f_i)(\varphi)\big) (V) \\
    \intertext{Тогда}
    \id(V) = &W_1 + \dotsb + W_m = V
  \end{align*}
  \begin{enumerate}[a)]
    \item Докажем, что $W_i$~--- $\varphi$-inv.
      Там многочлены в процессе коммутируют, мы это доказывали в~\ref{lem:linop::poly::commute}
      \[
        \begin{split}
          \varphi(W_i) &= \varphi\big(h_i f_i(\varphi)(V)\big) 
          = \big(\varphi \cdot h_i(\varphi) \cdot f_i(\varphi) \big)(V) \\
          &= \big( h_i(\varphi) \cdot f_i(\varphi) \big) (\varphi (V)) 
          < \big( h_i(\varphi) \cdot f_i(\varphi) \big) (V) = W_i
        \end{split}
      \]
    \item Докажем, что $W_i \subset V(\lambda_i)$. Пусть $y \in W_i$. Тогда
      \[
        \begin{split}
          y &= (h_1(\varphi) f_1(\varphi))(x) \\
          (\varphi - \lambda_i \id)^{k_i} (y) &= \big((\varphi - \lambda_i \id)^{k_i}  h_1(\varphi) f_1(\varphi)\big)(x) \\
          &= h_i(\varphi) \cdot \big( \underbrace{(\varphi - \lambda_i \id)^{k_i} f_1(\varphi)}_{\chi_\varphi(\varphi) = 0} \big)(x)
          = 0
        \end{split}
      \]
  \end{enumerate}
  Так как корневые подпространства~--- подпространства $V$, то 
  \[
    V \supset \bigoplus\limits_{i=1}^m V (\lambda_i) \Rightarrow \dim V \geqslant \sum_{i=1}^{m} \dim V(\lambda_i)
  \]
  С другой стороны, 
  \[
    \dim \sum_{i=1}^{m} W_i \leqslant \sum_{i=1}^{m} \dim W_i 
  \]
  При этом 
  \[
    W_i \subset V(\lambda_i) \Rightarrow \dim W_i \leqslant \dim V(\lambda_i)
  \]
  Так что
  \[
    \dim V = \dim \sum_{i=1}^{m} W_i \leqslant \sum_{i=1}^{m} \dim W_i \leqslant \sum_{i=1}^{m} \dim V(\lambda_i) \leqslant \dim V
  \]
\end{ittproof}

\paragraph{Размерность корневого подпространства}

\begin{thrm}\label{thrm:linop::rootdim}
  Пусть $\varphi \in \End (V)$, $\Spec \varphi = \{\lambda_1, \dotsc , \lambda_m\}$.
  Пусть ещё характеристический многочлен разложился на множители (ну в $\C$ заберёмся)
  \[
    \chi_\varphi(t) = \prod_{i=1}^m (t-\lambda_i)^{k_i}
  \]
  Тогда:
  \begin{enumerate}
    \item $\dim V(\lambda_i) = k_i$
    \item $\varphi_i = {}^{\varphi}\big\vert_{V(\lambda_i)}$ имеет единственное собственное число $\lambda_i$.
  \end{enumerate}
\end{thrm}
\begin{ittproof}
  \begin{enumerate}
    \item[2] Пусть $\mu$~--- собственное число $\varphi_i$ не равное $\lambda_i$.
      Тогда 
      \begin{align*}
        (\varphi - \mu \id) (x) = 0 \\
        (\varphi - \lambda_i \id)^n (x) = 0 \\
        \big((t-\mu),(t-\lambda_i)^n\big) = 1
      \end{align*}
      А по лемме~\ref{lem:linop::rootspsum::gcd} $x=0$. А тут что-то не так.\footnote{Собственные числа есть, так как любой
      характеристический многочлен приводим в $\C$. А тогда $\det(A - \lambda E_n) = 0 \Rightarrow \rk(A - \lambda E_n) <
      n$. Тогда и  размерность $\Ker (\varphi - \lambda \id)$ не ноль. Значит, ненулевой вектор там есть. }.

    \item[1] Так как мы уже доказали, что пространство~--- прямая сумма корневых, то его базис разбивается на базисы корневых
      подпространств.
      \[
        \underbrace{ \underbrace{e_1^1, \dotsc, e_{s_1}^1}_{\text{базис } V(\lambda_1)} , 
        \dotsc,  \underbrace{e_1^n, \dotsc, e_{s_n}^n}_{\text{базис } V(\lambda_m)}  }_{\text{базис } V}
      \]
      мы когда-то (\ref{lem:linop::invsubsb::dirsum}) доказали, что матрица $\varphi$ в таком случае выглядит так:
      \[
        \begin{pmatrix}
          \boxed{A_1} &    ~        &   0    &  \\
              ~       &\mkern -15mu \boxed{A_2} &   ~    &  \\
              ~       &    ~        & \ddots &  \\
              0       &    ~        &   ~    & \boxed{A_m}
        \end{pmatrix}
      \]
      Тогда 
      \[
        \begin{split}
          \chi_\varphi(t) &= \det (A - t E_n ) = \det (A_1 - t E_n) \dotsm \det (A_m - t E_n)  = \\
          &= \chi_{\varphi_1}(t) \dotsm \chi_{\varphi_m}(t)
        \end{split}
      \]
      Что может входить в $\chi_{\varphi_i}$? $(t - \lambda_j), j\neq i$ там точно нет из второго пункта.
      Но часть $t - \lambda_i$ в него не входить не может, иначе мы просто не наберём нужную степень в $\chi_\varphi(t)$.
      Так что 
      \[
        \chi_{\varphi_i}(t) = (t - \lambda_i)^k_i
      \]
      Но 
      \[
        s_i = \dim V(\lambda_i) = \deg \chi_{\varphi_i}(t) = k_i
      \]
  \end{enumerate}
\end{ittproof}


\parrange{2}{Жорданова нормальная форма}

Сначала пара определений
\begin{defn}[Относительная линейная независимость]\label{defn:linop::jnf::relind}
  Пусть $W$~--- подпространство $V$, $e_1, \dotsc, e_s \in V$ Тогда $\{e_i\}$ ЛНЗ относительно $W$, если
  \[
    \alpha_1 e_1 + \dotsb + \alpha_s e_s \in W \Rightarrow \forall\, \alpha_i = 0
  \]
  Или (эквивалентная формулировка) объединение с базисом подпространства линейно независимо в $V$.
\end{defn}

\begin{defn}[Относительный базис]\label{defn:linop::jnf::relbasis}
  Пусть $W$~--- подпространство $V$. Тогда дополнение базиса $W$ до базиса $V$ называется базисом $V$ относительно
  $W$

  Или, что тоже самое, они относительно линейно независимы и их линейная оболочка с $W$ равна $V$.
\end{defn}


\begin{lem}\label{lem:linop::jnf::relindtorelbasis}
  \marginpar{\footnotesize может это 1-ая корректность?  }
  Относительно линейно независимую систему можно дополнить до относительного базиса.
\end{lem}

Теперь что известно:
\begin{itemize}
  \item $V$ - линейное пространство над $K$, $\dim V = n$, $\varphi\in \End(V)$
  \item $\Spec \varphi = \{\lambda_1, \dotsc, \lambda_m\}$.
  \item $\displaystyle \chi_\varphi(t) = \prod_{i=1}^m(t-\lambda_i)^{k_i}$
  \item $V = \bigoplus\limits_{i=1}^m V (\lambda_i)$
  \item $\dim V(\lambda_i) = k_i$
  \item $A_i$~--- матрица ${}^\varphi\big|_{V(\lambda_i)}$
    \[
    A = \begin{pmatrix}
          \boxed{A_1} &    ~        &   0    &  \\
              ~       &\mkern -15mu \boxed{A_2} &   ~    &  \\
              ~       &    ~        & \ddots &  \\
              0       &    ~        &   ~    & \boxed{A_m}
        \end{pmatrix}
  \]
\end{itemize}

\begin{defn}[ЖНФ]\label{defn:linop::jnf::jnf}
  Такая форма записи матрицы линейного оператора:
  \[
    \begin{pmatrix}
      J_1 &        & ~ \\
      ~   & \ddots & ~ \\
      ~   & ~ & J_m \\
    \end{pmatrix}
  \]
  где $J_i$~--- жорданова клетка
  \[
    J_i = 
    \begin{pmatrix}
      \lambda_i & 1   & ~ & ~   & ~  \\
      ~   & \lambda_i & 1 & ~   & ~  \\
      ~   & ~   & \ddots & \ddots   & ~  \\
      ~   & ~   & ~ & \lambda_i & 1  \\
      ~   & ~   & ~ & ~   & \lambda_i \\
    \end{pmatrix}
  \]
\end{defn}
\begin{defn}[Жорданов базис]\label{defn:linop::jnf::jbasis}
  Базис, в котором матрица линейного оператора выглядит, как в~\ref{defn:linop::jnf::jnf}
\end{defn}


Рассмотрим $\psi = \varphi - \lambda \id$.
\newcommand{\eqrot}{\rotatebox{90}{=}}
Тогда 
\[
  V_\lambda = \begin{array}{c}
    \Ker \psi\\
    \eqrot \\
    W_1
\end{array} \supset 
  \begin{array}{c}
    \Ker \psi^2\\
    \eqrot \\
    W_1
\end{array} \supset \dotsb \supset
  \begin{array}{c}
    \Ker \psi^{\ell}\\
    \eqrot \\
    W_\ell
  \end{array} = V(\lambda)
\]
Во всей этой процедуре будем ещё базисы $W_j$ искать

\begin{itemize}
  \item $s_\ell$ векторов на первой ступеньке~--- базис $W_\ell$ относительно $W_{\ell-1}$, то есть что добавилось на последнем
    шаге. 
  \item Все ступеньки выше $r-1$~--- базис $W_\ell$ относительно $W_{r-1}$
  \item На каждом шаге считаем $\psi$ от всего, что было на предыдущей ступеньке и добавляем векторов, 
    чтобы выполнялось предыдущее условие.
\end{itemize}

\begin{lem}\label{lem:linop::jnf::stairscont}
  На $r$-ой ступеньке лежат векторы из $\Ker \psi^r$. 
\end{lem}
\begin{itlproof}
  По индукции:
  \begin{description}
    \item[База:] Векторы на $\ell$-ой ступеньке из $\Ker \psi^\ell$.
    \item[Переход:] $x\in \Ker \psi^{r+1} \Rightarrow \psi(x) \in \Ker \psi^{r}$. А оставшиеся векторы добираются из
      $W_r = \Ker \psi^r$
  \end{description}
\end{itlproof}

\begin{lem}[Корректность поиска базиса]\label{lem:linop::jnf::corralg}
  После дополнения до базиса относительно $W_{r-1}$ система будет ЛНЗ относительно $W_{r-2}$.
\end{lem}

\begin{itlproof}
  Пусть оно линейно зависимо относительно $W_{r-2}$. Тогда 
  \begin{equation}
    \sum_{i=r}^{\ell} \left( \sum_{j=1}^{s_i} \alpha_j^i x_j^i \right) + \sum_{j=1}^{s_r} \beta_j \psi(x^{r}_j) = w \in W_{r-2}
    \label{eq:linop::jnf::steprelind}
  \end{equation}
  \begin{enumerate}[1)]
    \item не все $\alpha_j^i = 0$ \\
      Пусть $t$~--- наибольший номер этажа на котором есть ненулевые $\alpha_i^t$. Применим $\psi^{t-1}$ к обеим частям
      равенства~\eqref{eq:linop::jnf::steprelind}.
      
      Второй член уберётся совсем, ведь $t \geqslant r$. Правая часть пропадёт по тем же причинам. 
      А вот от первого слагаемого левой останется кусок (это не весь, ещё штуки вида $x^{t-k}$ есть, но они пока не нужны):
      \[
        \sum_{j=1}^{s_t} \alpha_j^t \psi^{t-1}(x_j^t)= 0 \Rightarrow \psi^{t-1} \left( \sum_{j=1}^{s_t} \alpha_j^t x_j^t \right) = 0
        \Rightarrow \sum_{j=1}^{s_t} \alpha_j^t x_j^t \in W_{t-1}
      \]
      В итоге оно линейно зависимо над $W_{t-1}$,$t-1 > r-2$ а мы тут неявно предполагали по полной индукции, что нет.
    \item все $\alpha_j^i = 0$ 
      Тогда просто применяем $\psi^{r-1}$ к~\eqref{eq:linop::jnf::steprelind}. Выйдет, что
      \[
        \psi^{r-1}\left( \sum_{j=1}^{s_r} \beta_j x_j^r \right) = 0 \Rightarrow \sum_{j=1}^{s_r} \beta_j x_j^r \in W_{r-1}
      \]
      Но в таком случае снова проблемы с индукционным предположением.
  \end{enumerate}
\end{itlproof}

Теперь рассмотрим циклическое подпространство
\[
  N_x = \langle x, \psi(x), \dotsc, \psi^{\ell-1}(x) \rangle
\]
Пусть $B_{N_x}$~--- матрица $\psi \big\vert_{N_x}$. Тогда можно понять, как она выглядит:
\[
  \begin{cases}
    B x = \psi(x) &\\
    \hdotsfor{1} &\\
    B \psi^{\ell-1} (x) = \psi^\ell(x) = 0 &\\
  \end{cases}
  \Rightarrow
  B = \ell \left\{ \vphantom{ \begin{matrix}~ \\~ \\ ~ \\~ \\~ \\~\end{matrix} } \!\!\right.
  \overbrace{
    \boxed{
      \begin{matrix}
        0 & 1 & ~ & ~ & ~ \\
        ~ & 0 & 1 & ~ & ~ \\
        ~ & ~ & 0 & \ddots & ~ \\
        ~ & ~ & ~ & \ddots & 1 \\
        ~ & ~ & ~ & ~ & 0 
      \end{matrix}
    }
  }^\ell
\]

Тогда блок жордановой формы выглядит так:
\[
    J_i = \boxed{\begin{matrix}
          \lambda_i & 1 & ~ & ~ \cr
          ~ & \lambda_i & \ddots & ~ \cr
          ~ & ~ & \ddots & 1 \cr
          ~ & ~ & ~ & \lambda_i \cr
      \end{matrix}}
\]
где $\lambda_i$~--- собственное число.
\end{document}

