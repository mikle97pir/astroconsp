\documentclass[10pt]{../notes}
\usepackage{docmute}

\title{Определения из дифуров}
\date{\today}
\author{\texttt{t a x u s}}

\begin{document}

\maketitle

\begin{defn}[Дифференциальное уравнение]\label{defn:diffeq}
  Пусть $f \in C(G)$, $G \subset \R^2$. Тогда диффур~--- вот такая штука:
  \[
    y' = f(x, y)
  \]
\end{defn}
\begin{defn}\label{defn:diffsol}
  Решение~--- функция $y=\varphi(x)$, определённая на $\langle a, b \rangle$ :
  \begin{enumerate}
    \item $\varphi(x)$~--- дифференцируема
    \item $(x, \varphi(x)) \in G$
    \item $\varphi'(x) = f(x, \varphi(x))$
  \end{enumerate}
\end{defn}

\begin{defn}[Задача Коши и вокруг неё]\label{defn:cauchyprobl} 
  Основные понятия тут:
  \begin{enumerate}
    \item $(x_0, y_0)$~--- начальные данные
  \item Решение задачи Коши~--- \emph{частное} решение дифура + выполнение начальных условий
    \item Решение задачи Коши существует, если $\exists\, (a,b) \ni x_0,
      y=\varphi(x)\colon y_0 = \varphi(x_0)$
    \item Решение задачи Коши единственно, если любые 2 решения совпадают в окрестности
      $x_0$ 
  \end{enumerate}
\end{defn}

\begin{defn}\label{defn:uniqpnt}
  $(x_0, y_0) \in G$~--- тоска единственности, если решение задачи Коши в ней единственно.
\end{defn}
\begin{defn}\label{defn:uniqset}
  $\tilde{G} \in G$~--- область единственности, если каждая её точка~--- точка
  единственности.
\end{defn}


\end{document}
