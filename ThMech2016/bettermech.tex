%------------------------------------------------------------
% Description : 
% Author      : Iliya Tikhonenko <iliya.t@mail.ru>
% Created at  : Fri Jun 16 18:12:20 MSK 2017
%------------------------------------------------------------
\documentclass[timbord]{longnotes}
\usepackage{tmath}
\usepackage{cussymb}
\graphicspath{{img/}}

\makeatletter
\let\old@v=\v
\def\v{\@ifstar\v@star\v@choice}
\def\v@choice#1{\ifmmode \v@vec{#1} \else \old@v{#1} \fi}
\def\v@star#1{\v@lin{#1}}
\def\v@vec#1{\mathbf{#1}}
\def\v@lin#1{\mathrm{#1}}
\makeatother

\begin{document}
\chapter{Кинематика точки}

\setcounter{paragraph}{1}
\paragraph{Косоугольные координаты}

Здесь можно немного добавить строгости, а то ничерта не понятно.
Пусть $V$~--- евклидово пространство (линейное со скалярным произведением).
Как нам определяли, $g_{ik} = \v{e_i} \cdot \v{e_k}$,
\[
   \v a \cdot \v b  = \sum_{ij}  a^i b^j g_{ij}
\]
Здесь $a^k$~--- коэффициенты разложения по $\v{e_k}$~--- называются контравариантными координатами.

Пусть $V^*$~--- сопряжённое к $V$, его базисом являются координатные функции
$\v*{f_k} \that \v*{f_k}(\v x) = x^k$. 
Поскольку задано скалярное произведение, задан канонический изоморфизм $V \to V^*$.
Нам, правда, потребуется $V^* \to V$.

Введём ещё одну систему \emph{векторов} в $V$ : $\v{e^k} = \v*{f_k^*}$, то есть 
$\v*{f_k}(\v x) = \v{e^k} \cdot \v x$. 
Она и называется взаимным
базисом, коэффициенты разложения по ней~--- ковариантные координаты.
Из линейности скалярного произведения, ровно такие же координаты будут у соответствующей
формы в $V^*$.
Линейную независимость легко получить из ЛНЗ $\v*{f_k}$, a
раз их $\dim V$, то полученные векторы являются базисом.

Так что можно сформулировать правило:
\begin{itemize}
  \item Контравариантные координаты~--- коэффициенты разложения по базису линейного пространства.
  \item Ковариантные координаты~--- коэффициенты разложения по базису пространства линейных форм.
\end{itemize}

А вот теперь можно уже развлекаться с индексами.

\begin{prop}
  $\v{e^k} \cdot \v{e_j} = \delta_{kj}$
\end{prop}

\begin{prop}
  Пусть $\v r =  \xi^k \v {e_k}$ и $=\xi^k \v{e_k}$. Тогда $\xi_k = \v r \cdot \v{e_k}$
\end{prop}

\end{document}
% vim:tw=100 cc=100
