%------------------------------------------------------------
% Description : Билеты по теормеху 
% Author      : Iliya Tikhonenko <iliya.t@mail.ru>
% Created at  : Thu Jun 15 21:04:07 MSK 2017
%------------------------------------------------------------
% see `www.github.com/taxus-d/astroconsp' for `notes' class
\documentclass[landscape,10pt]{notes}
\geometry{
  left=0.7cm,
  right=0.7cm,
}
\usepackage{multicol}
\graphicspath{{img/}}

\title{Список вопросов к экзамену по курсу
<<Теоретическая механика>>}
\author{лектор:~В.Ш.~Шайдуллин}
\date{21.06.2017}
\titleformat{\section}%
{\normalfont\large\bfseries}
{\thesection.}
{1.7ex}{}
\setlength{\columnsep}{7em}

\begin{document}
\thispagestyle{empty}
\begin{multicols}{2}

  \section{Кинематика материальной точки}
  \begin{enumerate}
    \item Способы задания закона движения, скорости и ускорения материальной точки.
    \item Косоугольные декартовые и криволинейные координаты.
    \item Скорость и ускорение в криволинейных координатах.
    \item Скорость и ускорение при сложном движении, углы Эйлера.
  \end{enumerate}

  \section{Динамика материальной точки и системы материальных точек}
  \begin{enumerate}[resume]
    \item Уравнения движения для одной материальной точки и системы материальных
      точек, уравнения Лагранжа второго рода, понятие центра масс.
    \item Теорема изменения импульса и закон сохранения импульса для одной материальной точки и системы материальных точек.
    \item Теорема изменения момента импульса и закон сохранения момента импульса
      для одной материальной точки и системы материальных точек.
    \item Теорема изменения кинетической энергии и закон сохранения энергии для одной
      материальной точки и системы материальных точек.
    \item Движение в поле центральной потенциальной силы для одной материальной
      точки.
  \end{enumerate}
  \section{Кинематика твердого тела}

  \begin{enumerate}[resume]
    \item Закон движение твердого тела, углы Эйлера.
    \item Скорость и ускорение точек твердого тела.
    \item Сложение движений твердого тела.
    \item Кинематический винт.
    \item Плоское движение твердого тела.
  \end{enumerate}

  \section{Динамика твердого тела}
  \begin{enumerate}[resume]
    \item Уравнения движения твердого тела, тензор инерции.
    \item Вращение твердого тела вокруг неподвижной оси.
    \item Вращение твердого тела вокруг неподвижной точки, случай Эйлера.
    \item Вращение твердого тела вокруг неподвижной точки, случай Лагранжа.
  \end{enumerate}
  \section{Несвободное движение}

  \begin{enumerate}[resume]
    \item Связи, их виды, уравнения движения при наличии связей.
    \item Движение материальной точки по поверхности.
    \item Движение материальной точки по линии.
  \end{enumerate}

  \section{Вариационные принципы}
  \begin{enumerate}[resume]
    \item Принцип Даламбера – Лагранжа.
    \item Принцип Гамильтона – Остроградского.
    \item Принцип Лагранжа (интегральный).
    \item Уравнение Гамильтона – Якоби.
  \end{enumerate}

  \section{Канонические системы}
  \begin{enumerate}[resume]
    \item Канонические уравнения движения.
    \item Теорема Якоби.
    \item Интегральные инварианты.
    \item Канонические преобразования переменных.
  \end{enumerate}
\end{multicols}
\end{document}
% vim:tw=100 cc=100
