%------------------------------------------------------------
% Description : 
% Author      : Iliya Tikhonenko <iliya.t@mail.ru>
% Created at  : Tue Feb 21 00:00:16 MSK 2017
%------------------------------------------------------------
\documentclass[12pt, timbord]{longnotes}
\usepackage{tmath}
\usepackage{cussymb}
\usepackage{silence}
\WarningFilter{latex}{Reference}
\graphicspath{{../../img/}}

\begin{document}
\paragraph{Системы множеств}
\label{par:meas::setsys}

\begin{defn}\label{defn:meas::setsys::sub}
  Пусть здесь (и дальше) $X$~--- произвольное множество. Тогда $\pset(X) \equiv 2^X$~--- множество
  всех подмножеств $X$.
\end{defn}
\begin{exmp*}
  $X = \{1 \intrng n\} \Rightarrow \# \pset (X) = 2^n$ (это количество элементов, если что)
\end{exmp*}

\begin{defn}[Алгебра]\label{defn:meas::setsys::alg}
  Пусть $\alg \subset \pset (X)$. Тогда $\alg$~--- алгебра множеств, если
  \begin{enumerate}
    \item $\varnothing \in \alg$
    \item $X \in \alg$
    \item $A,B\in \alg \Rightarrow A \cap B, A \cup B, A \setminus B \in \alg$
  \end{enumerate}
\end{defn}
\begin{rem*}
  Заметим, что в алгебре пересечение (или объединение) \emph{конечного} числа её элементов лежит в
  алгебре.  Это можно доказать простой индукцией. А вот для бесконечных объединений пользоваться
  индукцией уже нельзя, ведь $\infty \not\in \N$. 
\end{rem*}

\begin{defn}[$ \sigma $-алгбера]\label{defn:meas::setsys::sigalg}
  Пусть $\alg \in \pset(X)$. Тогда $\alg$~--- $ \sigma $-алгебра, если 
  \begin{enumerate}
    \item $\alg$~--- алгебра
    \item $A_1, \dotsc, A_n \in \alg \Rightarrow \bigcup_{k=1}^ \infty A_k \in\alg, 
      \bigcap_{k=1}^{ \infty } A_k \in \alg $
  \end{enumerate}     
\end{defn}

\begin{defn}\label{defn:meas::setsys::sigshell}
  Пусть $\mathcal E \subset \pset(X)$. Тогда 
  \[
  \sigma(\mathcal E) := \bigcap\,\{\alg \mid \alg\text{~--- $\sigma$-алгебра, $\alg \supset
  \mathcal E $}\}
  \]
  эта конструкция~--- сигма-алгебра, просто аксиомы проверить.
\end{defn}

\paragraph{Борелевская сигма-алгебра}
\begin{defn}\label{defn:meas::setsys::borel}
  Пусть $\open$~--- все открытые множества в $\R^n$. Тогда $\mathpzc B_n = \sigma(\open)
  $~--- борелевская $\sigma$-алгебра в $\R^n$.
\end{defn}

\begin{defn}[Ячейка в $\R^n$ ]\label{defn:meas::setsys::cell}
  Обозначать её будем $\Delta^n$, по размерности соответствующего пространства.
  \[
    \Delta^1 = \begin{cases}
      [a;b) \\
      (-\infty;b) \\
      [a;+\infty) \\
      (-\infty;+\infty)
    \end{cases} \quad
    \forall\, n \;\: \Delta = \prod_{k=1}^n \Delta^1 _k 
  \]
  Ещё введём алгебру $\alg = \Cell_n = \{ A \mid A = \bigcup_{k=1}^p \Delta_k\}$
\end{defn} 

\begin{lem}\label{lem:meas::setsys::algsubset}
  Пусть $\mathcal E_1, \mathcal E_2 \subset \pset(X)$, $\sigma(\mathcal E_1) \supset \mathcal E_2$.
  Тогда $\sigma(\mathcal E_1) \supset \sigma(\mathcal E_2)$
\end{lem}
\begin{thrm}\label{thrm:meas::setsys::borelcell}
  $\borel_n = \sigma(\Cell_n)$.
\end{thrm}

\begin{exmp}\label{exmp:meas::setsys::borel}
%   \everymath{\displaystyle}
  Все множества нижё~--- борелевские.
  \begin{enumerate}[$\langle$1$\rangle$]
    \item $\open $.
    \item $\closed =\{A \mid \ov-{A} \in \open \}$.
    \item $\Biggl(A 
      = \bigcap_{\substack{{k=1}\\[0.12em]\mathclap{ G_k \in \open}}}^\infty G_k \Biggr)\in G_\delta$.
    \item $\Biggl(B 
      = \bigcup_{\substack{{k=1}\\[0.12em]\mathclap{ F_k \in \closed}}}^\infty F_k \Biggr)\in
      F_\sigma$.
    \item $\Biggl(C 
      = \bigcup_{\substack{{k=1}\\[0.12em]\mathclap{ A_k \in G_{\delta}}}}^\infty A_k \Biggr)
      \in G_{\delta\sigma}$.
  \end{enumerate}
  У всех этих множеств со сложными индексами $\delta$~--- пересечение, $\sigma$~--- объединение,
  $G$~--- операция над открытыми в самом начале, $F$~--- над замкнутыми.
\end{exmp}

\paragraph{Мера}
\label{par:meas::meas}

\begin{defn}\label{defn:meas::meas}
  Пусть задано $X$, $\alg \subset \pset (X)$, $A_k \in \alg$. Тогда $\mu \colon \alg \to [0;
  +\infty]$~--- мера, если 
  \begin{enumerate}
    \item $\mu(\varnothing) = 0$
    \item $\displaystyle\mu\underbrace{\left(\bigsqcup_{k=1}^\infty A_k\right)}_{\in \alg} 
      = \sum_{k=1}^\infty \mu(A_k)$. Здесь никто не обещает, что будет именно $\sigma$-алгебра.
  \end{enumerate}
  Множества $A \in \alg$ в таком случае называются $\mu$-измеримыми.
\end{defn}

\begin{exmp}\label{exmp:meas::meas::delta}
  $a \in X$, $\displaystyle\mu(A) = \begin{cases}
    1, & a\in A \\
    0, & a \not\in A
  \end{cases}$ ~--- $\delta$-мера Дирака.
\end{exmp}
\begin{exmp}\label{exmp:meas::meas::mol}
  $a_k \in x$, $m_k \geqslant 0$, $\displaystyle\mu(a) := \sum_{\mathclap {k\colon a_k \in a}} m_k$~---
  <<молекулярная>> мера.
\end{exmp}

\begin{exmp}\label{exmp:meas::meas::cnt}
  $\mu(A) = \# A$~--- считающая мера. \note{она считает, не считывает $\ddot\smile$}
\end{exmp}

\paragraph{Свойства меmы}

Здесь всюду будем рассматривать тройку $(X, \alg \subset \pset (X), \mu)$

\begin{prop}[Монотонность меры]\label{prop:meas::meas::monot}
  Пусть $A,B \in \alg$, $A \subset B$. \par Тогда $\mu(A) \leqslant \mu(B)$.
\end{prop}
\begin{prop}\label{prop:meas::meas::diff}
  Пусть $A,B \in \alg$, $A \subset B$, $\mu(B) < +\infty$.  \par 
  Тогда $ \mu(B\setminus A) = \mu(B) - \mu(A)$.
\end{prop}
\begin{prop}[Усиленная монотонность]\label{prop:meas::meas::enfmont}
  Пусть $A_{1 \intrng n}, B \in \alg$, $A_{1 \intrng n} \subset B$ и дизъюнктны. \par
  Тогда $\displaystyle\sum_{k=1}^n \mu(A_k) \leqslant \mu B$
\end{prop}

\begin{prop}[Полуаддитивность меры]\label{prop:meas::meas::semiadd}
  Пусть $B_{1 \intrng n}, A \in \alg$, $A \subset \displaystyle \bigcup_{k=1}^n B_k$. \par
  Тогда $\displaystyle\mu A \leqslant \sum_{k=1}^n \mu(B_k)$.
\end{prop}
\begin{lproof}
  Сделать $B_k$ дизъюнктными: $C_k = B_k \setminus \bigcup_{j <k} B_k$. Затем представить $A$ как
  дизъюнктное объединение $D_k \colon D_k = C_k \cap A$. Так можно сделать, потому что
  \[
    A = A \cap \bigcup_{k=1}^n B_k = A \cap \bigcup_{k=1}^n C_k = \bigcup_{k=1}^n A \cap C_k
  \]
  Ну а тогда
  \[
    \mu (A) = \sum_k \mu D_k \leqslant \sum_k \mu C_k \leqslant \sum_k \mu B_k
  \]
\end{lproof}


\begin{prop}[Непрерывность меры снизу]\label{prop:meas::meas::bcont}
  Пусть $A_1 \subset A_2 \subset \cdots$, $A_k \in \alg$,
  $\displaystyle A = \bigcup_{k=1}^\infty A_k \in \alg$.
  \note{Опять-таки никто не сказал, что $\alg$~--- $\sigma$-алгебра.} \par
  Тогда $\displaystyle \mu A = \lim_{n\to \infty} \mu A_n$
\end{prop}
\begin{prop}[Непрерывность меры сверху]\label{prop:meas::meas::ucont}
  Пусть $A_1 \supset A_2 \supset \cdots$, $A_k \in \alg$, 
  $\displaystyle A = \bigcap_{k=1}^\infty A_k \in \alg$, $\mu A_1 < +\infty $.
  \par
  Тогда $\displaystyle \mu A = \lim_{n\to \infty} \mu A_n$
\end{prop}

\plholdev{Тут будет картинка про метод исчерпывания Евдокса}

\begin{defn}\label{defn:meas::ledeg::compl}
  Пусть задана тройка $(X, \alg_\sigma, \mu)$. Тогда $\mu$~--- полная, если 
  \[
    \forall\, \in \alg \colon \mu A = 0 \; \forall\, B \subset A, B \in \alg \;::\; \mu B = 0
  \]
\end{defn}

\begin{defn}\label{defn:meas::ledeg::sfin}
  Мера $\mu$  на $\alg$ называется $\sigma$-конечной, если 
  \[
    \exists\, X_n \in \alg, \mu X_n < + \infty \that \bigcup_{n=1}^\infty X_n = X
  \]
\end{defn}

\begin{defn}\label{defn:meas::ledeg::mcont}
  Пусть $\alg_1, \alg_2$~--- сигма-алгебры подмножеств $X$, $\alg_1 \subset \alg_2$,
  $\mu_1 \colon A_1 \to [0;+\infty] $, $\mu_2 \colon A_2 \to [0;+\infty]$. 
  Тогда $\mu_2$ называется продолжением $\mu_1$.
\end{defn}

\begin{thrm}[Лебега-Каратеодора]\label{thrm:meas::ledeg::dist}
  Пусть $\mu$~--- сигма-конечная мера на $\alg$. Тогда
  \begin{enumerate}
    \item Существуют её полные сигма-конечные продожения
    \item Среди них есть наименьшее: $\ov-\mu$. 
      Её ещё называют стандартным продолжением.
  \end{enumerate}
\end{thrm}
\plholdev{идея доказательства}
Пока надо запомнить, что стандратное продолжение~--- сужение внешней <<меры>> 
на хорошо разбивающие множества.


\paragraph{Объём в \texorpdfstring{\(\R^n\)}{R\^{}n}. Мера Лебега }
\label{par:meas::lebeg}

\begin{defn}\label{defn:meas::lebeg::cell}
  Пусть $\Delta = \Delta_1 \times \dotsm \times \Delta_n$, $\Delta_k = [a_k, b_k)$.
  Тогда
  \[
    \begin{split}
      v_1 \Delta_k \equiv | \Delta_k| &:= \begin{cases}
        b_k - a_k, & a_k \in \R \land b_k \in \R \\
        \infty, & \text{иначе}
      \end{cases} \\ 
      v \Delta \overset{(\in R^n)}\equiv v_n \Delta &:= |\Delta_1| \dotsm |\Delta_n|
    \end{split}
  \]
  Для всего, что $\in \Cell_n$, представим его в виде дизъюнктного объединения $\Delta_j$.
  Тогда $vA := \sum_{j=1}^q v \Delta_j $.
\end{defn}

\begin{rem*}
  Здесь радикально всё равно, входят ли концы~--- у них мера ноль. 
\end{rem*}

\begin{thrm}\label{thrm:meas::ledeg::finadd}
  $v$~--- конечно-аддитивен, то есть \[
    \forall\, A, A_{1 \intrng p} \in \Cell\, , A = \bigsqcup_{k=1}^p A_k
    \:\; \Rightarrow  vA = \sum_{k=1}^p v A_k
  \]
\end{thrm}

\begin{thrm}\label{thrm:meas::ledeg::infadd}
  $v$~--- счётно-аддитивен, то есть \[
    \forall\, A, A_{1 \intrng } \in \Cell\, , A = \bigsqcup_{k=1}^\infty  A_k
    \:\; \Rightarrow  vA = \sum_{k=1}^ \infty  v A_k
  \]
\end{thrm}
\begin{tproof}
  Здесь в конспекте лишь частный случай про ячейки.
\end{tproof}



\begin{defn}[Мера Лебега]\label{defn:meas::ledeg::lebeg}
  $X=\R^n$, $\alg= \Cell_n$. Тогда $\lambda_n = \ov-{v_n}$, $\mathpzc M = \ov-{\alg}$~--- мера
  Лебега и алгебра множеств, измеримых по Лебегу, соответственно.
\end{defn}

\subparagraph{Свойства меры Лебега}

\begin{enumerate}[(1) $\triangleright$]
  \item $\lambda \{x\} = 0$
  \item $\lambda \bigl(\{x_k\}_{k}\bigr) = 0$
  \item $\borel \subset \lebesgue$
  \item $L \subset \R^m, m<n \Rightarrow \lambda_n L  = 0$
\end{enumerate}

А это уже целая теормема.
\begin{thrm}[Регулярность меры Лебега]\label{thrm:meas::ledeg::reg}
  Пусть $A \in \lebesgue$, $\varepsilon > 0$. Тогда 
  \[
    \exists\, G \in \open, F \in \closed \that F \subset A \subset G \land 
    \begin{cases}
      \lambda(G \setminus A) < \varepsilon \\
      \lambda(A \setminus F) < \varepsilon 
    \end{cases}
  \]
\end{thrm}
\begin{tproof}
  куча скучных оценок квадратиками.
\end{tproof}

\plholdev{Пример неизмеримого множества на окружности}


\paragraph{Измеримые функции}
\label{par:meas::mfun}

\begin{defn}\label{defn:meas::mfun}
  Пусть задана тройка $(X, \alg_\sigma, \mu)$. Пусть ещё $f\colon X \to \R$. Тогда 
  $f$ называется измеримой относительно $\alg$, если 
  \[
    \forall\, \Delta \subset \R  \holds f^{-1} (\Delta) \in \alg
  \]
\end{defn}

\begin{thrm}\label{thrm:meas::mfun::diftyps}
  Пусть $f$ измеримо относительно $\alg$. Тогда измеримы и следующие (Лебеговы) множества
  \begin{description}
    \item[1 типа] $\{x \in X \mid f(x) < a\} \equiv X [f < a]$
    \item[2 типа] $\{x \in X \mid f(x) \leqslant a\} \equiv X [f \leqslant a]$
    \item[3 типа] $\{x \in X \mid f(x) > a\} \equiv X [f > a]$
    \item[4 типа] $\{x \in X \mid f(x) \geqslant a\} \equiv X [f \geqslant a]$
  \end{description}
  При этом верно и обратное: если измеримы множества какого-то отдного типа, то
  $f$ измерима.
\end{thrm}

\begin{thrm}\label{thrm:meas::mfun::sp}
  Пусть $f_1, \dotsc, f_n$ измеримы относительно $\alg$ и $g\colon \R^n \to R$  
  непрерывна. Тогда измерима и $\varphi(x) = g(f_1(x), \dotsc, f_n(x))$.
\end{thrm}
\begin{rem*}
  В частности, $f_1 + f_2$ измерима.
\end{rem*}

\begin{thrm}\label{thrm:meas::mfun::lims}
  Пусть $f_1,f_2, \dotsc $ измеримы относительно $\alg$ .
  Тогда измеримы $\sup f_n, \inf f_n, \liminf f_n, \limsup f_n, \lim f_n $.
  Последний, правда, может не существовать.
\end{thrm}
\begin{tproof}
  Следует из непрерывности меры.
\end{tproof}

\begin{defn}\label{defn:meas::mfun::simp}
  Пусть $f \colon X \to \R$~--- измерима. Тогда она называется простой, если
  принимает конечное множество значений.
\end{defn}

\begin{defn}[Индикатор множества]\label{defn:meas::mfun::ind}
  \[
    E \subset X , \ind_E := \begin{cases}
      1, & x\in E \\
      0, & x\not\in E
    \end{cases}
  \]
  Он, как видно совсем простая функция.
\end{defn}

\begin{stat}\label{stat:meas::mfun::simpind}
  $f$~--- простая $ \Rightarrow f= \dsum_{k=1}^p c_k \ind_{E_k}$ 
\end{stat}

\begin{thrm}\label{thrm:meas::mfun::simpseq}
  Пусть $f\colon X \to \R$, измерима, $f \geqslant 0$. Тогда 
  \[
    \exists\, (\varphi_n)\colon 0 \leqslant \varphi_1 \leqslant \varphi_2 \leqslant \cdots
    \that \varphi_n \nearrow f \text{ (поточечно)}
  \]
\end{thrm}

\paragraph{Интеграл по мере}
\label{par:meas::int}

\begin{defn}\label{defn:meas::int}
  Пусть задана тройка $(X, \alg_\sigma, \mu)$, $f$~--- измерима.
\begin{enumerate}[{[1]}]
    \item $f$~--- простая. 
      \[
        \int _X f \, \del \mu := \sum_{k=1}^p c_k \mu E_k 
      \]
    \item $f \geqslant 0$.
      \[
        \int_X f \, \del \mu := \sup \Biggl\{ \int_X g \, \del \mu \:\Biggl|\: g\text{-простая},
          0 \leqslant g \leqslant f \Biggr\}
      \] 
    \item $f$ общего вида.
      \[
         \begin{split}
           f_+ &= \max \{f(x),0\} \\
           f_- &= \max \{-f(x),0\} \\
           \int _X f \, \del \mu = \int _X f_+ \, \del \mu - \int _X f_- \, \del \mu
         \end{split}
      \]
      Здесь нужно, чтобы хотя бы один из интегралов в разности существовал.
  \end{enumerate}
\end{defn}
\begin{rem}
  $\displaystyle
    \int _A f \, \del \mu := \sum_{k=1}^p c_k \mu (E_k \cap A)
  $
\end{rem}
\begin{rem}
  Дальше измеримость и неотрицательность или суммируемость $f$ будет периодически 
  называться <<обычными>> условиями.
\end{rem}


\begin{prop}\label{prop:meas::mfun::intind}
  $\displaystyle \int_A f \, \del \mu = \int _X f \cdot \ind _A \, \del \mu$.
\end{prop}

\subparagraph{Свойства интеграла от неотрицательных функций}

Здесь всюду функции неотрицательны и измеримы, что не лишено отсутствия внезапности.
\begin{enumerate}[\fbox{$\text{А}_{\arabic*}$}]
  \item $0 \leqslant f \leqslant g$. Тогда
    $\displaystyle \int_X f\, \del \mu  \leqslant \int_X g\, \del \mu$.
  \item $A \subset B \subset X$, $A , B \in \alg$, $f \geqslant 0$, измерима.
    Тогда $\displaystyle \int_A f \, \del \mu \leqslant \int_B f\, \del \mu$
  \item см теорему \ref{thrm:meas::levi}.
  \item $\displaystyle \int_X (f+g)\, \del \mu = \int_X f\,\del mu + \int_X g\,\del mu $
  \item $\displaystyle \int_X (\lambda g)\, \del \mu =  \lambda \int_X f\,\del mu $
\end{enumerate}

\paragraph{Теорема Беппо \texorpdfstring{Л\'eви}{Леви}}
\label{par:meas::levi}

\begin{thrm}\label{thrm:meas::levi}
  Пусть $(f_n)$~--- измеримы на $X$, $0 \leqslant f_1 \leqslant \cdots $, 
  $f = \lim_n f_n$. Тогда 
  \[
    \int_X f \, \del \mu  = \lim_{n\to \infty} \int_X f_n \, \del \mu 
  \]
\end{thrm}

\paragraph{Свойства интеграла от суммируемых функций}
\label{par:meas::summprop}

\begin{defn}\label{defn:meas::summprop::summ}
  $f$~--- суммируемая (на $X,\mu$), если $\displaystyle \int _X f \, \del \mu < \infty$.
  Весь класс суммируемых (на $X,\mu$) функций обозначается через $\summb (X,\mu)$.
\end{defn}

Здесь всюду функции $\in \summb$
\begin{enumerate}[\fbox{$\text{B}_{\arabic*}$}]
  \item $\displaystyle
    f \leqslant g \Rightarrow \int_X f \, \del \mu \leqslant \int_X g \, \del \mu$.
  
  \item $\displaystyle
    \int_X (f \pm g)\, \del \mu =  \int_X f \, \del \mu \pm \int_X g \, \del \mu$.

  \item $\displaystyle
    \int_X \lambda f \, \del \mu =  \lambda \int_X f \, \del \mu$.
  
  \item $\displaystyle
    |f| \leqslant g  \Rightarrow \left| \int_X f \, \del\mu \right|\leqslant \int_X g \, \del \mu$.

  \item $\displaystyle
    \left| \int_X f \, \del \mu \right| \leqslant \int_X |f| \, \del \mu$.
  
  \item $\displaystyle
    f \in \summb \Leftrightarrow |f| \in \summb$

  \item $\displaystyle
    |f| \leqslant M \leqslant +\infty 
    \Rightarrow \left| \int_X f \, \del\mu \right| \leqslant M \mu X$
\end{enumerate}

\paragraph{Счётная аддитивность интеграла}
\begin{thrm}\label{thrm:meas::infadd}
  Пусть задана тройка $(X,\alg,\mu)$, $f$~--- измерима и $f \geqslant 0 \lor f \in \summb$. 
  Пусть к тому же 
  \[
    A, A_{1\intrng} \subset X, A = \bigcup_{n=1}^\infty A_n
  \]
  Тогда 
  \[
    \int _A f \, \del \mu = \sum_{n=1}^\infty \int _{A_n} f\, \del \mu 
  \]
\end{thrm}

\paragraph{Абсолютная непрерывность интеграла}
\begin{thrm}\label{thrm:meas::abscont}
  Пусть $f\in \summb(X,\alg,\mu)$. Тогда
  \[
    \forall\, \varepsilon > 0 \; \exists\, \delta > 0 \that \forall\, A \in \alg , A \subset X
    \colon \mu A < \delta \holds \left| \int_A f \, \del \mu \right| < \varepsilon 
  \]
\end{thrm}

\paragraph{Интеграл от непрерывной функции по мере Лебега}
\label{par:meas::contint}

\begin{thrm}\label{thrm:meas::contint}
  Пусть $f \in C([a;b])$, $\lambda$~--- мера Лебега на $X = [a;b]$. Тогда
  \[
    f \in \summb , \; \int_{[a;b]} f \, \del \mu = \int_a ^b f = F(b) - F(a),
  \]
  где $F$~--- первообразная $f$.
\end{thrm}

\paragraph{Сравнение подходов Римана и Лебега}
\label{par:meas::rimleb}

Сначала вспомним определения того, о чём собираемся рассуждать.

% нет метки, а то бы и здесь сослался
\begin{defn}[Интеграл Римана]\label{defn:meas::rimleb::rim}
  Пусть $f\in C([a;b])\,\; a < x_1 < \dots < x_{n-1} < x_n = b,\; 
  \xi_i\in[x_i;x_{i+1}]$. Тогда 
    \begin{itemize}
      \item $\tau = \{x_1,\dots,x_{n-1}\}$~--- разбиение отрезка $[a;b]$
      \item $\xi = \{\xi_1,\dots,\xi_{n-1}\}$~--- оснащение разбиения $\tau$
      \item $\Delta x_i = x_{i+1}-x_i$~--- длина $i$-го отрезка
      \item $r=r(\tau) = \max\limits_i\{\Delta x_i\}$~--- ранг разбиения
      \item $\sigma = \sigma(\tau,\xi,f):= \dsum\limits_{i=0}^{n-1}f(\xi_i)\cdot\Delta x_i$ ~---
        сумма Римана
    \end{itemize}
    Сам интеграл определяется как-то так 
    \[
      \int_a^b f \, \del x = \lim_{r(\tau)\to 0} \sigma(\tau,\xi,f)
    \]
\end{defn}

\begin{defn}[Интеграл Лебега]\label{defn:meas::rimleb::leb}
  см.~\ref{defn:meas::int}. В качестве множества $X$ понятное дело, отрезок $[a;b]$.
\end{defn}

\begin{exmp}\label{exmp:meas::rimleb::notrim}
  Пусть $X=[0;1]$. Тогда $f(x) = \begin{cases}
    0, & x\not\in \Q \\
    1, & x\in \Q 
  \end{cases}$ интегрируема по Лебегу, но не по Риману.
\end{exmp}

\plholdev{картиночка с обоими интегралами}


\paragraph{Сравнение несобственного интеграла и интеграла Лебега}
\label{par:meas::impleb}

\begin{thrm}\label{thrm:meas::impleb}
  Пусть $f \geqslant 0 \lor f \in \summb\bigl([a;b\bigr),\lambda)$. Тогда
  $\displaystyle \int_{[a;b)} f \, \del \lambda = \int _a ^{\to b} f$.
\end{thrm}
\begin{tproof}\underdev
  Cвести к собственному, а дальше непрерывность меры.
\end{tproof}

\paragraph{Интеграл по дискретной мере и мере, задаваемой плотностью}
\label{par:meas::discint}

\begin{thrm}\label{thrm:meas::discint::mol}
  Пусть $\mu = \sum_{k} m_k \delta_{a_k}$, $\{a_k\} \in X$ и  $f\colon X \to \R$, 
  $f \geqslant 0$ или $f\in \summb (X,\mu)$.
  Тогда \[
    \int_X f \, \del \mu = \sum_{k} f(a_k) \cdot \underbrace{m_k}_{\mu \{a_k\}}
  \]
\end{thrm}

\begin{tproof}\underdev
  Счётная аддитивность интеграла поможет. \ref{thrm:meas::infadd}
\end{tproof}


\begin{exmp}\label{exmp:meas::discint::series}
  Пусть $\mu A=\#A$. Тогда \[
    \sum_{m,n\in\N} = \int _{\N^2} f(m,n) \, \del \mu
  \]
  Причем условия суммируемости \note{ 
  здесь объявим бесконечность приличным значением суммы ряда } 
  ряда такие же, как у интеграла Лебега: 
  \[
    \left[\;
    \begin{split}
      \forall\, m,n\in \N \holds a_{m,n} &\geqslant 0 \\
      \sum_{m,n\in\N} |a_{m,n}| &< \infty \\
    \end{split} \right.
  \]
\end{exmp}


\begin{defn}\label{defn:meas::discint::dens}
  Пусть задана пара \note{тройка, но все же поняли, что сигма-алгебра имелась в виду} 
  $(X, \mu)$, $\rho \colon X \to \R$, измерима, $\rho \geqslant 0$. 
  Тогда 
  \begin{itemize}
    \item $\displaystyle \nu(E) := \int_E \rho \, \del \mu $~--- мера, задаваемая плотностью
      $\rho$
    \item $\rho$~--- плотность меры $\nu$ относительно меры $\mu$.
  \end{itemize}
\end{defn}

\begin{rem*}
  Она правда мера, интеграл счётно-аддитивен.
\end{rem*}

\begin{thrm}\label{thrm:meas::discint::intchg}
  Пусть выполнены <<обычные>> условия на $f$. Тогда 
  $\displaystyle \int_X f \, \del \nu = \int_X f \rho \, \del \mu$.
\end{thrm}

\paragraph{Мера Лебега-Стилтьеса. Интеграл по распределению}
\label{par:meas::lebstil}

\begin{defn}\label{defn:meas::lebstil::meas}
  Пусть $I \subset \R$, $F \colon I \to \R$, $F \nearrow$, $F(x) = F(x-0)$ 
  (непрерывна слева).\note{А можно и без. Тогда $\nu([a;b)) = F(b-0) - F(a-0)$,
  см.~\ref{makpodk}}.
  Рассмотрим порождённую полуинтервалами $[a;b) \subset I$ $\sigma$-алгебру.
  Введём <<объём>> $\nu_F \colon \nu([a;b)) = F(b) - F(a)$.
  
  Тогда мера Лебега-Стилтьеса $\mu_F$~--- стандартное продолжение $\nu_F$ на некоторую
  $\sigma$-алгебру $\mathcal M_F$.
\end{defn}

\begin{rem}
  Здесь надо доказывать \emph{счётную} аддитивность, а то как продолжать $\nu$, если она~--- не
  мера?
\end{rem}

\subparagraph{Свойства мемы Лебега-Стилтьеса}

\begin{prop}\label{prop:meas::lebstil::clos}
  Пусть $\Delta = [a;b]$. Тогда $\mu \Delta  = F(b+0) - F(a)$.
\end{prop}

\begin{prop}\label{prop:meas::lebstil::point}
  Пусть $\Delta = \{a\}$. Тогда $\mu \Delta  = F(a+0) - F(a)$.
\end{prop}

\begin{prop}\label{prop:meas::lebstil::open}
  Пусть $\Delta = (a;b)$. Тогда $\mu \Delta  = F(b) - F(a+0)$.
\end{prop}

\begin{lem}\label{lem:meas::lebstil::smoothF}
  Пусть $F \in C(I)$, $\Delta \subset I$.
  Тогда $\mu_F (\Delta) = \dint_\Delta F'(t) \, \del \lambda$.
\end{lem}

\begin{thrm}\label{thrm:meas::lebstil::int}
  Пусть $F\nearrow$, кусочно-гладка на $I \subset \R$, а для $f$ выполнены обычные
  условия ($X = \borel$, $\mu = \mu_F$). Промежутки гладкости $F$ обозначим за $(c_k, c_{k+1})$.
  Тогда 
  \[
    \int_X f\, \del \mu_F = \sum_k \int_{c_k}^{c_{k+1}} f F' \, \del \lambda +
\sum_k f(c_k) \, \underbrace{\Delta_{c_k} F}_{\mathclap{\text{скачок в $c_k$}}}
  \]
\end{thrm}

\begin{defn}[Образ мемы]\label{defn:meas::lebstil::imag}
  Пусть $(X,\alg,\mu)$~--- пространство с мемой, $f\colon X \to Y$. 
  Превратим и $Y$ в пространство с мемой.
  \begin{enumerate}
    \item $\alg' = \{E \subset Y \mid f^{-1} (E) \in \alg\}$.
    \item $\mu' \equiv \nu = \mu \circ f^{-1} $.
  \end{enumerate}
\end{defn}

\begin{thrm}\label{thrm:meas::lebstil::imfun}
  Пусть для $g\colon Y \to \R$ выполнены обычные условия ($\alg = \alg'$, $\mu=\nu$).
  Тогда $\displaystyle \int_Y g \, \del \nu  = \int_X (g \circ f)\, \del \mu$.
\end{thrm}

\begin{defn}[Функция распределения]\label{defn:meas::lebstil::distr}
  Пусть задано $(X, \mu)$, $\mu X < + \infty$, $f\colon X \to \R$. Тогда
  $F(t) := \mu X  [f <t]$. Как видно, она возрастает и непрерывна слева.
\end{defn}

\begin{thrm}\label{thrm:meas::lebstil::distint}
  Пусть задано $(X, \mu)$, $\mu X < + \infty$, выполнены обычные условия для $f$.
  Тогда $\displaystyle \int_X f \, \del \mu = \intR t \, \del \mu_F$.
\end{thrm}

\paragraph{Интеграл Эйлера-Пуассона}
\label{par:meas::eulpuass}

\begin{prop}\label{prop:meas::eulpuass}
  $\displaystyle \int_{\R^2} e^{-(x^2 + y^2)}\, \del \lambda_2 = \pi $ 
\end{prop}

\paragraph{Вероятностный смысл мемы}
\label{par:meas::prob}

\plholdev{Табличка с соответствием}

\paragraph{Геометрический смысл меры Лебега. Принцип Кавальери}
\label{par:meas::geomleb}

\begin{defn}\label{defn:meas::almev}
  Пусть задано $(X,\mu)$, $P(x)$~--- предикат. Говорят, что $P(x) = 1$ почти везде (\alev),
  если $\mu \{x \mid P(x) = 0\} = 0$.
\end{defn}

\begin{defn}\label{defn:meas::almev::eq}
  $f \sim g \Leftrightarrow f(x) = g(x) \alev$.
\end{defn}

\begin{lem}[Беппо-Леви для рядов]\label{lem:meas::almev::blseries}
  Пусть заданы $(X,\mu)$, $u_n \colon X \to \R$, $n\in \N$, $u_n$ измеримы, $u_n \geqslant 0$.
  Тогда 
  \begin{enumerate}[a)]
    \item $\displaystyle \int _x \sum_{n=1}^\infty u_n \, \del \mu  
      = \sum_{n=1}^\infty \int _x  u_n \, \del \mu$.
    \item Если эти числа конечны, то ряд $\sum_n u_n \conv \alev$
  \end{enumerate}
  
\end{lem}
\begin{lem}[Беппо-Леви <<вверх ногами>>]\label{lem:meas::almev::blov}
  Пусть задано $(X,\mu)$, $(f_n)$, измеримы, $f_1 \geqslant f_2 \geqslant \cdots \geqslant 0$.
  Пусть ещё $f_1 \in \summb$. Тогда
  \[
    \lim_{n\to \infty} \int_X f_n \, \del \mu = \int_X \lim_{n\to \infty} f_n \, \del \mu 
  \]
\end{lem}

\plholdev{Здесь была ещё пара лемм, но они не особо используются дальше. Вроде}

\begin{defn}\label{defn:meas::almev::proj}
  Пусть $E \subset \R^m \times \R^n \in \lebesgue_{m+n}$.
  \begin{itemize}[$\triangleright$]
    \item $E_x      = \{ y\in \R^n \mid (x,y) \in E\}$~--- <<срез>>
    \item $\Pi_1(E) = \{ x\in \R^m \mid E_x \neq \varnothing\}$~--- <<проекция>>
  \end{itemize}
\end{defn}

\plholdev{картиночка для $\R^2$}

\begin{thrm}\label{thrm:meas::almev::cav}
  Пусть $E \in \lebesgue_{m+n}$, $E_x \in \lebesgue_{n} \alev x$,
  $\varphi(x) = \lambda_n(E_x)$ измерима относительно $\lebesgue_{m}$.

  Тогда 
  \[
    \lambda_{m+n} (E) = \int_{\R^m} \lambda_n (E_x) \, \del \lambda_m
  \]
\end{thrm}
\plholdev{много букв}

\begin{defn}[График]\label{defn:meas::almev::plot}
  $\Gamma^f = \{ (x,t) \in \R^{n+1} \mid t = f(x)\}$.
\end{defn}
\begin{defn}[Подграфик]\label{defn:meas::almev::subplot}
  $\Gamma_-^f = \{ (x,t) \in \R^{n+1} \mid 0 \leqslant t \leqslant f(x)\}$.
\end{defn}
\begin{defn}[Надграфик]\label{defn:meas::almev::supplot}
  $\Gamma_+^f = \{ (x,t) \in \R^{n+1} \mid t \geqslant f(x)\}$.
\end{defn}

\begin{thrm}[Геометрический смысл интеграла]\label{thrm:meas::almev::geomsense}
  Пусть $f\colon \R^n \to \R$, измерима, $ \geqslant 0$. Тогда
  \begin{enumerate}
    \item $\Gamma_-^f$ измеримо.
    \item $\lambda_{n+1}\Gamma_-^f = \int_{\R^n} f\, \del \lambda_n$ измеримо.
  \end{enumerate}
\end{thrm}

\paragraph{Сведение кратного интеграла к повторному}
\label{par:meas::mult}

Будем в дальнейшем обозначать интегрирование по мере через $\del x$ (ну или
$\del y$), размерность определяется из размерности $x$. Еще обозначим $\del (x,y)$ через
$\del x\del y$. 

\begin{thrm}[Тонелли]\label{thrm:meas::mult::tonn}
  Пусть $f\colon \R^{m+n} \to \R$, измерима, $ \geqslant 0$, $x\in \R^m$, $y \in \R^n$.
  Тогда
  \[
    \iint_{\R^m \times \R^n} f(x,y) \, \del x\del y 
    = \int_{\R^m} \del x \int_{\R^n} f(x,y) \, \del y
  \]
\end{thrm}

\begin{thrm}[Фубини]\label{thrm:meas::mult::fub}
  Пусть $f\colon \R^{m+n} \to \R$, измерима, $\in \summb(\R^{n+m}, \lambda_{m+n})$,
  $x\in \R^m$, $y \in \R^n$. Тогда
  \[
    \iint_{\R^m \times \R^n} f(x,y) \, \del x\del y 
    = \int_{\R^m} \del x \int_{\R^n} f(x,y) \, \del y
  \]
\end{thrm}

\paragraph{Мера Лебега и аффинные преобразования}
\label{par:meas::aff}

Главные герои этого параграфа:

\begin{itemize}[$\bigcirc$]
  \item Сдвиг: $T \colon \R^n \to \R^n$, $Tx = x+a$, $a\in \R^n$.
  \item Поворот с растяжением: $L \colon \R^n \to \R^n$, $L$~--- линейный император.
\end{itemize}

\begin{stat}\label{stat:meas::aff::shiftmeas}
  $E \in \lebesgue \Rightarrow T(E) \in \lebesgue$. 
\end{stat}
\begin{stat}\label{stat:meas::aff::linmeas}
  $E \in \lebesgue \Rightarrow L(E) \in \lebesgue$. 
\end{stat}

\begin{stat}\label{stat:meas::aff::disturb}
  Пусть $L \colon \R^n \to \R$, линейно. Тогда 
  \[
    \exists\, C \geqslant 0 \forall\, E \in \lebesgue \holds \lambda L(E) = C \lambda E
  \]
\end{stat}

\begin{thrm}\label{thrm:meas::aff::deter}
  $C$ из прошлой теоремы равно $\bigl|\det [L]\bigr|$.
\end{thrm}
\plholdev{тут декомпозиция на ортогональный и диагональные операторы и 2 леммы}

\paragraph{Мера образа при гладком отображении}
\label{par:meas::smoothimgmeas}

{\denot $J_F(x) \equiv \det F'(x)$}

\begin{thrm}\label{thrm:meas::smoothimgmeas}
  Пусть $E \in \lebesgue$, $F \colon G \subset \R^n \to R^n$, гладкая биекция.
  Тогда $F(E) \in \lebesgue$ и $\lambda F(E) = \dint_E |\det F'(x) |\del x$.
\end{thrm}
\begin{tproof}
  \sour\underdev
\end{tproof}


\paragraph{Глакая замена переменной в интеграле}
\label{par:meas::smoothvarch}

\begin{thrm}\label{thrm:meas::smoothvarch}
  Пусть $F \colon G \subset \R^n \to R^n$, гладкая биекция.
  Пусть к тому же $E \subset F(G) \in \lebesgue$, $f\colon E \to \R$ с обычными условиями.

  Тогда\[
    \int_E f(y) \, \del y = \int_{F^{-1}(E)} f(F(x)) \cdot| J_F(x) | \, \del x
  \]
\end{thrm}

\begin{exmp}[Полярные координаты]\label{exmp:meas::smoothvarch::polar}
  \underdev
  $|J| = r$
\end{exmp}

\begin{exmp}[Сферические координаты]\label{exmp:meas::smoothvarch::sph}
  \underdev
  $|J| = r^2 \cos \psi $
\end{exmp}

\paragraph{Предельный переход под знаком интеграла}
\label{par:meas::limint}

\begin{defn}[Всякие сходимости]\label{defn:meas::limint::conv}
  Пусть $(f_n) \colon X \to \R$, $f \colon X \to \R$, $\mu$~--- мера на $X$.
  \begin{align*}
    & f_n \to f       & &:= & &\forall\, x \in X \holds f_n(x) \to f(x)      \\
    & f_n \unito^X f  & &:= & &\sup\limits_X |f_n-f| \to 0                   \\
    & f_n \to f \alev & &:= &&\exists\, N \subset X \colon \mu(N) = 0 
    \that \forall\, x\in X \setminus N \holds f_n(x) \to f(x).            \\
    & f_n \to^\mu f & &:= &&\forall\, \sigma > 0 \holds \mu X [ |f_n - f| \geqslant \sigma ]\to 0
    \end{align*}
\end{defn}

\begin{rem}\label{rem:meas::limint::convseq}
  $ f \unito^X f \Rightarrow f_n \to f \Rightarrow f_n \to f \alev$.
\end{rem}
\begin{rem}\label{rem:meas::limint::convmeas}
  Пусть $\mu X < \infty $, тогда
  $ f_n \to f \alev \Rightarrow f_n \to^\mu f$.
\end{rem}

\begin{rem}[Теорема Рисса]\label{rem:meas::limint::convpv}
  $ f_n \to^\mu  f \alev \Rightarrow \exists\, (n_k) \that f_{n_k} \to f \alev $.
\end{rem}

\begin{thrm}\label{thrm:meas::limint::uniint}
  $f_n \unito^X f, \mu X < \infty \Rightarrow \dint_X f_n \, \del \mu \to \int _X f$
\end{thrm}

\begin{thrm}
  см теорему Беппо-Леви (\ref{thrm:meas::levi}) или её вариацию \ref{lem:meas::almev::blov}.
\end{thrm}

\begin{thrm}[Фату]\label{thrm:meas::limint::fatu}
  Пусть заданы $(X,\mu)$, $f_n \geqslant 0$, измеримы. Тогда
  \[
    \int_X \varliminf f_n \, \del \mu \leqslant \varliminf \int_X f_n \, \del \mu 
  \]
\end{thrm}

\paragraph{Теорема Лебега об ограниченной сходимости}
\label{par:meas::bndconv}

\begin{thrm}\label{thrm:meas::bndconv}
  Пусть снова заданы $(X,\mu)$, $(f_n)$ измерима, $f_n \to f \alev$. К тому же
  \[
    \exists\, \varphi \in \summb \that \forall\, n \holds |f_n|  \leqslant |\varphi|
  \]
  Тогда
  \[
    \lim_{n\to \infty} \int _X f_n \, \del \mu = \int_X f\, \del \mu 
  \]
\end{thrm}

{\denot $(\mathcal L)$~--- условия теоремы Лебега об ограниченной сходимости.}

\begin{cor}
  Пусть $f\colon T \times X \to \R$, $T \subset \R^k$, $f_t \xto{t\to t_0}{} f \alev$,
  и \[
    \exists\, V(t^0), \varphi \in \summb \that \forall\, t \in \punct V \cap T
    \holds |f_t| \leqslant |\varphi| 
  \]
  Тогда
  \[
    \int _X f_t \, \del \mu \xto{t\to t_0}{} \int_X f\, \del \mu 
  \] 
\end{cor}

{\denot $(\mathcal L_{\rm loc})$~--- условия локальной теормемы Лебега об ограниченной сходимости.}

\begin{cor}
  Непрерывность интеграла по параметру при выполнении $(\mathcal L_{\rm loc})$ и 
  непрерывности $f_t$.
\end{cor}

\paragraph*{Интеграл по меме с параметром}

Здесь часто придётся подчёркивать, что является параметром, а что~--- определяет функцию
В таких случаях параметр будет записан, как индекс

\begin{defn}[Собственный интеграл с параметром]\label{defn:meas::paruniconv::prop}
  Пусть $f\colon X \times T \to \R$, $f_t(x)\in \summb([a,b],\mu) \; \forall\, t \in T$. 
  Тогда, 
  \[
    I(t) = \int_a^{b} f(x,t) \, \del x 
  \]
  Мы здесь определяем некоторую функцию от $t$, как видно $\dom_I = T$.
\end{defn}

По идее, надо здесь переформулировать все-все-все утверждения про последовательности функций.
Надо бы узнать, что с этим делать.
\flame 
У нас в конспекте этот кусок почему-то написан про несобственные интегралы, но всюду полагается
$(\mathcal L_{\rm loc})$. Так что по сути они~--- просто интегралы по меме.

Здесь тоже есть непрерывность, дифференциируемость и интегрирование по параметру, но
все тривиально\note{ну..} следует из \ref{thrm:meas::bndconv} и \ref{thrm:meas::mult::fub}.

\paragraph{Равномерная сходимость несобственного параметрического интеграла. Признаки}
\label{par:meas::paruniconv}


\begin{defn}[Несобственный интеграл с параметром]\label{defn:meas::paruniconv::improp}
  Пусть $f\colon X \times T \to \R$,
  $f\in \summb([a,B],\mu) \;\forall\, B <b$. Тогда,
  \[
    I(t) = \int_a^{\to b} f(x,t) \, \del x := \lim_{B\to b-0} \int_a^B f(x,t) \, \del x 
    = \lim_{B\to b-0} I^B(t)
  \]
  Предполагается, что $\forall\, t \in T$ интеграл сходится поточечно. А вот суммируемость
  никто не обещал.
\end{defn}


\begin{defn}\label{defn:meas::paruniconv::uniconv}
  Говорят, что $I^B(t) \unito^T I(t)$ (сходится равномерно относительно $t, t\in T$), если 
  \note{Никто же не любит $\varepsilon$-$\delta$-определения?}
  \[
    \sup_{t\in T} \left| \int_B^{\to b} f(x,t)\right| \xto{B\to b} 0
  \]
\end{defn}

Здесь дальше всюду предполагается поточечная сходимость интеграла $\forall\, t \in T$.

\begin{thrm}[Признак Больцано-Коши]\label{thrm:meas::paruniconv::bk}
  \[
    I^B(t) \unito^T I(t) \Leftrightarrow 
    \sup_{T} \left| \int_{B_1}^{B_2} f(x,t)\, \del x \right| \xto{B_1, B_2 \to b} 0
  \]
\end{thrm}

\begin{thrm}[Признак Вейерштрасса]\label{thrm:meas::paruniconv::wei}
  Пусть $\exists\, \varphi\in \summb([a;b))\that |f(x,t)| \leqslant \varphi (x)\; \forall\, t$.
  Тогда $I^B(t) \unito^T I(T)$.
\end{thrm}

\begin{thrm}[Признак Дирихле]\label{thrm:meas::paruniconv::dir}
  Пусть $I(t) = \dint_{a}^{\to b} f(x,t) \cdot g(x,t) \, \del x$ и
  \begin{enumerate}[a)]
    \item $f(x,t) \unito^T 0$, $f(x,t) \searrow^x$ ($x\to b-0$)
    \item $G(x,t) = \dint_a^x g(\xi, t) \, \del \xi$
      \[
        \exists\, M \colon \forall\, x \in [a;b), t\in T \that | G(x,t) | \leqslant M   
      \]
  \end{enumerate}
  Тогда $I^B(t) \unito^T I(T)$.
\end{thrm}

\begin{thrm}[Признак Абеля]\label{thrm:meas::paruniconv::abel}
  Пусть $I(t) = \dint_{a}^{\to b} f(x,t) \cdot g(x,t) \, \del x$ и
  \begin{enumerate}[a)]
    \item $\exists\, M \colon \forall\, t\in T \holds f(x,t) \leqslant M$,
      $f(x,t) \searrow^x$.
    \item 
      $\displaystyle
      \int_a^{B} g(x,t) \, \del x \unito^T_{B\to b} \int_a^{\to b} g(x,t) \, \del x 
      $
  \end{enumerate}
  Тогда $I^B(t) \unito^T I(T)$.
\end{thrm}

\paragraph{Несобственные интегралы с параметром и операции анализа над параметром \underdev}
\label{par:meas::parimpconv}

\begin{thrm}\label{thrm:meas::parimpconv::lim}
  Пусть $f(x,t) \to f(x,t_0)$ для \alev $x \in [a;b)$ и $I^B(t) \unito^{V(t^0)} I(t)$.
  \note{Это не очень докажется без конечности меры $V(t_0)$ ,а то интеграл может сходится, а
  функция не быть суммируемой}
  Тогда $I \xto{t\to t_0} I(t_0)$.
\end{thrm}

\begin{thrm}\label{thrm:meas::parimpconv::diff}
  Пусть для $\alev x \; \exists\, f_t'(x,t)$, непрерывна на $[a;b) \times \underbrace{[c;d)}_T$. 
  Допустим,
  \begin{enumerate}[a)]
    \item $I(t) = \dint_a^{\to b} f (x,t)\, \del x$ сходится $\forall\, t \in T$
    \item $\dint_a^{\to b} f_t' (x,t)\, \del x$ равномерно сходится относительно $t \in T$
  \end{enumerate}
  Тогда $\exists\, I'(t_0) = \dint_a^{\to b} f_t'(x,t_0)\, \del x$
\end{thrm}
\begin{rem*}
  Здесь нужна сходимость $I$, чтобы хоть где-то были конечные значения $I(t)$, нам их
  разность считать.
\end{rem*}

\begin{thrm}\label{thrm:meas::parimpconv::diff}
  Пусть для $\alev x \; \exists\, f(x,t)$, непрерывна на $[a;b) \times \underbrace{[c;d)}_T$. 
  Допустим,
  $I(t) = \dint_a^{\to b} f (x,t)\, \del x$ равномерно сходится относительно $t \in T$
  
  Тогда \[
    \dint_c^d I(t)\, \del t = \dint_a^{\to b} \del x \int_c^d f(x,t)\, \del t
  \]
\end{thrm}


\paragraph{\texorpdfstring{$\Gamma$}{Г}-функция Эйлера}
\label{par:meas::gamma}

\begin{defn}\label{defn:meas::gamma}
  $\displaystyle \Gamma(t) = \int_0^\infty x^{t-1} e^{-x} \, \del x$
\end{defn}

\subparagraph{Свойства}

\begin{enumerate}[$1^\circ$]
  \item Определена для всех $t>0$.
  \item $\Gamma (1) = 1$
  \item $\forall\, t \Gamma (t+1) = t \Gamma (t)$
  \item $n\in \N$ $\Gamma(n+1) = n!$
  \item $\Gamma$-выпукла
  \item $\Gamma \sim \frac{1}{t} $ при $t\to 0$
  \item $\Gamma(t+1) \sim \sqrt{2\pi} \sqrt{t} t^t e^{-t}$ при $t\to \infty$.
  \item $\Gamma(t) \cdot \Gamma(1-t) = \frac{\pi}{\sin \pi t}$. (формула отражения)
\end{enumerate}

Гамма-функцию можно продолжить на отрицательную область, через формулу отражения.
И на комплексную, там будет сходимость при $\Im z > 0$.

\paragraph{B-функция}
\label{par:meas::beta}

\begin{defn}\label{defn:meas::beta}
  $\displaystyle B(y,z) = \int_0^1 x^{y-1} (1-x)^{z-1} \, \del x$.
\end{defn}

\subparagraph{Свойства}
\begin{enumerate}[$1^\circ$]
  \item $B(y,z) = B(z,y)$.
  \item $B(y,z) = \dfrac{\Gamma(y) \Gamma(z)}{\Gamma(y+z)}$.
\end{enumerate}

\paragraph{Объём $n$-мерного шара}
\label{par:meas::ball}

\begin{thrm}\label{thrm:meas::ball}
  Пусть $B_n(R) = \{x \in \R^n \mid \|x\| \leqslant R \}$~-- $n$-мерный шар.
  Тогда 
  \[
    \lambda_n B_n(R) = \frac{\pi ^{n/2} R^n}{\tfrac{n}{2} \cdot \Gamma(\tfrac{n}{2})} 
  \]
\end{thrm}

\end{document}
% vim: tw=100 cc=100
