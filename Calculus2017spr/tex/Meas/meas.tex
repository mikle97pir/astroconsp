%------------------------------------------------------------
% Description : 
% Author      : Iliya Tikhonenko <iliya.t@mail.ru>
% Created at  : Tue Feb 21 00:00:16 MSK 2017
%------------------------------------------------------------
\documentclass[12pt, timbord]{../../../notes}
\usepackage{silence}
\WarningFilter{latex}{Reference}
\graphicspath{{../../img/}}

\begin{document}
\paragraph{Системы множеств}
\label{par:meas::setsys}

\begin{defn}\label{defn:meas::setsys::sub}
  Пусть здесь (и дальше) $X$~--- произвольное множество. Тогда $\pset(X) \equiv 2^X$~--- множество
  всех подмножеств $X$.
\end{defn}
\begin{exmp*}
  $X = \{1 \intrng n\} \Rightarrow \# \pset (X) = 2^n$ (это количество элементов, если что)
\end{exmp*}

\begin{defn}[Алгебра]\label{defn:meas::setsys::alg}
  Пусть $\alg \subset \pset (X)$. Тогда $\alg$~--- алгебра множеств, если
  \begin{enumerate}
    \item $\varnothing \in \alg$
    \item $X \in \alg$
    \item $A,B\in \alg \Rightarrow A \cap B, A \cup B, A \setminus B \in \alg$
  \end{enumerate}
\end{defn}
\begin{rem*}
  Заметим, что в алгебре пересечение (или объединение) \emph{конечного} числа её элементов лежит в
  алгебре.  Это можно доказать простой индукцией. А вот для бесконечных объединений пользоваться
  индукцией уже нельзя, ведь $\infty \not\in \N$. 
\end{rem*}

\begin{defn}[$ \sigma $-алгбера]\label{defn:meas::setsys::sigalg}
  Пусть $\alg \in \pset(X)$. Тогда $\alg$~--- $ \sigma $-алгебра, если 
  \begin{enumerate}
    \item $\alg$~--- алгебра
    \item $A_1, \dotsc, A_n \in \alg \Rightarrow \bigcup_{k=1}^ \infty A_k \in\alg, 
      \bigcap_{k=1}^{ \infty } A_k \in \alg $
  \end{enumerate}     
\end{defn}

\begin{defn}\label{defn:meas::setsys::sigshell}
  Пусть $\mathcal E \subset \pset(X)$. Тогда 
  \[
  \sigma(\mathcal E) := \bigcap\,\{\alg \mid \alg\text{~--- $\sigma$-алгебра, $\alg \supset
  \mathcal E $}\}
  \]
  эта конструкция~--- сигма-алгебра, просто аксиомы проверить.
\end{defn}

\begin{defn}\label{defn:meas::setsys::borel}
  Пусть $\open$~--- все открытые множества в $\R^n$. Тогда $\mathpzc B_n = \sigma(\open)
  $~--- борелевская $\sigma$-алгебра в $\R^n$.
\end{defn}

\begin{defn}[Ячейка в $\R^n$ ]\label{defn:meas::setsys::cell}
  Обозначать её будем $\Delta^n$, по размерности соответствующего пространства.
  \[
    \Delta^1 = \begin{cases}
      [a;b) \\
      (-\infty;b) \\
      [a;+\infty) \\
      (-\infty;+\infty)
    \end{cases} \quad
    \forall\, n \;\: \Delta = \prod_{k=1}^n \Delta^1 _k 
  \]
  Ещё введём алгебру $\alg = \Cell_n = \{ A \mid A = \bigcup_{k=1}^p \Delta_k\}$
\end{defn} 

\begin{lem}\label{lem:meas::setsys::algsubset}
  Пусть $\mathcal E_1, \mathcal E_2 \subset \pset(X)$, $\sigma(\mathcal E_1) \supset \mathcal E_2$.
  Тогда $\sigma(\mathcal E_1) \supset \sigma(\mathcal E_2)$
\end{lem}
\begin{thrm}\label{thrm:meas::setsys::borelcell}
  $\borel_n = \sigma(\Cell_n)$.
\end{thrm}

\begin{exmp}\label{exmp:meas::setsys::borel}
%   \everymath{\displaystyle}
  Все множества нижё~--- борелевские.
  \begin{enumerate}[$\langle$1$\rangle$]
    \item $\open $.
    \item $\closed =\{A \mid \ov-{A} \in \open \}$.
    \item $\Biggl(A 
      = \bigcap_{\substack{{k=1}\\[0.12em]\mathclap{ G_k \in \open}}}^\infty G_k \Biggr)\in G_\delta$.
    \item $\Biggl(B 
      = \bigcup_{\substack{{k=1}\\[0.12em]\mathclap{ F_k \in \closed}}}^\infty F_k \Biggr)\in
      F_\sigma$.
    \item $\Biggl(C 
      = \bigcup_{\substack{{k=1}\\[0.12em]\mathclap{ A_k \in G_{\delta}}}}^\infty A_k \Biggr)
      \in G_{\delta\sigma}$.
  \end{enumerate}
  У всех этих множеств со сложными индексами $\delta$~--- пересечение, $\sigma$~--- объединение,
  $G$~--- операция над открытыми в самом начале, $F$~--- над замкнутыми.
\end{exmp}

\paragraph{Мера}
\label{par:meas::meas}

\begin{defn}\label{defn:meas::meas}
  Пусть задано $X$, $\alg \subset \pset (X)$, $A_k \in \alg$. Тогда $\mu \colon \alg \to [0;
  +\infty]$~--- мера, если 
  \begin{enumerate}
    \item $\mu(\varnothing) = 0$
    \item $\displaystyle\mu\underbrace{\left(\bigsqcup_{k=1}^\infty A_k\right)}_{\in \alg} 
      = \sum_{k=1}^\infty \mu(A_k)$. Здесь никто не обещает, что будет именно $\sigma$-алгебра.
  \end{enumerate}
\end{defn}

\begin{exmp}\label{exmp:meas::meas::delta}
  $a \in X$, $\displaystyle\mu(A) = \begin{cases}
    1, & a\in A \\
    0, & a \not\in A
  \end{cases}$ ~--- $\delta$-мера Дирака.
\end{exmp}
\begin{exmp}\label{exmp:meas::meas::mol}
  $a_k \in x$, $m_k \geqslant 0$, $\displaystyle\mu(a) := \sum_{\mathclap {k\colon a_k \in a}} m_k$~---
  <<молекулярная>> мера.
\end{exmp}

\begin{exmp}\label{exmp:meas::meas::cnt}
  $\mu(A) = \# A$~--- считающая мера. \note{она считает, не считывает $\ddot\smile$}
\end{exmp}

\subparagraph{Свойства меmы:}

Здесь всюду будем рассматривать тройку $(X, \alg \subset \pset (X), \mu)$

\begin{prop}[Монотонность меры]\label{prop:meas::meas::monot}
  Пусть $A,B \in \alg$, $A \subset B$. \par Тогда $\mu(A) \leqslant \mu(B)$.
\end{prop}
\begin{prop}\label{prop:meas::meas::diff}
  Пусть $A,B \in \alg$, $A \subset B$, $\mu(B) < +\infty$.  \par 
  Тогда $ \mu(B\setminus A) = \mu(B) - \mu(A)$.
\end{prop}
\begin{prop}[Усиленная монотонность]\label{prop:meas::meas::enfmont}
  Пусть $A_{1 \intrng n}, B \in \alg$, $A_{1 \intrng n} \subset B$ и дизъюнктны. \par
  Тогда $\displaystyle\sum_{k=1}^n \mu(A_k) \leqslant \mu B$
\end{prop}

\begin{prop}[Полуаддитивность меры]\label{prop:meas::meas::semiadd}
  Пусть $B_{1 \intrng n}, A \in \alg$, $A \subset \displaystyle \bigcup_{k=1}^n B_k$. \par
  Тогда $\displaystyle\mu A \leqslant \sum_{k=1}^n \mu(B_k)$.
\end{prop}
\begin{itlproof}
  Сделать $B_k$ дизъюнктными: $C_k = B_k \setminus \bigcup_{j <k} B_k$. Затем представить $A$ как
  дизъюнктное объединение $D_k \colon D_k = C_k \cap A$. Так можно сделать, потому что
  \[
    A = A \cap \bigcup_{k=1}^n B_k = A \cap \bigcup_{k=1}^n C_k = \bigcup_{k=1}^n A \cap C_k
  \]
  Ну а тогда
  \[
    \mu (A) = \sum_k \mu D_k \leqslant \sum_k \mu C_k \leqslant \sum_k \mu B_k
  \]
\end{itlproof}


\begin{prop}[Непрерывность меры снизу]\label{prop:meas::meas::bcont}
  Пусть $A_1 \subset A_2 \subset \cdots$, $A_k \in \alg$,
  $\displaystyle A = \bigcup_{k=1}^\infty A_k \in \alg$.
  \note{Опять-таки никто не сказал, что $\alg$~--- $\sigma$-алгебра.} \par
  Тогда $\displaystyle \mu A = \lim_{n\to \infty} \mu A_n$
\end{prop}
\begin{prop}[Непрерывность меры сверху]\label{prop:meas::meas::ucont}
  Пусть $A_1 \supset A_2 \supset \cdots$, $A_k \in \alg$, 
  $\displaystyle A = \bigcap_{k=1}^\infty A_k \in \alg$, $\mu A_1 < +\infty $.
  \par
  Тогда $\displaystyle \mu A = \lim_{n\to \infty} \mu A_n$
\end{prop}

\plholdev{Тут будет картинка про метод исчерпывания Евдокса}


\paragraph{Объём в \texorpdfstring{$\R^n$}{}}
\label{par:meas::vol}

\end{document}
% vim: tw=100 cc=100
