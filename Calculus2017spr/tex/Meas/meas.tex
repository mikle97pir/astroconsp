%------------------------------------------------------------
% Description : 
% Author      : Iliya Tikhonenko <iliya.t@mail.ru>
% Created at  : Tue Feb 21 00:00:16 MSK 2017
%------------------------------------------------------------
\documentclass[12pt, timbord]{../../../notes}
\usepackage{silence}
\WarningFilter{latex}{Reference}
\graphicspath{{../../img/}}

\begin{document}
\paragraph{Системы множеств}
\label{par:meas::setsys}

\begin{defn}\label{defn:meas::setsys::sub}
  Пусть здесь (и дальше) $X$~--- проивольное множество. Тогда $\pset(X) \equiv 2^X$~--- множество
  всех подмножеств $X$.
\end{defn}
\begin{exmp*}
  $X = \{1 \intrng n\} \Rightarrow \# \pset (X) = 2^n$ (это количество элементов, если что)
\end{exmp*}

\begin{defn}[Алгебра]\label{defn:meas::setsys::alg}
  Пусть $\alg \subset \pset (X)$. Тогда $\alg$~--- алгебра множеств, если
  \begin{enumerate}
    \item $\varnothing \in \alg$
    \item $X \in \alg$
    \item $A,B\in \alg \Rightarrow A \cap B, A \cup B, A \setminus B \in \alg$
  \end{enumerate}
\end{defn}
\begin{rem*}
  Заметим, что в алгебре пересечение (или объединение) \emph{конечного} числа её элементов лежит в
  алгебре.  Это можно доказать простой индукцией. А вот для бесконечных объединений пользоваться
  индукцией уже нельзя, ведь $\infty \not\in \N$. 
\end{rem*}

\begin{defn}[$ \sigma $-алгбера]\label{defn:meas::setsys::sigalg}
  Пусть $\alg \in \pset(X)$. Тогда $\alg$~--- $ \sigma $-алгебра, если 
  \begin{enumerate}
    \item $\alg$~--- алгебра
    \item $A_1, \dotsc, A_n \in \alg \Rightarrow \bigcup_{k=1}^ \infty A_k \in\alg, 
      \bigcap_{k=1}^{ \infty } A_k \in \alg $
  \end{enumerate}     
\end{defn}

\begin{defn}\label{defn:meas::setsys::sigshell}
  Пусть $\mathcal E \subset \pset(X)$. Тогда 
  \[
  \sigma(\mathcal E) := \bigcap\,\{\alg \mid \alg\text{~--- $\sigma$-алгебра, $\alg \supset
  \mathcal E $}\}
  \]
  эта конструкция~--- сигма-алгебра, просто аксиомы проверить.
\end{defn}

\begin{defn}\label{defn:meas::setsys::borel}
  Пусть $\mathcal O$~--- все открытые множества в $\R^n$. Тогда $\mathpzc B_n = \sigma(\mathcal O)
  $~--- борелевская $\sigma$-алгебра в $\R^n$.
\end{defn}

\begin{defn}[Ячейка в $\R^n$ ]\label{defn:meas::setsys::cell}
  Обозначать её будем $\Delta^n$, по размерности соответствующего пространства.
  \[
    \Delta^1 = \begin{cases}
      [a;b) \\
      (-\infty;b) \\
      [a;+\infty) \\
      (-\infty;+\infty)
    \end{cases} \quad
    \forall\, n \;\: \Delta = \prod_{k=1}^n \Delta^1 _k 
  \]
  Ещё введём алгебру $\alg = \Cell_n = \{ A \mid A = \bigcup_{k=1}^p \Delta_k\}$
\end{defn} 

\begin{lem}\label{lem:meas::setsys::algsubset}
  Пусть $\mathcal E_1, \mathcal E_2 \subset \pset(X)$, $\sigma(\mathcal E_1) \supset \mathcal E_2$.
  Тогда $\sigma(\mathcal E_1) \supset \sigma(\mathcal E_2)$
\end{lem}
\begin{thrm}\label{thrm:meas::setsys::borelcell}
  $\mathpzc B_n = \sigma(\Cell_n)$.
\end{thrm}



\end{document}
% vim: tw=100 cc=100
