%------------------------------------------------------------
% Description : 
% Author      : Iliya Tikhonenko <iliya.t@mail.ru>
% Created at  : Tue Feb 21 00:00:16 MSK 2017
%------------------------------------------------------------
\documentclass[12pt, timbord]{../../../notes}
\usepackage{silence}
\WarningFilter{latex}{Reference}
\graphicspath{{../../img/}}

\begin{document}
\paragraph{Системы множеств}
\label{par:meas::setsys}

\begin{defn}\label{defn:meas::setsys::sub}
  Пусть здесь (и дальше) $X$~--- проивольное множество. Тогда $\pset P(X) \equiv 2^X$~--- множество всех 
  подмножеств $X$.
\end{defn}
\begin{exmp*}
  $X = \{1 \intrng n\} \Rightarrow \# \pset (X) = 2^n$ (это количество элементов, если что)
\end{exmp*}

\begin{defn}[Алгебра]\label{defn:meas::setsys::alg}
  Пусть $\alg \subset \pset (X)$. Тогда $\alg$~--- алгебра множеств, если
  \begin{enumerate}
    \item $\varnothing \in \alg$
    \item $X \in \alg$
    \item $A,B\in \alg \Rightarrow A \cap B, A \cup B, A \setminus B \in \alg$
  \end{enumerate}
\end{defn}
\begin{rem*}\underdev
  Заметим, что в алгебре пересечение (или объединение) \emph{конечного} числа её элементов лежит в алгебре. 
  Это можно доказать простой индукцией. А вот для бесконечных объединений пользоваться индукцией уже нельзя.
\end{rem*}
\end{document}

