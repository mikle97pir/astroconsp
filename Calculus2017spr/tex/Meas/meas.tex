%------------------------------------------------------------
% Description : 
% Author      : Iliya Tikhonenko <iliya.t@mail.ru>
% Created at  : Tue Feb 21 00:00:16 MSK 2017
%------------------------------------------------------------
\documentclass[12pt, timbord]{longnotes}
\usepackage{tmath}
\usepackage{cussymb}
\usepackage{silence}
\WarningFilter{latex}{Reference}
\graphicspath{{../../img/}}

\begin{document}
\paragraph{Системы множеств}
\label{par:meas::setsys}

\begin{defn}\label{defn:meas::setsys::sub}
  Пусть здесь (и дальше) $X$~--- произвольное множество. Тогда $\pset(X) \equiv 2^X$~--- множество
  всех подмножеств $X$.
\end{defn}
\begin{exmp*}
  $X = \{1 \intrng n\} \Rightarrow \# \pset (X) = 2^n$ (это количество элементов, если что)
\end{exmp*}

\begin{defn}[Алгебра]\label{defn:meas::setsys::alg}
  Пусть $\alg \subset \pset (X)$. Тогда $\alg$~--- алгебра множеств, если
  \begin{enumerate}
    \item $\varnothing \in \alg$
    \item $X \in \alg$
    \item $A,B\in \alg \Rightarrow A \cap B, A \cup B, A \setminus B \in \alg$
  \end{enumerate}
\end{defn}
\begin{rem*}
  Заметим, что в алгебре пересечение (или объединение) \emph{конечного} числа её элементов лежит в
  алгебре.  Это можно доказать простой индукцией. А вот для бесконечных объединений пользоваться
  индукцией уже нельзя, ведь $\infty \not\in \N$. 
\end{rem*}

\begin{defn}[$ \sigma $-алгбера]\label{defn:meas::setsys::sigalg}
  Пусть $\alg \in \pset(X)$. Тогда $\alg$~--- $ \sigma $-алгебра, если 
  \begin{enumerate}
    \item $\alg$~--- алгебра
    \item $A_1, \dotsc, A_n \in \alg \Rightarrow \bigcup_{k=1}^ \infty A_k \in\alg, 
      \bigcap_{k=1}^{ \infty } A_k \in \alg $
  \end{enumerate}     
\end{defn}

\begin{defn}\label{defn:meas::setsys::sigshell}
  Пусть $\mathcal E \subset \pset(X)$. Тогда 
  \[
  \sigma(\mathcal E) := \bigcap\,\{\alg \mid \alg\text{~--- $\sigma$-алгебра, $\alg \supset
  \mathcal E $}\}
  \]
  эта конструкция~--- сигма-алгебра, просто аксиомы проверить.
\end{defn}

\paragraph{Борелевская сигма-алгебра}
\begin{defn}\label{defn:meas::setsys::borel}
  Пусть $\open$~--- все открытые множества в $\R^n$. Тогда $\mathpzc B_n = \sigma(\open)
  $~--- борелевская $\sigma$-алгебра в $\R^n$.
\end{defn}

\begin{defn}[Ячейка в $\R^n$ ]\label{defn:meas::setsys::cell}
  Обозначать её будем $\Delta^n$, по размерности соответствующего пространства.
  \[
    \Delta^1 = \begin{cases}
      [a;b) \\
      (-\infty;b) \\
      [a;+\infty) \\
      (-\infty;+\infty)
    \end{cases} \quad
    \forall\, n \;\: \Delta = \prod_{k=1}^n \Delta^1 _k 
  \]
  Ещё введём алгебру $\alg = \Cell_n = \{ A \mid A = \bigcup_{k=1}^p \Delta_k\}$
\end{defn} 

\begin{lem}\label{lem:meas::setsys::algsubset}
  Пусть $\mathcal E_1, \mathcal E_2 \subset \pset(X)$, $\sigma(\mathcal E_1) \supset \mathcal E_2$.
  Тогда $\sigma(\mathcal E_1) \supset \sigma(\mathcal E_2)$
\end{lem}
\begin{thrm}\label{thrm:meas::setsys::borelcell}
  $\borel_n = \sigma(\Cell_n)$.
\end{thrm}

\begin{exmp}\label{exmp:meas::setsys::borel}
%   \everymath{\displaystyle}
  Все множества нижё~--- борелевские.
  \begin{enumerate}[$\langle$1$\rangle$]
    \item $\open $.
    \item $\closed =\{A \mid \ov-{A} \in \open \}$.
    \item $\Biggl(A 
      = \bigcap_{\substack{{k=1}\\[0.12em]\mathclap{ G_k \in \open}}}^\infty G_k \Biggr)\in G_\delta$.
    \item $\Biggl(B 
      = \bigcup_{\substack{{k=1}\\[0.12em]\mathclap{ F_k \in \closed}}}^\infty F_k \Biggr)\in
      F_\sigma$.
    \item $\Biggl(C 
      = \bigcup_{\substack{{k=1}\\[0.12em]\mathclap{ A_k \in G_{\delta}}}}^\infty A_k \Biggr)
      \in G_{\delta\sigma}$.
  \end{enumerate}
  У всех этих множеств со сложными индексами $\delta$~--- пересечение, $\sigma$~--- объединение,
  $G$~--- операция над открытыми в самом начале, $F$~--- над замкнутыми.
\end{exmp}

\paragraph{Мера}
\label{par:meas::meas}

\begin{defn}\label{defn:meas::meas}
  Пусть задано $X$, $\alg \subset \pset (X)$, $A_k \in \alg$. Тогда $\mu \colon \alg \to [0;
  +\infty]$~--- мера, если 
  \begin{enumerate}
    \item $\mu(\varnothing) = 0$
    \item $\displaystyle\mu\underbrace{\left(\bigsqcup_{k=1}^\infty A_k\right)}_{\in \alg} 
      = \sum_{k=1}^\infty \mu(A_k)$. Здесь никто не обещает, что будет именно $\sigma$-алгебра.
  \end{enumerate}
  Множества $A \in \alg$ в таком случае называются $\mu$-измеримыми.
\end{defn}

\begin{exmp}\label{exmp:meas::meas::delta}
  $a \in X$, $\displaystyle\mu(A) = \begin{cases}
    1, & a\in A \\
    0, & a \not\in A
  \end{cases}$ ~--- $\delta$-мера Дирака.
\end{exmp}
\begin{exmp}\label{exmp:meas::meas::mol}
  $a_k \in x$, $m_k \geqslant 0$, $\displaystyle\mu(a) := \sum_{\mathclap {k\colon a_k \in a}} m_k$~---
  <<молекулярная>> мера.
\end{exmp}

\begin{exmp}\label{exmp:meas::meas::cnt}
  $\mu(A) = \# A$~--- считающая мера. \note{она считает, не считывает $\ddot\smile$}
\end{exmp}

\paragraph{Свойства меmы}

Здесь всюду будем рассматривать тройку $(X, \alg \subset \pset (X), \mu)$

\begin{prop}[Монотонность меры]\label{prop:meas::meas::monot}
  Пусть $A,B \in \alg$, $A \subset B$. \par Тогда $\mu(A) \leqslant \mu(B)$.
\end{prop}
\begin{prop}\label{prop:meas::meas::diff}
  Пусть $A,B \in \alg$, $A \subset B$, $\mu(B) < +\infty$.  \par 
  Тогда $ \mu(B\setminus A) = \mu(B) - \mu(A)$.
\end{prop}
\begin{prop}[Усиленная монотонность]\label{prop:meas::meas::enfmont}
  Пусть $A_{1 \intrng n}, B \in \alg$, $A_{1 \intrng n} \subset B$ и дизъюнктны. \par
  Тогда $\displaystyle\sum_{k=1}^n \mu(A_k) \leqslant \mu B$
\end{prop}

\begin{prop}[Полуаддитивность меры]\label{prop:meas::meas::semiadd}
  Пусть $B_{1 \intrng n}, A \in \alg$, $A \subset \displaystyle \bigcup_{k=1}^n B_k$. \par
  Тогда $\displaystyle\mu A \leqslant \sum_{k=1}^n \mu(B_k)$.
\end{prop}
\begin{lproof}
  Сделать $B_k$ дизъюнктными: $C_k = B_k \setminus \bigcup_{j <k} B_k$. Затем представить $A$ как
  дизъюнктное объединение $D_k \colon D_k = C_k \cap A$. Так можно сделать, потому что
  \[
    A = A \cap \bigcup_{k=1}^n B_k = A \cap \bigcup_{k=1}^n C_k = \bigcup_{k=1}^n A \cap C_k
  \]
  Ну а тогда
  \[
    \mu (A) = \sum_k \mu D_k \leqslant \sum_k \mu C_k \leqslant \sum_k \mu B_k
  \]
\end{lproof}


\begin{prop}[Непрерывность меры снизу]\label{prop:meas::meas::bcont}
  Пусть $A_1 \subset A_2 \subset \cdots$, $A_k \in \alg$,
  $\displaystyle A = \bigcup_{k=1}^\infty A_k \in \alg$.
  \note{Опять-таки никто не сказал, что $\alg$~--- $\sigma$-алгебра.} \par
  Тогда $\displaystyle \mu A = \lim_{n\to \infty} \mu A_n$
\end{prop}
\begin{prop}[Непрерывность меры сверху]\label{prop:meas::meas::ucont}
  Пусть $A_1 \supset A_2 \supset \cdots$, $A_k \in \alg$, 
  $\displaystyle A = \bigcap_{k=1}^\infty A_k \in \alg$, $\mu A_1 < +\infty $.
  \par
  Тогда $\displaystyle \mu A = \lim_{n\to \infty} \mu A_n$
\end{prop}

\plholdev{Тут будет картинка про метод исчерпывания Евдокса}

\begin{defn}\label{defn:meas::ledeg::compl}
  Пусть задана тройка $(X, \alg_\sigma, \mu)$. Тогда $\mu$~--- полная, если 
  \[
    \forall\, \in \alg \colon \mu A = 0 \; \forall\, B \subset A, B \in \alg \;::\; \mu B = 0
  \]
\end{defn}

\begin{defn}\label{defn:meas::ledeg::sfin}
  Мера $\mu$  на $\alg$ называется $\sigma$-конечной, если 
  \[
    \exists\, X_n \in \alg, \mu X_n < + \infty \that \bigcup_{n=1}^\infty X_n = X
  \]
\end{defn}

\begin{defn}\label{defn:meas::ledeg::mcont}
  Пусть $\alg_1, \alg_2$~--- сигма-алгебры подмножеств $X$, $\alg_1 \subset \alg_2$,
  $\mu_1 \colon A_1 \to [0;+\infty] $, $\mu_2 \colon A_2 \to [0;+\infty]$. 
  Тогда $\mu_2$ называется продолжением $\mu_1$.
\end{defn}

\begin{thrm}[Лебега-Каратеодора]\label{thrm:meas::ledeg::dist}
  Пусть $\mu$~--- сигма-конечная мера на $\alg$. Тогда
  \begin{enumerate}
    \item Существуют её полные сигма-конечные продожения
    \item Среди них есть наименьшее: $\ov-\mu$. 
      Её ещё называют стандартным продолжением.
  \end{enumerate}
\end{thrm}
\plholdev{идея доказательства}
Пока надо запомнить, что стандратное продолжение~--- сужение внешней <<меры>> 
на хорошо разбивающие множества.


\paragraph{Объём в \texorpdfstring{\(\R^n\)}{R\^{}n}. Мера Лебега }
\label{par:meas::lebeg}

\begin{defn}\label{defn:meas::lebeg::cell}
  Пусть $\Delta = \Delta_1 \times \dotsm \times \Delta_n$, $\Delta_k = [a_k, b_k)$.
  Тогда
  \[
    \begin{split}
      v_1 \Delta_k \equiv | \Delta_k| &:= \begin{cases}
        b_k - a_k, & a_k \in \R \land b_k \in \R \\
        \infty, & \text{иначе}
      \end{cases} \\ 
      v \Delta \overset{(\in R^n)}\equiv v_n \Delta &:= |\Delta_1| \dotsm |\Delta_n|
    \end{split}
  \]
  Для всего, что $\in \Cell_n$, представим его в виде дизъюнктного объединения $\Delta_j$.
  Тогда $vA := \sum_{j=1}^q v \Delta_j $.
\end{defn}

\begin{rem*}
  Здесь радикально всё равно, входят ли концы~--- у них мера ноль. 
\end{rem*}

\begin{thrm}\label{thrm:meas::ledeg::finadd}
  $v$~--- конечно-аддитивен, то есть \[
    \forall\, A, A_{1 \intrng p} \in \Cell\, , A = \bigsqcup_{k=1}^p A_k
    \:\; \Rightarrow  vA = \sum_{k=1}^p v A_k
  \]
\end{thrm}

\begin{thrm}\label{thrm:meas::ledeg::infadd}
  $v$~--- счётно-аддитивен, то есть \[
    \forall\, A, A_{1 \intrng } \in \Cell\, , A = \bigsqcup_{k=1}^\infty  A_k
    \:\; \Rightarrow  vA = \sum_{k=1}^ \infty  v A_k
  \]
\end{thrm}
\begin{tproof}
  Здесь в конспекте лишь частный случай про ячейки.
\end{tproof}



\begin{defn}[Мера Лебега]\label{defn:meas::ledeg::lebeg}
  $X=\R^n$, $\alg= \Cell_n$. Тогда $\lambda_n = \ov-{v_n}$, $\mathpzc M = \ov-{\alg}$~--- мера
  Лебега и алгебра множеств, измеримых по Лебегу, соответственно.
\end{defn}

\subparagraph{Свойства меры Лебега}

\begin{enumerate}[(1) $\triangleright$]
  \item $\lambda \{x\} = 0$
  \item $\lambda \bigl(\{x_k\}_{k}\bigr) = 0$
  \item $\borel \subset \lebesgue$
  \item $L \subset \R^m, m<n \Rightarrow \lambda_n L  = 0$
\end{enumerate}

А это уже целая теормема.
\begin{thrm}[Регулярность меры Лебега]\label{thrm:meas::ledeg::reg}
  Пусть $A \in \lebesgue$, $\varepsilon > 0$. Тогда 
  \[
    \exists\, G \in \open, F \in \closed \that F \subset A \subset G \land 
    \begin{cases}
      \lambda(G \setminus A) < \varepsilon \\
      \lambda(A \setminus F) < \varepsilon 
    \end{cases}
  \]
\end{thrm}
\begin{tproof}
  куча скучных оценок квадратиками.
\end{tproof}


\paragraph{Измеримые функции}
\label{par:meas::mfun}

\begin{defn}\label{defn:meas::mfun}
  Пусть задана тройка $(X, \alg_\sigma, \mu)$. Пусть ещё $f\colon X \to \R$. Тогда 
  $f$ называется измеримой относительно $\alg$, если 
  \[
    \forall\, \Delta \subset \R  \holds f^{-1} (\Delta) \in \alg
  \]
\end{defn}

\begin{thrm}\label{thrm:meas::mfun::diftyps}
  Пусть $f$ измеримо относительно $\alg$. Тогда измеримы и следующие (Лебеговы) множества
  \begin{description}
    \item[1 типа] $\{x \in X \mid f(x) < a\} \equiv X [f < a]$
    \item[2 типа] $\{x \in X \mid f(x) \leqslant a\} \equiv X [f \leqslant a]$
    \item[3 типа] $\{x \in X \mid f(x) > a\} \equiv X [f > a]$
    \item[4 типа] $\{x \in X \mid f(x) \geqslant a\} \equiv X [f \geqslant a]$
  \end{description}
  При этом верно и обратное: если измеримы множества какого-то отдного типа, то
  $f$ измерима.
\end{thrm}

\begin{thrm}\label{thrm:meas::mfun::sp}
  Пусть $f_1, \dotsc, f_n$ измеримы относительно $\alg$ и $g\colon \R^n \to R$  
  непрерывна. Тогда измерима и $\varphi(x) = g(f_1(x), \dotsc, f_n(x))$.
\end{thrm}
\begin{rem*}
  В частности, $f_1 + f_2$ измерима.
\end{rem*}

\begin{thrm}\label{thrm:meas::mfun::lims}
  Пусть $f_1,f_2, \dotsc $ измеримы относительно $\alg$ .
  Тогда измеримы $\sup f_n, \inf f_n, \liminf f_n, \limsup f_n, \lim f_n $.
  Последний, правда, может не существовать.
\end{thrm}
\begin{tproof}
  Следует из непрерывности меры.
\end{tproof}

\begin{defn}\label{defn:meas::mfun::simp}
  Пусть $f \colon X \to \R$~--- измерима. Тогда она называется простой, если
  принимает конечное множество значений.
\end{defn}

\begin{defn}[Индикатор множества]\label{defn:meas::mfun::ind}
  \[
    E \subset X , \ind_E := \begin{cases}
      1, & x\in E \\
      0, & x\not\in E
    \end{cases}
  \]
  Он, как видно совсем простая функция.
\end{defn}

\begin{stat}\label{stat:meas::mfun::simpind}
  $f$~--- простая $ \Rightarrow f= \dsum_{k=1}^p c_k \ind_{E_k}$ 
\end{stat}

\begin{thrm}\label{thrm:meas::mfun::simpseq}
  Пусть $f\colon X \to \R$, измерима, $f \geqslant 0$. Тогда 
  \[
    \exists\, (\varphi_n)\colon 0 \leqslant \varphi_1 \leqslant \varphi_2 \leqslant \cdots
    \that \varphi_n \nearrow f \text{ (поточечно)}
  \]
\end{thrm}

\paragraph{Интеграл по мере}
\label{par:meas::int}

\begin{defn}\label{defn:meas::int}
  Пусть задана тройка $(X, \alg_\sigma, \mu)$, $f$~--- измерима.
\begin{enumerate}[{[1]}]
    \item $f$~--- простая. 
      \[
        \int _X f \, \del \mu := \sum_{k=1}^p c_k \mu E_k 
      \]
    \item $f \geqslant 0$.
      \[
        \int_X f \, \del \mu := \sup \Biggl\{ \int_X g \, \del \mu \:\Biggl|\: g\text{-простая},
          0 \leqslant g \leqslant f \Biggr\}
      \] 
    \item $f$ общего вида.
      \[
         \begin{split}
           f_+ &= \max \{f(x),0\} \\
           f_- &= \max \{-f(x),0\} \\
           \int _X f \, \del \mu = \int _X f_+ \, \del \mu - \int _X f_- \, \del \mu
         \end{split}
      \]
      Здесь нужно, чтобы хотя бы один из интегралов в разности существовал.
  \end{enumerate}
\end{defn}
\begin{rem*}
  $\displaystyle
    \int _A f \, \del \mu := \sum_{k=1}^p c_k \mu (E_k \cap A)
  $
\end{rem*}


\begin{prop}\label{prop:meas::mfun::intind}
  $\displaystyle \int_A f \, \del \mu = \int _X f \cdot \ind _A \, \del \mu$.
\end{prop}

\subparagraph{Свойства интеграла от неотрицательных функций}

Здесь всюду функции неотрицательны и измеримы, что не лишено отсутствия внезапности.
\begin{enumerate}[\fbox{$\text{А}_{\arabic*}$}]
  \item $0 \leqslant f \leqslant g$. Тогда
    $\displaystyle \int_X f\, \del \mu  \leqslant \int_X g\, \del \mu$.
  \item $A \subset B \subset X$, $A , B \in \alg$, $f \geqslant 0$, измерима.
    Тогда $\displaystyle \int_A f \, \del \mu \leqslant \int_B f\, \del \mu$
  \item см теорему \ref{thrm:meas::levi}.
  \item $\displaystyle \int_X (f+g)\, \del \mu = \int_X f\,\del mu + \int_X g\,\del mu $
  \item $\displaystyle \int_X (\lambda g)\, \del \mu =  \lambda \int_X f\,\del mu $
\end{enumerate}

\paragraph{Теорема Беппо \texorpdfstring{Л\'eви}{Леви}}
\label{par:meas::levi}

\begin{thrm}\label{thrm:meas::levi}
  Пусть $(f_n)$~--- измеримы на $X$, $0 \leqslant f_1 \leqslant \cdots $, 
  $f = \lim_n f_n$. Тогда 
  \[
    \int_X f \, \del \mu  = \lim_{n\to \infty} \int_X f_n \, \del \mu 
  \]
\end{thrm}

\paragraph{Свойства интеграла от суммируемых функций}
\label{par:meas::summprop}

\begin{defn}\label{defn:meas::summprop::summ}
  $f$~--- суммируемая (на $X,\mu$), если $\displaystyle \int _X f \, \del \mu < \infty$.
  Весь класс суммируемых (на $X,\mu$) функций обозначается через $\summb (X,\mu)$.
\end{defn}

Здесь всюду функции $\in \summb$
\begin{enumerate}[\fbox{$\text{B}_{\arabic*}$}]
  \item $\displaystyle
    f \leqslant g \Rightarrow \int_X f \, \del \mu \leqslant \int_X g \, \del \mu$.
  
  \item $\displaystyle
    \int_X (f \pm g)\, \del \mu =  \int_X f \, \del \mu \pm \int_X g \, \del \mu$.

  \item $\displaystyle
    \int_X \lambda f \, \del \mu =  \lambda \int_X f \, \del \mu$.
  
  \item $\displaystyle
    |f| \leqslant g  \Rightarrow \left| \int_X f \, \del\mu \right|\leqslant \int_X g \, \del \mu$.

  \item $\displaystyle
    \left| \int_X f \, \del \mu \right| \leqslant \int_X |f| \, \del \mu$.
  
  \item $\displaystyle
    f \in \summb \Leftrightarrow |f| \in \summb$

  \item $\displaystyle
    |f| \leqslant M \leqslant +\infty 
    \Rightarrow \left| \int_X f \, \del\mu \right| \leqslant M \mu X$
\end{enumerate}

\paragraph{Счётная аддитивность интеграла}
\begin{thrm}\label{thrm:meas::infadd}
  Пусть задана тройка $(X,\alg,\mu)$, $f$~--- измерима и $f \geqslant 0 \lor f \in \summb$. 
  Пусть к тому же 
  \[
    A, A_{1\intrng} \subset X, A = \bigcup_{n=1}^\infty A_n
  \]
  Тогда 
  \[
    \int _A f \, \del \mu = \sum_{n=1}^\infty \int _{A_n} f\, \del \mu 
  \]
\end{thrm}

\paragraph{Абсолютная непрерывность интеграла}
\begin{thrm}\label{thrm:meas::abscont}
  Пусть $f\in \summb(X,\alg,\mu)$. Тогда
  \[
    \forall\, \varepsilon > 0 \; \exists\, \delta > 0 \that \forall\, A \in \alg , A \subset X
    \colon \mu A < \delta \holds \left| \int_A f \, \del \mu \right| < \varepsilon 
  \]
\end{thrm}


\end{document}
% vim: tw=100 cc=100
