%------------------------------------------------------------
% Description : Differential Geometry
% Author      : Iliya Tikhonenko <iliya.t@mail.ru>
% Created at  : Thu Jun  1 18:33:27 MSK 2017
%------------------------------------------------------------
\documentclass[12pt,draft,timbord]{longnotes}
\usepackage{tmath} 
\usepackage{cussymb} 
\usepackage{silence}
\WarningFilter{latex}{Reference}
\graphicspath{{../../img/}}

\begin{document}
\paragraph{Регулярная кривая и её естественная параметризация}
\label{par:dg::curve}

\begin{defn}[Кривая, как отображение]\label{defn:dg::curve::map}
  Пусть задано гладкое отображение $t\in [a;b] \mapsto r(t) \in \R^3$, регулярное, то есть
  $rk r'(t) \equiv 1$. $t$~--- параметр, само отображение ещё можно
  называть параметризацией.

\end{defn}

\begin{defn}[Кривая, как класс отображений]\label{defn:dg::curve::class}
  Введём отношение эквивалентности отображений:
  \[
    r(t) \sim \rho(\tau) \Leftrightarrow 
    \exists\, \delta \colon [a;b] \leftrightarrow [\alpha, \beta ] \that \rho (\delta (t)) = r(t)
  \]
\end{defn}

А теперь будем их путать. \flame

\begin{defn}[Естественная параметризация]\label{defn:dg::curve::nat}
  Пусть $[a;b] = [t_0, t_1]$.
  Рассмотрим $\ov~s(t) = \dint_{t_0}^t | r'(t) | \, \del \tau $. Она, как видно, является 
  пройденным путём и неубывает  $ \Rightarrow $ годится на роль $\delta$.

  Так что можно рассматривать $s$ как параметр,  это собственно и есть
  естественная (натуральная) параметризация.
\end{defn}


\begin{prop}\label{prop:dg::curve::repar}
  Пусть есть две разных параметризации: $r(t)$ и $r(s)$ одной кривой. Тогда 
  \[
    \dot r \equiv \pder{r(s)}{s} = \Bigl(r'(t) \cdot (s'(t))^{-1}\Bigr)(t) = \frac{r'}{|r'|} 
  \]
  Как видно, натуральная почему-то обозначается точкой.
\end{prop}

\paragraph{Кривизна кривой}
\label{par:dg::curvature}

\begin{defn}[Касатальный вектор]\label{defn:dg::curvature::tang}
  $\tau := \dot r(s)$.
\end{defn}

\begin{defn}[Кривизна]\label{defn:dg::curvature}
  $k_1 = |\dot \tau |$
\end{defn}
\begin{defn}[Радиус кривизны]\label{defn:dg::curvature::rad}
  $R = k_1 ^{-1}$
\end{defn}

\begin{prop}\label{prop:dg::curvature::orth}
  $\tau \perp \dot \tau$
\end{prop}

\begin{thrm}\label{thrm:dg::curvature::repar}
  $\displaystyle k_1 = \frac{| r' \times r''|}{|r'|^3} $
\end{thrm}

\paragraph{Кручение и нормаль}
\label{par:dg::norm}

\begin{defn}[Нормаль]\label{defn:dg::norm::norm}
  Пусть $k_1 \neq 0$. Тогда $\nu  := \frac{\dot \tau } {k_1}$.
\end{defn}
\begin{defn}[Бинормаль]\label{defn:dg::norm::binorm}
  $\beta = \tau \times \nu $.
\end{defn}

\begin{rem*}
  $(\tau,\nu,\beta)$~--- хороший кандидат для репера в какой-нибудь точке $P$.
\end{rem*}
\begin{defn}[Соприкасающаяся плоскость]\label{defn:dg::norm::tangpl}
  Пусть $k_1 > 0$, $P = r(s_0)$, $T$~--- плоскость, $T \ni P$, $N \perp T$~--- нормаль к ней.
  Допустим, $r(s+\Delta s) \cdot N  = h$, $h = o(\Delta s^2)$. Тогда $T$~--- соприкасающаяся
  плоскость.
\end{defn}
\begin{prop}\label{prop:dg::norm::tangpl}
  $ \tau , \nu \perp N  $;
  $(r-r_0, \dot r_0 , \ddot r_0)=0$~--- её уравнение
\end{prop}

\begin{defn}[Абсолютное кручение]\label{defn:dg::norm::abscrvn}
  $|k_2| := | \dot\beta|$
\end{defn}
\begin{thrm}\label{thrm:dg::norm::abscrvn}
  $|k_2| = \left|\dfrac{(\dot r, \ddot r, \dddot r\,)}{k_1^2}\right|$
\end{thrm}

\begin{defn}[Кручение]\label{defn:dg::norm::crvn}
  $k_2 := \dfrac{-(\dot r, \ddot r, \dddot r\,)}{k_1^2}$
\end{defn}

\paragraph{Формулы Френе}
\label{par:dg::frene}

\begin{thrm}\label{thrm:dg::frene}
  \begin{equation}
    \label{eq:dg::frene}
    \begin{pmatrix}
      \dot\tau \\\dot\nu \\ \dot\beta
    \end{pmatrix}
    = 
    \begin{pmatrix}
      0 & k_1 & 0 \\
      -k_1 & 0 & -k_2 \\
      0 & k_2 & 0
    \end{pmatrix}\cdot
    \begin{pmatrix}
      \tau \\ \nu \\ \beta
    \end{pmatrix}
  \end{equation}
\end{thrm}

\begin{thrm}\label{thrm:dg::frene::reconst}
  Пусть $r(s)$~--- гладкая кривая с заданными $k_1$ и $k_2$, $k_1>0$. 
  Тогда система \eqref{eq:dg::frene} определит её с точностью до движения.
\end{thrm}

\paragraph{Регулярная поверхность. Касательная плоскость. Первая квадратичная форма}
\label{par:dg::tangplane}

\begin{defn}[Поверзность (двумерная)]\label{defn:dg::tangplane::manifold}
  Пусть задано гладкое отображение \[
    \varphi \colon (u,v) \in D \subset \R^2 \mapsto r=(x,y,z) \in \R^3
  \]
  Добавим условие регулярности $\rk \varphi' \equiv 2$ и условимся путать отображение и класс 
  оных.
\end{defn}

\begin{defn}\label{defn:dg::tangplane::tanv}
  \[
    \begin{split}
      r_u &:= (x_u', y_u', z_u') \\
      r_v &:= (x_v', y_v', z_v') \\
      n &:= \frac{r_u \times r_v}{|r_u \times r_v|} = \frac{N}{|N|} 
    \end{split}
  \]
  Отметим, что условие регулярности не дает векторному произведению обращаться в 0.
  
  Касательную плоскость можно было бы здесь определить через нормаль, но лучше пока ещё подумать.
  Может, абстракций добавить.
\end{defn}



\begin{defn}[Первая квадратичная форма]\label{defn:dg::tangplane::I}
  \[
    \begin{split}
      I :&= |\del r|^2  = r_u^2\, \del u^2 + 2r_u r_v\, \del u \,\del v + r_v^2\, \del v^2 \\
        &= E \, \del u^2 + 2F\, \del u \,\del v + G\, \del v^2 
    \end{split}
  \]
\end{defn}

\paragraph{Вычисление длин и площадей на поверхности}
\label{par:dg::area}

\begin{thrm}\label{thrm:dg::area::len}
  Пусть $M$~--- поверхность, $\gamma \colon t \to r \in M$.
  Тогда \[
    \ell(\gamma)= \dint_{t_0}^{t_1} \sqrt I.\; (\del s = I)
  \]
\end{thrm}

\begin{thrm}\label{thrm:dg::area::area}
  Пусть $M$~--- поверхность, $u,v\in D$, $I = E \, \del u^2 + 2F\, \del u \,\del v + G\, \del v^2$.
  Тогда
  \[
    S(M) = \iint_D \sqrt{EG - F^2}\, \del u\,\del v
  \]
\end{thrm}


\quest\plholdev{вкусный абстрактный кусок про меру на многообразии}

\begin{defn}\label{thrm:dg::area::manifold}
  Пусть $M$~--- подмногообразие $\R^n$.  Тогда 
  \[
      \lambda_k := \int _D \sqrt {\det g(t)} \, \del t, \quad
      g(t)_{ij} = \left(\pder{x}{t_i} \cdot \pder{x}{t_j}\right)\, (t)
  \]
\end{defn}

\begin{rem}
  Как видно, в $\R^2$, $g$ очень похож на матрицу 1ой квадратичной формы
\end{rem}

\begin{defn}\label{defn:dg::area::isom}
  Пусть $M_1$, $M_2$~--- пара поверхностей. Допустим, $\exists\, F \colon M_1 \to M_2$, 
  сохраняющее длины кривых. Тогда они называются  изометричными.
\end{defn}

\begin{thrm}\label{thrm:dg::area::Iisom}
  Пусть $M_1$, $M_2$~--- пара поверхностей. Допустим, что существуют их параметризации, 
  при которых $I_1= I_2$. Тогда они изометричны.
\end{thrm}

\paragraph{Вторая квадратичная форма}
\label{par:dg::II}

\begin{defn}\label{defn:dg::II}
  Снова рассмотрим поверхность с  какой-то параметризацией. Тогда 
  ${\rm II} := - \del r \, \del n = L \, \del u^2 + 2N\, \del u \,\del v + M\, \del v^2$.
\end{defn}

\begin{prop}\label{prop:dg::II}
  ${\rm II} = n \cdot \del^2 r$
\end{prop}

\begin{prop}[Типы точек на поверхности]\label{prop:dg::II::ptypes}
  Здесь названия связаны с типом соприкасающегося параболоида. Его можно добыть, рассматривая
  $\Delta r \cdot n$.
  \begin{description}
    \item[${\rm II} > 0$:] Эллиптический
    \item[${\rm II} < 0$:] Он же
    \item[${\rm II} \lessgtr 0$:] Гиперболический
    \item[${\rm II} \geqslant 0 \lor {\rm II} \leqslant 0$:] Параболический (вроде цилиндра)
    \item[${\rm II} = 0$:] Точка уплощения
  \end{description}
\end{prop}


\paragraph{Нормальная кривизна в данном направлении. Главные кривизны}
\label{par:meas::curfctr}

\begin{defn}\label{defn:meas::curfctr::normsec}
  Нормальное сечение поверхности~--- сечение плоскостью, 
  содержащей нормаль к поверхности (в точке).
\end{defn}

\begin{lem}\label{lem:meas::curfctr::normsec}
  Нормальное сечение~--- кривая.
\end{lem}

Сначала рассмотрим несколько более общий случай

\begin{thrm}[Менье]\label{thrm:meas::curfctr::menie}
  Пусть $\gamma$~--- кривая $ \subset M$, $ \gamma \ni P$.
  Тогда $k_0 = k_1 \cos (\underbrace{\nu \,\text{\^;}\, n}_\theta) = \frac{\rm II}{\rm I} $.
\end{thrm}

\begin{rem}\label{rem:meas::curfctr::menie::ref}
  Ещё можно сформулировать так: для всякой кривой на повехности, проходящей через точку 
  в заданном направлении $k_0 = \rm const$
\end{rem}

а теперь сузим обратно.
\begin{defn}\label{defn:meas::curfctr::normcrvf}
  Нормальная кривизна~--- кривизна нормального сечения.
\end{defn}

Для нормального сечения $\cos\theta = \pm 1$.

Если немного переписать и ввести параметр $t= \del v /\del u $
\[
  k_1(t) = |k_0(t)| = \left|\frac{L + 2Nt + Mt^2}{E + 2Ft +Gt^2} \right|
\]
Этот параметр $t$ и задаёт <<направление>> нормального сечения. 
Так что $k_0(t)$ и есть та самая <<кривизна в данном направлении>>.

Теперь найдем экстремумы $\frac{\rm II}{\rm I} (t)$. 
\begin{thrm}\label{thrm:meas::curfctr::minmax}
  $\exists\, k_{\min}, k_{\max}$, $k_{\min} \cdot k_{\max} = \frac{LM - N^2}{EG - F^2}$.
\end{thrm}

\begin{defn}\label{defn:meas::curfctr::main}
  $k_{\min}, k_{\max}$~--- главные кривызны.
\end{defn}


\paragraph{Гауссова кривизна поверхности. Теорема Гаусса}
\label{par:meas::gauss}

\begin{defn}[Гауссова кривизна]\label{defn:meas::gauss::crvf}
  $K = k_{\min} \cdot k_{\max}$.
\end{defn}

\begin{defn}[Гауссово отображение]\label{defn:meas::gauss::map}
  Пусть $M$~--- поверхность, $n$~--- нормаль к ней в точке $P$, $S$~--- единичная сфера.
  Тогда $G: n \mapsto C \in S$ ($C$~--- точка на сфере).
\end{defn}

\begin{thrm}\label{thrm:meas::gauss::lim}
  Пусть $U$~--- окрестность $P \subset M$, $M$~--- поверхность, $\mathcal N$~--- поле нормалей
  на $U$. Допустим, что $V = G(\mathcal N)$, она вроде как окрестность $G(n_P)$. 

  Тогда \[
    |K| = \lim_{U \to P} \frac{\iint_V |n_u \times n_v|}{\iint_U|r_u \times r_v|} 
  \]
\end{thrm}

\paragraph{Геодезическая кривизна. Теорема Гаусса-Бонне.}
\label{par:dg::bonnet}

\begin{defn}[Геодезическая кривизна]\label{defn:dg::bonnet::geodcrvn}
  Пусть $M$~--- поверхность, $T$~--- касательная к ней в точке $P$. Допустим,
  $\gamma \subset M$ проходит через $P$. Рассмотрим проекцию $\gamma$ на $T$.
  Тогда $\varkappa := k_\gamma $~--- и есть геодезическая кривизна.
\end{defn}

\begin{defn}\label{defn:dg::bonnet::geodcurv}
  Если для кривой $\varkappa(s) \equiv 0$, то она называется геодезической.
\end{defn}

\begin{thrm}[Гаусса-Бонне]\label{thrm:dg::bonnet}
  Пусть $M$~--- гладкая поверхность, $P_1, \dotsc, P_n$~--- вершины криволинейного многоугольника,
  $P_i, P_{i+1} = \gamma $, $\alpha_i$~--- углы при вершинах. Тогда \[
    \sum_i \alpha_i + \sum_i \int _{\gamma_i} \varkappa \, \del s  = 2 \pi - \iint_P K \, \del s
  \]
\end{thrm}





\paragraph{Ориентация кривой и поверхности}
\label{par:dg::orient}

Здесь сначала введём всякие конкретные определения, потом абстрактное, потом конкретные примеры.

\begin{defn}[Векторное поле]\label{defn:dg::orient::vecf}
  Пусть $G \subset \R^n$, $V$~--- векторное пространство. Тогда $f \colon G \to V$ и есть векторное 
  поле.
\end{defn}
\begin{exmp}\label{exmp:dg::orient::vecf}
  $V = \R^k$.
\end{exmp}

\begin{rem}\label{rem:dg::orient::vecf} 
  Если захотеть гладкого векторного поля,
  то нужно уметь вводить на $V$ норму\note{$o(\|h\|)$}.  Но как правило имеют
  дело c $V = \R^n$ где это всё уже есть.
\end{rem}

\begin{defn}\label{defn:dg::orient::curve}
  Ориентация на кривой~--- непрерывное поле $\tau(x(t))$. Они все единичные, так что варианта
  выбрать $\tau(x)$ всего 2. Соответсвенно, и ориентаций две.
\end{defn}
\begin{rem}\label{rem:dg::orient::curvecor}
  Регулярность избавит от изломов, а все пересечения разделяются по $t$.
\end{rem}

\begin{rem}[\flame]\label{rem:dg::orient::curverman}
  В нашем понимании кривая~--- не многообразие. У многообразия были бы проблемы с окрестностью
  пересечения. Это можно показать рассмотрев 4 точки в окрестности пересечения и устремив ту, что
  с самым далёким прообразом к пересечению.
  \note{я же тот ещё велосипедостроитель?} 
\end{rem}

\begin{defn}\label{defn:dg::orient::curvepar}
  Ориентация на кривой~--- класс эквивалентности параметризаций по отношению
  $r(t) \sim \rho(\tau )  \Leftrightarrow  \delta' > 0$(всегда). 
\end{defn}

\begin{prop}\label{prop:dg::orient::eqcurve}
  Определения \ref{defn:dg::orient::curve} и \ref{defn:dg::orient::curvepar} эквиваленты.
\end{prop}
\begin{lproof}
  банан.
\end{lproof}

\begin{defn}\label{defn:dg::orient::curorabl}
  Если на кривой вводится ориентация, то она ориентируемая.
\end{defn}

Тут нужно отметить, что подход выше совсем ломается, когда дело заходит о поверхностях.
Обобщив рассуждения выше на поверхности, мы придём к тому, что лента Мёбиуса окажется
ориентируемой. Ну, в самом деле, если привязать нормали к параметрам, а не к координатам
пространства содержащего поверхность, то окажется, что нормаль всегда <<вращается>> 
непрерывно. 


Так что надо сейчас заняться ориентацией многообразий.

\def\H{\mathbbm H}
\begin{defn}\label{defn:dg::AAAAA::manifold}
  Пусть $M \subset \R^n$. Выберем на нем произвольную точку $x$ и рассмотрим 
  $V(x) = V_{\R^n}(x) \cap M$. Допустим, 
  \[
    \exists\, f \in C^1 \that V(x) \leftrightarrow^f \R^k\;(\text{или }\H^k).
  \]
  
  Тогда $M$~--- гладкое подмногообразие $\R^n$, а $f$~--- локальная карта многообразия.
  Набор всех карт называется атласом. $t \in \R^k$~--- локальные координаты в $V$.
  
  Атлас : $A(M) = \{(\varphi_k, V_k)_k\}$~--- все окрестности и карты на них. 

  Если 
  \[
    \exists\, x\in M \that V \leftrightarrow \H (\H = \{x\in \R^k \mid x^1 \leqslant 1\},
    \] 
  тогда это многообразие с краем.  
  Край обычно обозначается как $\partial M$.

  По идее, в атлас ещё надо включать информацию, карта на $\R^k$ или на $\H^k$. Так что
  \[
    A(M) = \{ (\H^k, \varphi_i, V_i)_i\} \cup \{(\R^k, \varphi_j, V_j)_j\}
  \]
\end{defn}


Теперь про ориентацию.

\begin{defn}\label{defn:dg::orient::map}
  Две карты называются согласованными, если отображение 
  $t_1 \mapsto  x\in V_1 \cap V_2 \mapsto t_2$ имеет положительный якобиан.
\end{defn}
\begin{defn}\label{defn:dg::orient::mapsmatch}
  Если все карты попарно согласованы, то атлас называется ориентирующим.
  Многообразие тогда называется ориентированным.
\end{defn}

Представить все это проще всего на примере города, покрытого точками сотовой связи.
Пересечение границы области покрытия одной вышки не приводит к потере связи.


Нетрудно понять, что ориентирующих атласов много. Город может покрывать
хорошее количество сотовых операторов.

\begin{defn}\label{defn:dg::orient::atleq}
  Атласы эквивалентны, если составленный из них атлас~--- тоже ориентирующий.
\end{defn}

\begin{prop}\label{prop:dg::orient::conn}
  Если многообразие связно, то они линейно связно.
\end{prop}
\begin{prop}\label{prop:dg::orient::atlbin}
  Классов эквивалентности атласов для связного многообразия~--- два.
\end{prop}
\begin{lproof}[\quest]
  Пусть какая-нибудь точка $M$ содержится в пересечении двух карт из разных атласов.

  Пусть в её окрестности репараметризация между атласами происходит с положительным якобианом.
  До любой другой точки можно добраться по цепочке карт из одного атласа (из линейной связности).

  Так что в её окрестности переход между атласами происходит с тем же знаком, что и в окрестности
  исходной точки. От выбора карт по дороге ничего не зависит, так как они из одного атласа.
\end{lproof}

\begin{defn}\label{defn:dg::orient::edge}
  Пусть на $M$ задан ориентирующий атлас. 
  Тогда сужение этого атласа на край задаёт ориентацию края.
\end{defn}

А теперь минутка конкретики. 
\begin{defn}\label{defn:dg::orient::surf}
  Поверхность (регулярная)~--- связное \quest подмногообразие $\R^3$ с рангом карт $2$.
\end{defn}

\begin{prop}\label{prop:dg::orient::surfnorm}
  Ориентация на поверхности задаётся непрерывным векторным полем нормалей.
  <<Сторона>> поверхности задаётся им же.
  \[
    n= \frac{r_u \times r_v }{|r_u \times r_v|} 
  \]
\end{prop}
\begin{lproof}
  Связка бананов. Бананы тут ни при чём, но они кончились.
\end{lproof}

\begin{rem}\label{rem:dg::orient::curvesfolds}
  С кривыми наверное тоже стоит иметь дело, как с многообразиями, но вот тут \quest.
  Дальше я так буду делать, но не очень законно.
\end{rem}

\paragraph{Интеграл второго рода}
\label{par:dg::secint}

Здесь всюды $\del s$~--- мера на многообразии.
\begin{defn}\label{defn:dg::secint::curv}
  Интеграл второго рода по кривой $\Gamma$  от векторного поля $F$ определяется, как 
  \[
  \int_\Gamma  \langle F ,\tau \rangle \, \del s
  \]
\end{defn}

\begin{defn}\label{defn:dg::secint::curv}
  Интеграл второго рода по поверхности $M$  от векторного поля $F$ определяется, как 
  \[
  \int_\Gamma  \langle F ,n \rangle \, \del s
  \]
\end{defn}

\begin{defn}[Касательное пространство в точке $x$]\label{defn:dg::AAAA::tangentbndl}
  Пусть $M$~--- гладкое многообразие. Допустим, $\varphi_i$~--- карта в $V(x)$.
  Тогда
  \[
    T_x M = \Bigl(\del \varphi_i (x)\Bigr)(\R^k)
  \]
  Кокасательное пространство~--- сопряжённое к нему. Собственно, пространство линейных форм,
  действующих из $T_xM$.
\end{defn}

\begin{defn}\label{defn:dg::secint::difform}
  Дифференциальная форма $p$-го порядка на многообразии $M$ в точке $x$~--- 
  кососимметрическая линейная функция 
  \[
    \omega^p \colon \underbrace{T_x M \times \dotsm \times T_x M}_p  \to \R \in (T^*_x M)^p
  \]
\end{defn}
Умножение векторных пространств тут на самом деле тензорное, как я понял, так что
очевидно следущее
\begin{stat}\label{stat:dg::secint::difformbasis}
  $\omega^p$ разложится по базису $\bigwedge_{i_k} \del x^{i_k} \in (T^*_x M)^p$
\end{stat}
А ещё $(T_xM)^p$ надо бы обозначать как-то так, подчёркивая, что это внешняя степень: 
$\Lambda^p (T_xM)$

\begin{exmp}\label{exmp:dg::secint::difformbasis}
  Поскольку эта ерунда косокоммутативна, надо думать что засунуть в базис.
  Вот давайте все для $\R^3$ напишем.
  \begin{align*}
    \omega^1 &= a_x\,\del x + a_y\,\del y + a_z\,\del z \\
    \omega^2 &= a_{yz}\,\del y\wedge\del z+a_{zx}\,\del z \wedge\del xa_{xy}\,\del x\wedge\del y\\
    \omega^3 &= a_{xyz}\,\del x \wedge\del y\wedge z
  \end{align*}
\end{exmp}

\underdev\plholdev{понять меры Хаара. Когда-нибудь...}

Положим, все формы имеют гладкие коэффициенты.
Тогда пока интеграл от гладкой дифференциальной формы на многообразии определим так:
\begin{defn}\label{defn:dg::secint::difformint}
  Пусть $M$~--- простое $n$-мерное многообразие (покрывается одной картой $f\colon D \to M$),
  $u\in D$,
  а $\omega^n$~--- дифференциальная форма с коэффициентами $a_{i_1, \dotsc, i_n}(x)$.
  Давайте её поподробней напишем
  \[
    \omega =\sum_{i_{1}<\cdots <i_{n}}
    a_{i_1,\dotsc ,i_n} (x) \,\del x^{i_1} \wedge \dotsm \wedge \del x^{i_n}
  \]
  Тогда можно написать такое определение:
  \[ 
    \int_M \omega^n 
    := \int_D a_{i_1,\dotsc ,i_n}\bigl(x\bigr) 
    \, \bigwedge_{j=1}^n \del x^{i_j} 
    := \int_D a_{i_1,\dotsc ,i_n}\bigl(f(u)\bigr) 
    \, \pder{x^{i_1 \intrng i_k}}{u^{i_1 \intrng i_k }} \, \del \lambda_n(u) 
  \]
  Здесь на самом деле обычный интеграл Римана, все функции под интегралом непрерывны.
\end{defn}
\begin{rem}\label{rem:dg::secint::noord}
  Здесь нужно и можно вспомнить, что в интеграле 1 рода был 
  $\sqrt{g(u)} = \left|(\pder{x}{u})^T\pder{x}{u}\right|$. Те есть, корень из суммы квадратов
  тех миноров, что здесь.
\end{rem}
Общее определение требует понимания разбиения единицы, а я пока так не умею.

Теперь минутка конкретики

\begin{prop}\label{prop:dg::secint::diffrmline}
  Пусть $F=(P,Q,R)$, $\omega_F^1 = P\, \del x + Q \, \del y + R\, \del y$.
  Положим, $G$~--- кривая (одномерное многообразие).
  Тогда
  \[
    \int_ \Gamma \langle F,\tau \rangle\, \del s = \int_\Gamma \omega_F^1
  \]
\end{prop}
\begin{lproof}
  Заметим, что $\del s = |r'|\, \del t$, тогда $\tau \,\del s = (\del x, \del y, \del z)$.
  Кажется, всё.
\end{lproof}

\begin{prop}\label{prop:dg::secint::prec}
  Пусть $\omega_F^1$ точна, то есть $\omega = \del \Phi$. Тогда \[
    \int_\Gamma \omega_F^1 = \Phi(B) - \Phi (A).
  \]
\end{prop}

Физический смысл этого дела~--- работа.

\def\wdgcoef#1#2#3{#1\, \del #2 \wedge \del #3}
\begin{prop}\label{prop:dg::secint::surfdiffform}
  Пусть $M$~--- 2-мерная гадкая ориентируемая поверхность, $F=(P,Q,R)$,
  $\omega_F^2 = \wdgcoef{P}{y}{z} + \wdgcoef{Q}{z}{x} + \wdgcoef{R}{x}{y} $. Тогда \[
    \int_M \omega_F^2 = \int_M  \langle F,n  \rangle\, \del s
  \]
\end{prop}
\begin{lproof}
  \def\arraystretch{1.5}
  Пусть $N = (A,B,C)$. $\del S$ можно расписать получше.
  \[
    L = \pder{r}{(u,v)} = 
    \begin{pmatrix}
      \pder{x}{u} & \pder{x}{v} \\
      \pder{y}{u} & \pder{y}{v} \\
      \pder{z}{u} & \pder{z}{v} 
    \end{pmatrix}
  \]
  При умножении на транспонированную воспольземся известной формулой с суммой миноров:
  \[
    g = L^T L = I_1^2 + I_2^2 + I_3^2 = A^2 + B^2 + C^2 \Rightarrow \del S = \sqrt{g} = |N|
  \]

  Тогда $F n\, \del S = (PA+QB+RC) \, \del u$. А теперь смотрим на определение \ref{defn:dg::secint::difformint} 
  и понимаем что там ровно то же самое.
\end{lproof}


\paragraph{Дифференцирование векторных полей}
\label{par:dg::vecfione}

по методичке Лодкина

\paragraph{Формула Грина}
\label{par:dg::green}

\begin{thrm}\label{thrm:dg::green}
  Пусть $D$~--- связное двумерное ориентируемое гладкое подмногообразие $\R^2$ с краем,
  $\omega = P \, \del x + Q\, \del y$~--- гладкая дифференциальная форма.
  Тогда
  \[
    \int_{\partial D} \omega = \iint _D \left(\pder{Q}{x} - \pder{P}{y}\right) \, \del x\wedge \del y
  \]
\end{thrm}

\begin{tproof}
  Много пунктов. Сначала разбить на области типа $y$ (с вертикальными краями). 
  И ещё занулить $P$, например.  А потом складывать разные случаи, пользуясь тем, что
  по вертикальным краям интеграл обратиться в ноль.
  
  А потом сложить это с областями типа $x$.
\end{tproof}

\paragraph{Классическая формула Стокса}
\label{par:dg::stocks}

\begin{thrm}\label{thrm:dg::stocks}
  Пусть $M$~--- компактная ориентируемая поверхность в $\R^3$ с краем, $F$~--- гладкое векторное
  поле.
  Тогда
  \[
    \iint_M \langle \rot F , n\rangle \, \del S = \oint_{\partial M} \langle F,\tau \rangle\, \del s
  \]
\end{thrm}

\paragraph{Формула Гаусса-Остроградского}
\label{par:dg::gaussost}


\begin{thrm}\label{thrm:dg::gaussost}
  Пусть $V$~--- компактное тело в $\R^3$ с гладкой границей (гладким подмногообразием $\R^3$).
  Нормаль выберем <<наружу>>.  Тогда
  \[
    \oiint_M \langle F,n \rangle\, \del S = \iiint_V \div F\, \del V
  \]
\end{thrm}

\paragraph{Физический смысл дивергенции и ротора}
\label{par:dg::physdiv}

Дивергенция~--- удельный (по объему) поток через через бесконечно малую поверхность.

\end{document}
% vim: tw=100 cc=100

