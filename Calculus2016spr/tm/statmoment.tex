\documentclass{article}
\usepackage{amsmath,amssymb,theorem}

%%%%%%%%%% Start TeXmacs macros
\usepackage[utf8]{inputenc}
\usepackage[russian]{babel}
\newcommand{\emdash}{---}
\newcommand{\nocomma}{}
{\theorembodyfont{\rmfamily}\newtheorem{example}{\CYRP\cyrr\cyri\cyrm\cyre\cyrr}}
%%%%%%%%%% End TeXmacs macros

\begin{document}

\begin{example}
  Статический момент плоской фигуры
  относительно оси.
  
  Докажем, что
  \[ N = \int_{x_1}^{x_2} \sigma xa \left( x \right) d \nocomma x \]
  где $a \left( x \right)$~{\emdash} длина ``сечения''
  фигуры, $\sigma$~{\emdash} поверхностная
  плотность
  
  Пусть $\Delta = \left[ \alpha, \beta \right] \subset \left[ x_1, x_2
  \right]$~{\emdash} промежуток на оси $x$, $\Phi \left( \Delta
  \right)$~{\emdash} момент такой ``полоски''.
  
  Будем считать статический момент
  аддитивным по определению. Ещё мы умеем
  считать момент точки: он равен $m_i x_i$.
  Чтобы воспользоваться тестом \ref{denstest3}
  докажем, что (то, что они все
  положительные, очевидно)
  \[ \left( \min_{\Delta} a \left( x \right) \right)  \left( \min_{\Delta} x
     \sigma \right)  \left| \Delta \right| \leqslant \Phi \left( \Delta
     \right) \leqslant \left( \max_{\Delta} a \left( x \right) \right)  \left(
     \max_{\Delta} x \sigma \right) \left| \Delta \right| \]
  \begin{eqnarray*}
    &  & \left( \min_{\Delta} a \left( x \right) \right)  \left| \Delta
    \right| = \left| \Delta \right| a_{\min} = S_1 \text{~ \emdash
    вписаннаяплощадьполоски}\\
    &  & \left( \max_{\Delta} a \left( x \right) \right)  \left| \Delta
    \right| = \left| \Delta \right| a_{\max} = S_2 \text{~ \emdash
    описаннаяплощадьполоски}\\
    &  & \left( \min_{\Delta} x \right) = \alpha \hspace{1em}  \left(
    \max_{\Delta} x \right) = \beta
  \end{eqnarray*}
  
  
  Тогда нижний предел~{\emdash} это если бы мы
  сгребли всю массу с $S_1$ и поместили в
  ближний к оси край и посчитали момент
  всего этого. Видно, что момент полоски на
  самом деле больше: и масса оценена снизу,
  и есть хотя бы одна точка с ненулевой
  массой дальше от оси чем $\alpha$.
  Аналогичные рассуждения применимы про
  оценку сверху.
  
  Все условия теста \ref{denstest3} выполнены,
  значит
  \[ N_{\Delta} = \Phi \left( \Delta \right) = \int_{\Delta} a \left( x
     \right) x \sigma d \nocomma x \]
\end{example}

\begin{example}
  Работа, которую нужно затратить на
  возведение пирамиды.
  
  Пусть $\Delta = \left[ \alpha, \beta \right] \subset \left[ 0 ; H
  \right]${\emdash}промежуток на оси высот, $\Phi \left(
  \Delta \right) ${\emdash} работа которую нужно
  затратить чтобы поднять слой толщины $\left|
  \Delta \right|$ на нужную высоту, $a \left( h \right)$
  {\emdash} сторона пирамиды в зависимости от
  высоты (она правильная и с квадратом в
  основании).
  
  Чтобы получить функцию плотности,
  посмотрим сначала, что происходит с
  блоком в форме параллелепипеда со
  строной основания $a$. Если поднять его на
  высоту $h$ то работа, затраченна на это
  {\emdash} $mgh = \rho a \left( h \right)^2 hg$.
  
  Теперь, давайте докажем, что
  \[ \Phi \left( \Delta \right) = \int_{\Delta} a \left( h \right)^2 hg \, d
     \nocomma h \]
  
  
  Будем пользоваться условием теста
  \ref{denstest3}
  \[ \left( \min_{\Delta} a \left( x \right)^2 \right)  \left( \min_{\Delta}
     x \rho \right)  \left| \Delta \right| \leqslant \Phi \left( \Delta
     \right) \leqslant \left( \max_{\Delta} a \left( x \right)^2 \right) 
     \left( \max_{\Delta} x \rho \right) \left| \Delta \right| \]
  
  
  Как видно, от предыдущего примера
  отличается только степенью при $a \left( x
  \right)$. Доказательство здесь почти такое
  же.
  
  У нормальной пирамиды $a \left( h \right) = A \frac{H -
  h}{H}$. Тогда
  \[ A = \int_0^H A^2  \left( 1 - \frac{h}{H} \right)^2 h \rho \, d \, x =
     \rho \frac{A^2}{H^2}  \left( \frac{H^4}{2} - \frac{2 H^4}{3} +
     \frac{H^4}{4} \right) = \frac{1}{12} \rho A^2 H^2 \]
  
\end{example}



\end{document}
